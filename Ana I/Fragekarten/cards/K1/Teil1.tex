\card{Was ist eine Aussage}{1}{
	Eine Aussage ist etwas, dem der Wahrheitsgehalt \gqq{wahr}\ oder \gqq{falsch}\ zugeordnet ist.
}
\card{Was ist eine Aussageform}{1}{
	Eine Aussageform ist eine Aussage, die eine noch unbestimmte oder freie Variable enthält.
}
\card{Wie ist die Negation einer Aussage definiert}{3}{
	Ist $p$ eine Aussage, so bezeichnet $\neg p$ die Negation dieser Aussage.
}
\card{Was ist die Konjunktion von Aussagen}{5}{
	Die Konjunktion entspricht $p\und q$ (\gqq{$p$ und $q$}) mit der Wahrheitstabelle
	\begin{center}
		\pundq
	\end{center}
}
\card{Was ist die Disjunktion von Aussagen}{6}{
	Die Disjunktion entspricht $p\oder q$ (\gqq{$p$ oder $q$}) mit der Wahrheitstabelle
	\begin{center}
		\poderq
	\end{center}
}
\card{Was ist die Kontravalenz von Aussagen}{7}{
	Die Kontravalenz entspricht $p\overset{.}{\oder}q$ (\textpeoderq) mit der Wahrheitstabelle
	\begin{center}
		\peoderq
	\end{center}
}
\card{Was ist die Implikation von Aussagen}{8}{
	Die Implikation entspricht $p\Rightarrow q$ (\textpimplq) mit der Wahrheitstabelle
	\begin{center}
		\pimplq
	\end{center}
}
\card{Wie wird $p$ bei der Implikation $p\Rightarrow q$ bezeichnet}{8}{
	Voraussetzung, Prämisse oder hinreichende Bedingung für $q$.
}
\card{Wie wird $q$ bei der Implikation $p\Rightarrow q$ bezeichnet}{8}{
	Behauptung, Konklusion oder notwendige Bedingung.
}
\card{Wie wird die \gqq{$p$ und $q$ sind äquivalent} definiert}{10}{
	Die \gqq{genau dann, wenn}-Aussage wird durch die folgende Wahrheitstabelle definiert
	\begin{center}
		\wtab(\Leftrightarrow;w;f;f;w)
	\end{center}
}
\card{Wie ist die Äquivalenz zwischen einer Menge von Aussagen definiert}{10}{
	Eine Menge von Aussagen sind äquivalent, falls für je zwei der Aussagen, $p$ und $q$ die Äquivalenz $p\Leftrightarrow q$ gilt.
}