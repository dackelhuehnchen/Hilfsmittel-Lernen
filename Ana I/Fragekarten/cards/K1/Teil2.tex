\card{Wie ist die naive Definition einer Menge}{12}{
	Eine Menge ist eine Zusammenfassung von Objekten, Elemente genannt. Ist $A$ eine Menge und $x$ ein Objekt, so schreiben wir $x\in A$, falls $x$ ein Element von $A$ ist. $x\notin A:\Leftrightarrow\neg(x\in\ A)$.
}
\card{Wie ist die Teilmenge $B$ einer Menge $A$ definiert}{13}{
	$B$ ist eine Teilmenge von $A$, $B\subseteq A, B\subset A$, falls für alle $x\in B$ auch $x\in A$ gilt.
}
\card{Was besagt das Extensionalitätsaxiom für Mengen}{13}{
	Es besagt, dass zwei Mengen $A,B$ gleich sind ($A=B$), falls $A\subset B$ und $B\subset A$ gilt.
}
\card{Wie ist die echte Teilmenge definiert}{13}{
	Die echte Teilmenge bedeutet, dass $A\subset B$ und $A\neq B$ gilt. Man schreibt $A\subsetneq B$
}