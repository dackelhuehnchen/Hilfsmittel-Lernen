\card{Wann sind zwei Mengen $A$ und $B$ gleich mächtig}{84}{
	Zwei Mengen $A$ und $B$ heißen gleich mächtig, $A\sim B$, falls es eine Bijektion $f :A\rightarrow B$ gibt.
}
\card{Wann ist eine Menge $A$ mächtiger als $B$}{84}{
	$A$ ist mächtiger als $B$, $A\succ B$, falls es eine injektive Abbildung $f:B\rightarrow A$ gibt.
}
\card{Wann ist eine Menge $A$ abzählbar}{84}{
	$A$ ist abzählbar, falls $A\sim\N$ gilt.
}
\card{Wann ist eine Menge $A$ höchstens abzählbar}{84}{
	$A$ ist höchstens abzählbar, falls $A\prec\N$ gilt.
}
\card{Wann ist eine Menge $A$ überabzählbar}{84}{
	$A$ ist überabzählbar, falls $A$ nicht höchstens\\abzählbar ist.
}
\card{Was ist eine Abzählung einer Menge $A$}{84}{
	Die Folge $\folge(x;i;\N)$ ist eine Abzählung von $A$, falls $\mathbin{\dot{\bigcup\limits_{i\in\N}}}\{x_i\}=A$ gilt.
}
\card{Wann ist eine Menge $A$ endlich}{89}{
	Eine Menge $A$ ist endlich, wenn es eine injektive Abbildung $f:A\rightarrow \N$ und ein $m\in\N$ mit $f(a)<m,~\forall a\in A$ gibt.
}
\card{Wann ist eine Menge $A$ unendlich}{89}{
	Eine Menge $A$ ist unendlich, wenn sie nicht endlich ist.
}
\card{Was ist die Kardinalität einer Menge}{89}{
	Die Kardinalität bezeichnet die Anzahl der Elemente in $A$. Ist $A$ unendlich, so gilt $|A|=\infty$.
}
\card{Wann gilt eine Aussageform $P$ auf $A$ für fast alle $i\in A$}{89}{
	Die Aussageform gilt für fast alle $i\in A$, falls die Menge $\{i\in A:\neg P(i)\}$ endlich ist.
}