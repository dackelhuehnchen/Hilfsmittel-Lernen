\card{Was ist eine Metrik}{1}{
	Eine Funktion $d:E\times E\rightarrow \R$ auf einer Menge $E$ heißt Metrik, falls
		\begin{enumerate}\itemsep=-1pt
			\item $d(x,y)=d(y,x)$, $d$ ist symmetrisch
			\item $d(x,y)=0\Longleftrightarrow x=y$, $d$ ist positiv definit
			\item $d(x,z)\leq d(x,y)+d(y,z)$, $d$ erfüllt die\\Dreiecksungleichung
		\end{enumerate}
	$(E,d)$ ist ein metrischer Raum.
}
\card{Was ist eine Norm}{4}{
	Eine Funktion $\norminit : E\Rightarrow \R_+$ auf einem $\K$-Vektorraum $E$ heißt Norm, falls
		\begin{enumerate}\itemsep=-1pt
			\item $\norm(x)=0\Longrightarrow x=0$, $\norminit$ ist positiv definit
			\item $\norm(\lambda x)=|\lambda|\cdot \norm(x)$, $\norminit$ ist homogen
			\item $\norm(x+y)\leq \norm(x)+\norm(y)$, $\norminit$ erfüllt die\\Dreiecksungleichung
		\end{enumerate}
	$(E,\norminit)$ heißt normierter Raum.
}
\card{Was ist ein Skalarprodukt}{6}{
	Eine Funktion $\norminit : E\Rightarrow \R_+$ auf einem $\K$-Vektorraum $E$ heißt Norm, falls\\
	\vspace*{-1.5\baselineskip}
		\begin{enumerate}\itemsep=-1pt
			\item $\skal(\lambda x+y,z)=\lambda \skal(x,z)+\skal(y,z)$,\\
				$\skalinit$ ist linear im ersten Argument
			\item $\skal(x,y)=\skal(y,x)$,\\
				$\skalinit$ ist symmetrisch ($\R$) bzw. hermizit ($\C$)
			\item $\skal(x,x)\geq 0$ und $\skal(x,x)=0\Leftrightarrow x=0$,\\
				$\skalinit$ ist positiv definit
		\end{enumerate}
		\vspace*{-0.5\baselineskip}
	$(E,\skalinit)$ heißt Skalarproduktraum.
	\vspace*{-1cm}
}