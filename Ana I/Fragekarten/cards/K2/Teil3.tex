\card{Was ist eine Reihe}{52}{
	Zwei Folgen $\folge(a;n;\N)$ und $\folge(s;n;\N)$ heißen Reihe, falls $s_n:=\sum\limits_{i=0}^n a_i$ gilt.
}
\card{Wie ist der Grenzwert einer Reihe definiert}{52}{
	$\lim\limits_{n\rightarrow\infty}s_n=\sum a_n=\sum\limits_{i=0}^{\infty}$
}
\card{Was ist eine Nullfolge}{55}{
	Eine Folge, die gegen $0$ konvergiert.
}
\card{Wann ist eine Reihe in einem Banachraum absolut konvergent}{69}{
	Die Reihe $\reihe(a;n;\N)$ heißt absolut konvergent, falls $((\norm(a_n)))_{n\in\N}$ konvergiert.
}
\card{Wann ist eine Reihe in einem Banachraum bedingt konvergent}{69}{
	Wenn sie konvergent aber nicht absolut konvergent ist.
}
\card{Was ist der Konvergenzradius einer Reihe}{72}{
	$\dfrac{1}{\limsup\limits_{n\rightarrow \infty}|a_n|^{\frac{1}{n}}}$
}
\card{Was ist der Raum aller quadratsummierbaren Folgen}{76}{
	$\ell^2(\N)=\left\{\folge(a;n;\N):\sum\limits_{n=0}^{\infty}|a_n|^2<\infty \right\}$\\
	Es gilt für $a,b\in\ell^2(\N)$:\\
	$\skal(a,b)=\sum\limits_{n=0}^{\infty}a_nb_n$
}
\card{Was ist die $\ell^p(\N)$-Norm}{76}{
	$\norm(a)_{\ell^p(\N)}=\left(\sum\limits_{n=0}^{\infty}|a_n|^p\right)^{\frac{1}{p}}$
}
\card{Was ist eine Umordnung einer Reihe}{78}{
	Seien $\reihe(a;n;\N),\reihe(b;n;\N)$ Reihen in einem normierten Raum $E$. Dann ist $\reihe(b;n;\N)$ eine Umordnung von $\reihe(a;n;\N)$, falls es eine Bijektion\\
	$\varphi:\N\rightarrow \N$ mit $b_n=a_{\varphi(n)}$ gibt.
}
\card{Wann ist eine Familie $\folge(a;i;\mathcal{I})$, $\mathcal{I}$ ist abzählbar, in einem Banachraum $E$ absolut summierbar}{80}{
	Falls es eine Bijektion $\varphi : \N\rightarrow\mathcal{I}$ gibt, sodass $((a_{\varphi(n)}))_{n\in\N}$ absolut konvergiert. 
}