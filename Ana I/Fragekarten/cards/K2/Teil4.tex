\card{Wann konvergiert eine Folge von Funktionen eines vollständigen metrischen Raumes gleichmäßig}{88}{
	Sei $\varepsilon>0$. $\folge(f;n;\N)$ konvergiert gleichmäßig, falls
		\[\underset{\varepsilon>0}{\forall}~\underset{n_0\in\N}{\exists}~\underset{x\in E}{\forall}~\underset{n,m\geq n_0}{\forall}~d(f_n(x),f_m(x))<\varepsilon\]
}
\card{Wann konvergiert eine Folge von Funktionen eines vollständigen metrischen Raumes lokal gleichmäßig}{88}{
	Sei $f_n:E\rightarrow F$. Die Folge $\folge(f;n;\N)$ konvergiert lokal gleichmäßig, falls es zu jedem $x\in E$ ein $\delta>0$ gibt, sodass $f_n|_{B_{\delta(x)}}:B_{\delta(x)} \rightarrow F$ gleichmäßig konvergiert.
}
\card{Wann ist $\reihe(f;n;\N),~f_n: E\rightarrow F$ mit $E$ metr. Raum, $F$ BR, gleichmäßig, absolut, lokal glm. konvergent}{88}{
	Falls dies für die Folge der Partialsummen gilt.
}
\card{Was ist eine Doppelfolge}{92}{
	Eine Funktionsfolge $f_n : \N\rightarrow E : a_{nm}=f_n(m)$.
}