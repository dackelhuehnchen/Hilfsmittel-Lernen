\cardP{Was besagt die Russelsche Antinomie}{Bem}{18}{
	Es gibt einen Widerspruch, wenn man beim Aussonderungsaxiom die Allmenge einsetzt.
}
\cardP{Was besagen die Peanoaxiome}{L}{60}{
	\begin{enumerate}\itemsep=-1pt
		\item $0\in\N$
		\item jedes $a\in\N$ besitzt genau einen Nachfolger $a^{+}\in\N$
		\item $0$ ist kein Nachfolger eine natürlichen Zahl
		\item Für alle $n,m\in\N$ gilt $m^{+}=n^{+}\Rightarrow n=m$
		\item Sei $X\subset\R$ beliebig mit $0\in X, n^{+}\in X,~\forall n\in X$. Dann gilt $\N\subset X$.
	\end{enumerate}
	\vspace*{-1cm}
}
\cardP{Was besagt das Zornsche Lemma}{L}{76}{
	Sei $M\neq \emptyset$ mit einer Teilordnung (= partielle Ordnung) $\leq$. Nehme an, jede total geordnete Teilmenge $\Lambda\subset M$ (= Kette) besitzt eine obere Schranke $b\in M$, d.h. $x\leq b,\forall x\in\Lambda$. Dann enthält $M$ ein maximales Element $x_0$, d.h. $\exists x_0\in M : x\geq x_0 \Rightarrow x= x_0$.
}
\cardP{Was besagt das Theorem von Schröder und Bernstein}{T}{86}{
	Aus $A\prec B$ und $B\prec A$ folgt $A\sim B$.
}
\cardP{Was sagt das Cantorsche Theorem}{T}{101}{
	Für eine Menge $A$ gilt $\mathcal{P}(A)\succ A$ und $\mathcal{P}(A)\not\sim A$.
}
%Ohne Namen
\cardP{Was bedeutet es, dass $\R$ archimedisch ist}{T}{61}{
	Zu jeden $x\in \R$ gibt es ein $n_0\in\N$, sodass für alle $n\in \N$ mit $n\geq n_0$ auch $n\geq x$ gilt.
}
\cardP{Was gilt für das kartesische Produkt einer Familie von Mengen $\folge(A;i;\mathcal{I})$, bei der keine Menge leer ist}{A}{74}{
	$\prod\limits_{i\in\mathcal{I}}A_i\neq \emptyset$, d.h. es gibt eine Familie $\folge(x;i;\mathcal{I})$ mit $x_i\in A_i$, für alle $i\in\mathcal{I}$.
}
\cardP{Was gilt für das kartesische Produkt einer Familie von Mengen $\folge(A;i;\mathcal{I})$, für die gilt $\prod\limits_{i\in\mathcal{I}}A_i= \emptyset$}{P}{75}{
	Mindestens eine Menge $A_i$ ist leer.
}
\cardP{Was bilden die Restklassen einer Äquivalenzrelation auf $A$}{P}{78}{
	Eine Partition von $A$.
}
\cardP{Wie kann man aus einer Partition $\folge(A;i;\mathcal{I})$ einer Menge $A$ eine Äquivalenzrelation bilden}{P}{78}{
	$x\sim y:\Leftrightarrow \exists i\in\mathcal{I}: x,y\in A_i$ definiert eine Äquivalenzrelation auf $A$.
}
\cardP{Was gilt für die Komposition der Abbildungen $\varphi:A\rightarrow B,~\psi:B\rightarrow C$ auf Mengen}{P}{87}{
	Ist $\psi\circ \varphi$ injektiv, so ist $\varphi$ injektiv.\\
	Ist $\psi\circ \varphi$ surjektiv, so ist $\psi$ surjektiv.
}
\cardP{Was gilt für eine Abbildung $f:A\rightarrow B$ auf den Mengen $A$, $B$}{P}{87}{
	$f$ ist surjektiv, genau dann, wenn es eine Funktion $g:B\rightarrow A$ gibt, sodass $f\circ g=id_B$.\\
	$f$ ist injektiv, genau dann, wenn es eine Funktion $g:B\rightarrow A$ gibt, sodass $g\circ f=id_A$.\\
}
\cardP{Wann gilt $A\prec B$}{K}{88}{
	Genau dann, wenn es eine surjektive Funktion $f:B\rightarrow A$ gibt.
}
\cardP{Was gilt für die Teilmengen einer unendlichen menge $A$}{L}{94}{
	Es gibt mindestens eine abzählbare Teilmenge.
}
\cardP{Wann ist eine Menge $A$ höchstens abzählbar}{L}{95/96}{
	Genau dann, wenn $A$ endlich, $A\sim \N$ oder wenn es eine surjektive Abbildung $f:\N\rightarrow A$ gibt.
}
\cardP{Was gilt für die Vereinigung von abzählbaren Mengen}{L}{99}{
	Sie ist auch abzählbar.
}
\cardP{Wann ist eine Menge $A\subset\R$ beschränkt}{K}{105}{
	Genau dann, wenn es ein $a\in\R$ mit $|x|\leq a$, für alle $x\in A$ gibt.
}