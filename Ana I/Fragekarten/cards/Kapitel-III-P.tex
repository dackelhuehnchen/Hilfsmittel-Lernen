\cardP{Was ist die Komposition von Abbildungen}{T}{27}{
	Sei $f:E\rightarrow F$ in $x_0$ stetig, $g:F\rightarrow G$ in $f(x_0=)$ stetig, so ist $g\circ f$ in $x_0$ stetig.
}
%ohne Namen
\cardP{Was gilt für die Menge der inneren Punkte für zwei Mengen $A\subset B\subset E$}{P}{10}{
	\ \\
	$\interior (A)\subset\interior(B)\\
	\interior(A\cap B)=\interior(A)\cap\interior(B)$
}
\cardP{Was gilt für die abgeschlossenen Mengen für zwei Mengen $A\subset B\subset E$}{P}{10}{
	\ \\
	$\overline{A}\subset\overline{B}\\
	\\overline{A}=\bigcap\{F\subset\mathcal{F}:A\subset F\}$\\
	Es gilt somit, dass $\overline{A}$ die kleinste abgeschlossene Menge ist, die $A$ enthält.
}
\cardP{Wann ist dei Teilmenge $U\subset A\subset E$ eine Umgebung von $x\in A$ bezüglich der auf $A$ induzierten Metrik}{K}{20}{
	Wenn es ein $V\in\mathcal{U}(x)$ ($\mathcal{U}$ bezüglich $E$) mit $v\cap A=U$ gibt.
}
\cardP{Wenn $E,F$ metrische Räume, $f:E\rightarrow F$ eine Abbildung ist, welche Aussagen über Stetigkeit sind dann äquivalent}{T}{27}{
	\begin{enumerate}\itemsep=-1pt
		\item $f$ ist in $x_0$ $\varepsilon-\delta$-stetig
		\item $f$ ist in $x_0$ als topologische Abbildung stetig
		\item $f$ ist in $x_0$ als folgenstetig
	\end{enumerate}
}
\cardP{Was gilt für die Komposition von zwei gleichmäßig stetigen Abbildungen $f:X\rightarrow Y,~g:Y\rightarrow Z$}{P}{34}{
	$g\circ f$ ist gleichmäßig stetig.
}