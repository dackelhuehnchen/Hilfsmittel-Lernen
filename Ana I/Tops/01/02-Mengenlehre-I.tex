\teil{Erste Mengenlehre}{-1.45}
\defi[naive Definition einer Menge]{-1.45}{
	Eine Menge ist eine Zusammenfassung von Objekten, Elemente genannt. Ist $A$ eine Menge, $x$ ein Objekt, so schreiben wir $x\in A$, falls $x$ ein Element von $A$ ist. $x\notin A:\Leftrightarrow \neg (x\in A)$\\
	Für eine Menge $A$, die genau die Elemente $a,b$ und $c$ enthält, schreiben wir $A=\{a,b,c\}$. Es ist irrelevant, ob $a$ mehrfach auftaucht oder wie die Elemente angeordnet werden.
}
\defi{-1.45}{
	Seien $A,B$ Mengen.
	\vspace*{-0.25cm}
	\begin{comenum}{(\roman*)}
		\item Dann ist $A$ eine Teilmenge von $B$ ($A\subset B$ oder $A\subseteq B$), falls aus $x\in A$ auch $x\in B$ folgt.
		\item $A$ und $B$ heißen gleich ($A=B$), falls $A\subset B$ und $B\subset A$ gelten.\\
		$A\neq B :\Leftrightarrow \neg (A=B)$\hfill(Extensionalitätsaxiom)
		\item Schreibe $A\subsetneq B$ für $A\subset B$ und $A\neq B$.
	\end{comenum}
}
\topbreak
\lem{-1.45}{
	Seien $A,B,C$ Mengen. Dann gelten:
	\vspace*{-0.25cm}
		\begin{comenum}{(\roman*)}
			\item $A\subset A$\hfill(Reflexivität)
			\item $x\in A$ und $A\subset B$ implizieren $x\in B$
			\item $A\subset B\subset C\Rightarrow A\subset C$\hfill(Transitivität)
		\end{comenum}
}
\ax[Aussonderungsaxiom]{-1.45}{
	Sei $A$ eine Menge und $a(x)$ eine Aussageform. Dann gibt es eine Menge $B$, deren Elemente genau die $x\in A$ sind, die $a(x)$ erfüllen.\\
	Schreibe $B=\{x\in A : a(x)\}$.
}
\step
\bem{-1.45}{
	Zu jeder Menge $A$ gibt es eine Menge $B$ und eine Aussageform $a(x) : A=\{x\in B : a(x)\}$.\\
	Nehme $B=A, a(x)=(x\in A)$.
}
\bem[Russelsche Antinomie]{-1.45}{
	Nimmt man im Aussonderungsaxiom statt $A$ die \gqq{Allmenge} (Menge aller Elemente), dann bekommt man Probleme:\\
	Sei $A=$ Allmenge, $B=\{X\in A : X\notin X\}$. Es gilt $y\in B\Leftrightarrow (y\in A\und y\notin y)\Leftrightarrow y\notin y$.\\
	Gilt $B\in B$? $\rightarrow$ Widerspruch.
}
\lem[Existenz der leeren Menge]{-1.45}{
	Es gibt eine Menge $\emptyset$, die leere Menge, die kein Element enthält. Sie erfüllt:
	\vspace*{-0.25cm}
		\begin{comenum}{(\roman*)}
			\item $\emptyset\subset A$ für alle Mengen $A$
			\item $\emptyset$ ist eindeutig bestimmt.
		\end{comenum}
}