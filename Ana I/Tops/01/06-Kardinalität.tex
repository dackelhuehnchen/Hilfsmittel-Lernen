\teil{Kardinalität}{-1.45}
\defi[Mächtigkeit]{-1.45}{
	Seien $A,B$ Mengen.
	\vspace*{-0.25cm}
	\begin{comenum}{(\roman{*})}
		\item $A,B$ heißen \textbf{\textit{gleich mächtig}} ($A\sim B$), falls es eine Bijektion $f : A\rightarrow B$ gibt.
		\item $B$ heißt \textbf{\textit{mächtiger}} als $A$ ($B\succ A$) oder $A$ \textbf{\textit{weniger mächtig}} als $B$ ($A\prec B$), falls es eine injektive Abbildung $f: A\rightarrow B$ gibt.
		\item $A$ heißt \textbf{\textit{abzählbar}}, falls $A\sim\N$.
		\item $A$ heißt \textbf{\textit{höchstens abzählbar}}, falls $A\prec \N$.
		\item $A$ heißt \textbf{\textit{überabzählbar}}, falls $A$ nicht höchstens abzählbar ist.
		\item Sei $A$ abzählbar, so heißt die Folge $\folge(x;i)$ eine \textbf{\textit{Abzählung}} von $A$, falls $x_i\neq x_j$ für $i\neq j$ und $\bigcup\limits_{i\in\N}\{x_i\}=A$.
	\end{comenum}
}
\topbreak
\bem{-1.45}{
	\begin{compactenum}
		\item $\sim$ ist Äquivalenzrelation
		\item $A\prec B\prec C \Rightarrow A\prec C$
		\item $A\prec A$
		\item $G : =\{2n : n\in\N\}, G\prec \N : 2n\mapsto 2n$ und $\N \prec G: n\mapsto 2n$. Bijektiv: $\N\sim G$
	\end{compactenum}
}
\theo[Schröder-Bernstein]{-1.45}{
	Aus $A\prec B$ und $B\prec A$ folgt $A\sim B$.
}
\prop{-1.45}{
	$A,B,C$ sind Mengen. Seien $\varphi : A\rightarrow B, \psi : B\rightarrow C$ Abbildungen. Sei $f: A\rightarrow B$ Abbildung. Dann gelten:
	\vspace*{-0.25cm}
	\begin{comenum}{(\roman{*})}
		\item Ist $\psi \circ\varphi$ injektiv, so ist $\varphi$ injektiv 
		\item Ist $\psi \circ\varphi$ surjektiv, so ist $\psi$ surjektiv
		\item $f$ surjektiv $\Leftrightarrow \exists g : B\rightarrow A, f\circ g = id_B$
		\item $f$ injektiv $\Leftrightarrow \exists g : B\rightarrow A, g\circ f = id_A$
	\end{comenum}
}
\kor{-1.45}{
	$A\prec B \Leftrightarrow \exists f : B\rightarrow A$, $f$ ist surjektiv.
}
\defi{-1.45}{
	Sei $A$ eine Menge.
	\vspace*{-0.25cm}
	\begin{comenum}{(\roman{*})}
		\item $A$ heißt \textbf{\textit{endlich}}, falls es eine injektive Abbildung $f : A\rightarrow \N$ und $m\in\N$ mit $f(a)ym,\forall a\in A$ gibt.
		\item $A$ heißt \textbf{\textit{unendlich}}, falls $A$ nicht endlich ist.
		\item Gibt es eine bijektive Abbildung $f:A\rightarrow\{0,1,\dots,m-1\}\subset \N$, so hat $A$ die \textbf{\textit{Kardinalität}} $m(|A|=m)$. Gibt es keine solche Abbildung, so gilt $|A|=\infty$.
		\item Sei $P$ eine Aussageform auf $A$. Dann gilt $P$ für \textbf{\textit{fast alle}} $i\in A$, falls $\{i\in A : \neg P(i)\}$ endlich ist.
	\end{comenum}
}
\step
\lem{-1.45}{
	\begin{compactenum}
		\item Für jede endliche Menge $A$ gilt $|A|<\infty$, d.h. es gibt ein $m\in\N$ und eine Bijektion $f : A\rightarrow\{0,\dots,m-1\}$.
		\item Seien $m,n\in\N$ und $f:\{0,\dots,m\}\rightarrow \{0,\dots,n\}$ eine Bijektion. Dann gilt $n=m$. ($\Rightarrow$ Kardinalität ist wohldefiniert).
	\end{compactenum}
}
\lem{-1.45}{
	Sei $m\in\N\setminus\{0\}$ und $(a_i)_{1\leq i\leq m}$ eine endliche Familie natürlicher Zahlen (oder reeller). Dann gibt es ein $i\in\{a,\dots,m\} : a_i\leq a_j,\forall 1\leq j\leq m$.\\
	Schreibe $a_i=\min\{a_1,\dots,a_m\}\equiv \min(a_1,\dots,a_n)$.\\
	Entsprechend $\max\{a_1,\dots,a_m\}\equiv \max(a_1,\dots,a_n)$.
}
\lem{-1.45}{
	Die natürlichen Zahlen sind wohlgeordnet, d.h. jede Menge $M\subset\N, M\neq \emptyset$, besitzt ein kleinstes Element, d.h. $\exists a\in M : a\leq b,\forall b\in M$.
}
\lem{-1.45}{
	Sei $A$ eine unendliche Menge. Dann besitzt $A$ eine abzählbare Teilmenge.
}
\lem{-1.45}{
	Sei $A$ eine Menge. Dann ist $A$ genau dann höchstes abzählbar, wenn $A$ endlich ist oder $A\sim\N$.
}
\lem{-1.45}{
	Sei $A$ eine Menge. Dann ist $A$ genau dann höchstens abzählbar, wenn es eine surjektive Abbildung $f:\N\rightarrow A$ gibt.
}
\prop{-1.45}{
	$\N\times\N\sim\N$.
}
\topbreak
\prop{-1.45}{
	Sei $k\in\N_{\geq 0}$. Dann ist $\prod\limits_{i=1}^{k}\N = \N^k$ abzählbar. Dies gilt auch, wenn wir $\N$ überall durch $A\sim\N$ ersetzen.
}
\lem{-1.45}{
	Sei $\folge(A;i)$ eine Folge abzählbarer Mengen. Dann ist $A:=\bigcup\limits_{i\in\N}A_i$ abzählbar.
}
\bem{-1.45}{
	P. 1.98 und L. 1.99 gelten auch mit \gqq{höchstens abzählbar} statt abzählbar.
}
\theo[Cantor]{-1.45}{
	Sei $A$ eine Menge $\Rightarrow \mathcal{P}(A)\succ A$ und $\mathcal{P}(A)\not\sim A$.
}