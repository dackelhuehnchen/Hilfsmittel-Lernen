\teil{Kardinalität}{-1.45}
\defi[Mächtigkeit]{-1.45}{
	Seien $A,B$ Mengen.
	\vspace*{-0.25cm}
	\begin{comenum}{(\roman{*})}
		\item $A,B$ heißen \textbf{\textit{gleich mächtig}} ($A\sim B$), falls es eine Bijektion $f : A\rightarrow B$ gibt.
		\item $B$ heißt \textbf{\textit{mächtiger}} als $A$ ($B\succ A$) oder $A$ \textbf{\textit{weniger mächtig}} als $B$ ($A\prec B$), falls es eine injektive Abbildung $f: A\rightarrow B$ gibt.
		\item $A$ heißt \textbf{\textit{abzählbar}}, falls $A\sim\N$.
		\item $A$ heißt \textbf{\textit{höchstens abzählbar}}, falls $A\prec \N$.
		\item $A$ heißt \textbf{\textit{überabzählbar}}, falls $A$ nicht höchstens abzählbar ist.
		\item Sei $A$ abzählbar, so heißt die Folge $\folge(x;i)$ eine \textbf{\textit{Abzählung}} von $A$, falls $x_i\neq x_j$ für $i\neq j$ und $\bigcup\limits_{i\in\N}\{x_i\}=A$.
	\end{comenum}
}
\topbreak
\bem{-1.45}{
	\begin{compactenum}
		\item $\sim$ ist Äquivalenzrelation
		\item $A\prec B\prec C \Rightarrow A\prec C$
		\item $A\prec A$
		\item $G : =\{2n : n\in\N\}, G\prec \N : 2n\mapsto 2n$ und $\N \prec G: n\mapsto 2n$. Bijektiv: $\N\sim G$
	\end{compactenum}
}
\theo[Schröder-Bernstein]{-1.45}{
	Aus $A\prec B$ und $B\prec A$ folgt $A\sim B$.
}
\prop{-1.45}{
	$A,B,C$ sind Mengen. Seien $\varphi : A\rightarrow B, \psi : B\rightarrow C$ Abbildungen. Sei $f: A\rightarrow B$ Abbildung. Dann gelten:
	\vspace*{-0.25cm}
	\begin{comenum}{(\roman{*})}
		\item Ist $\psi \circ\varphi$ injektiv, so ist $\varphi$ injektiv 
		\item Ist $\psi \circ\varphi$ surjektiv, so ist $\psi$ surjektiv
		\item $f$ surjektiv $\Leftrightarrow \exists g : B\rightarrow A, f\circ g = id_B$
		\item $f$ injektiv $\Leftrightarrow \exists g : B\rightarrow A, g\circ f = id_A$
	\end{comenum}
}
\kor{-1.45}{
	$A\prec B \Leftrightarrow \exists f : B\rightarrow A$, $f$ ist surjektiv.
}