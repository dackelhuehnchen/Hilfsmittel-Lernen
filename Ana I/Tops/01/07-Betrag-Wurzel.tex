\teil{Betrag und Wurzel}{-1.45}
\defi{-1.45}{
	\begin{compactenum}
		\item Sei $x\in\R$. Definiere den \textbf{\textit{Betrag}} von $x$ wie folgt:
			$|x|:=\left\{ \begin{array}{lr}
				x,&x\geq 0\\
				-x,&x\leq 0
			\end{array}\right.$
		\item Ist $I\subset\R$ ein Intervall mit Endpunkten $a$ und $b$, so heißt $|a-b|$ \textbf{\textit{Länge von $I$}}.
	\end{compactenum}
}
\step
\prop{-1.45}{
	Seien $x,a\in\R$. Dann gelten
	\vspace*{-0.25cm}
	\begin{comenum}{(\roman{*})}
		\item $x\leq |x|$
		\item $|x|\leq a\Leftrightarrow -a\leq x\leq a$
		\item $|x| < a\Leftrightarrow -a<x<a$
	\end{comenum}
}
\kor{-1.45}{
	Sei $A\subset\R$. Dann ist $A$ genau dann beschränkt, wenn es ein $a\in\R$ mit $|x|\leq a,\forall x\in A$ gibt.
}
\theo[Dreiecksungleichung]{-1.45}{
	Seien $a,b\in\R$. Dann gilt
	\vspace*{-0.25cm}
	\begin{comenum}{(\roman{*})}
		\item $|a+b|\leq |a|+|b|$
		\item $|a-b|\geq |a|-|b|$
		\item $|a-b|\geq \left|~|a|-|b|~\right|$
	\end{comenum}
}
\prop[Existenz der $m$-ten Wurzel]{-1.45}{
	Seien $m\in\N\setminus\{0\},a\in\R_{geq 0}$. Dann gibt es \underline{genau} ein $x\in\R_{\geq 0} : x^m=a$.
}
\defi{-1.45}{
	\begin{compactenum}
		\item $\sqrt{a}$ ist die Zahl in $\R_{+}$ mit $\left(\sqrt{a}\right)^2=a$
		\item $\sqrt[m]{a}$ oder $a^{\frac{1}{m}}$ ist die Zahl in $\R_{+}$ mit $\left(\sqrt[m]{a}\right)^m=a$
		\item $a^0:=1, a^{\frac{n}{m}}:=\left(a^{\frac{1}{m}}\right)^n$
	\end{compactenum}
}