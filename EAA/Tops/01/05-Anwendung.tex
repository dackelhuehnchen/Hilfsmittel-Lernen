\subtop{Anwendung}\\\vspace*{-1.5\baselineskip}
\subsection{Matrix Multiplikation}
\begin{description}
	\item[Problem:] Multiplikation zweier $n\times n $ Matrizen
	\item[Eingabe:] Matrizen $A,B \in \mathbb{R}^{n\times n}$\\
		$\overbrace{\matriceDotsZeile{a}}^{A} \cdot \overbrace{\matriceDotsSpalte{b}}^{B} = \overbrace{\matriceDots{c}}^{C}$
	\item[Ausgabe:] Matrix $C$
	\item[Laufzeit:] $n^3+n^2(n-1) \in \Theta(n^3)$
	\item[Idee zur Verbesserung der Laufzeit 1 (Divide and Conquer):]\ \\\vspace*{-\baselineskip}
		\begin{enumerate}
			\item Aufteilung der Matrizen in 4 $\dfrac{n}{2}\times \dfrac{n}{2}$ Matrizen $\Rightarrow C_{ij} = A_{i1} \cdot B_{1j} + A_{i2} \cdot B_{2j}, 1\leq i,j\leq 2$
			\item Laufzeit:\\
			$T(n)= 8T\left(\half{n}\right)+4\cdot n^2\\
			\overset{\text{Master Theorem (1)}}{\Rightarrow} \Theta(n^3)$
		\end{enumerate}
\end{description}
\topbreak
\vspace*{-1\baselineskip}
\begin{description}
	\item[Idee zur Verbesserung der Laufzeit 2 (Strassen):]\ \\\vspace*{-\baselineskip}
		\begin{enumerate}
			\item Multiplikation von nur sieben Matrizenpaaren, sowie nur 18 Additionen von Matrizen (Idee: Merken von berechneten Werten)
			\item Laufzeit:\\$T(n)=
				\left\lbrace
					\begin{array}{ll}
						n^3 + n^2(n-1) & \text{falls } n\leq 2^{k_0} \text{ für eine Konstante } k_0 \geq 0\\
						7T\left(\half{n}\right)+18\cdot \left(\half{n}\right) & \text{sonst}
					\end{array}
				\right. 
				\\
				\overset{\text{Master Theorem (1)}}{\Rightarrow} \Theta(n^{\log_2 7})\subset\BigO(n^{2.91})$ (wobei $n$ eine Zweierpotenz ist)
		\end{enumerate}
	\item[Beste asymptotische Laufzeit:] Bei einem Algorithmus von Coppersmith und Winograd (1990): $\BigO(n^{2.37\dots})$. Es gibt auch Algorithmen mit einer geringeren asymptotischen Laufzeit, aber mit riesigen Konstanten.
\end{description}

\subsection{Selection}
\begin{itemize}
	\item in einer Menge $A$ mit $n$ Elementen mit einer totalen Ordnung $\leq$ wird das $k$-kleinste Element gesucht
	\item einfacher Algorithmus: sortieren der Elemente und herausnehmen des $k$-ten (Laufzeit: $\BigO(n\log n)$)
	\item rekursiver Ansatz in $\BigO(n)$:
		\begin{enumerate}
			\item die Menge $A$ wird in zwei Teile $A_1,A_2$ geteilt, sodass $x < y$ für jedes $x \in A_1, y\in A_2$
			\item je nachdem ob $|A_1| \geq k$ arbeitet der Algorithmus auf $A_1$ oder $A_2$ weiter
			\item zuerst wird $A$ in Gruppen der Größße 5 aufgeteilt, dann kann der Median $m$ der Mediane der $\ceilFrac{n}{5}$ Gruppen durch den rekursiven Aufruf von \select~berechnet werden
		\end{enumerate}
	\item \vergleichTwo~~\algoTwo{Selection}{\begin{algorithm}[H]
	\SetAlgoVlined
	\SetKwProg{Fn}{Function}{}{end}
	\KwIn{Menge $A$, Integer $k$}
	\KwOut{$k$-kleinstes Element in $A$}
	\BlankLine
	\Begin{
		\If{$|A|\leq 10$}{
	 		\KwRet{direkt das $k$-kleinste Element aus $A$}
		}
		$(A_1,A_2) \leftarrow \split(A)$\\
		\eIf{$|A|\geq k$}{
	 		\KwRet{\select($A_1,k$)}
		}{
			\KwRet{\select($A_2,k-|A_1|$)}
		}
	}
\end{algorithm}},~\algoTwo{Split}{\begin{algorithm}[H]
	\SetAlgoVlined
	\SetKwProg{Fn}{Function}{}{end}
	\KwIn{Menge $A$}
	\KwOut{geteilte Menge $A$ als Mengen $A_1,A_2$}
	\BlankLine
	\Begin{
		teile $A$ in $\floorFrac{n}{5}$ Gruppen von $5$ Elementen (und einer möglichen übrigen Gruppe)\\
		$M \leftarrow$ Menge der $\ceilFrac{n}{5}$ (oberen) Mediane aller Gruppen\\
		$m \leftarrow \select\left(M,\ceilFrac{|M|}{2}\right)$\\
		$A_1,A_2 \leftarrow \emptyset$\\
		\For{$x\in A$}{
			\eIf{$x\leq m$}{
		 		$A_1 \leftarrow A_1 \cup \{x\}$
			}{
				$A_2 \leftarrow A_2 \cup \{x\}$
			}
		}
		\KwRet{$(A_1,A_2)$}
	}
\end{algorithm}}
\end{itemize}