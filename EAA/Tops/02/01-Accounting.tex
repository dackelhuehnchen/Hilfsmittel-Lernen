\subtop{Accounting Methode (Abrechnungsverfahren)}{-1.45}
\begin{enumerate}
	\item Idee: Bezahlen für mögliche kommende Operationen mithilfe von amortisierten Kosten $\hat{c}$
	\item $c-\hat{c}$ ($c$ sind die wirklichen Kosten) sind die reservierten Kosten für spätere Operationen, dessen $\hat{c}$ nicht für die wirklichen Kosten ausreichen
	\item für $\hat{c}$ gilt: $\sum\limits_{i=1}^{n}c_i \leq \sum_{i=1}^{n} \hat{c}_i$ und ist somit eine obere Grenze der Gesamtkosten
\end{enumerate}
\example{MultiPop~(Fortsetzung)}{\ \\\vspace*{-\baselineskip}
	\begin{enumerate}
		\item aktuelle Kosten für \push: 1 Einheit
		\item aktuelle Kosten für \multipop: $\min(k,|S|+1)$
		\item amortisierte Kosten für \push: 2 Einheit (1 für \push, die andere für \multipop)
		\item amortisierte Kosten für \multipop: 1 Einheit (benötigt, falls $k>|S|$)
	\end{enumerate}
	Alle Kosten sind konstant $\Rightarrow$ Laufzeit ist linear (in $\BigO(n)$)
}