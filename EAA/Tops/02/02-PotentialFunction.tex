\subtop{Potentialfunktionsverfahren}{-1.45}
\begin{enumerate}
	\item definieren einer Potentialfunktion $\Phi$, die jedem möglichen Zustand einer Datenstruktur einen Wert zuweist
	\item bei einer Abfolge von $n$ Operationen erhalten wir: $\hat{c}_i = c_i + \underbrace{\Phi(D_i) - \Phi(D_{i-1})}_{\text{Potentialdifferenz}}$\\
	mit $D_i$ ist Zustand der Datenstruktur nach der $i$-ten Operation und $D_0$ Startzustand vor der ersten Operation\\
	$\Rightarrow \sum\limits_{i=1}^{n} c_i = \sum\limits_{i=1}^{n} \hat{c}_i + \Phi(D_0) - \Phi(D_n)$
	\item wenn $\Phi$ so gewählt ist, dass $\Phi(D_n) \geq \Phi(D_0)$, dann ist $\sum\limits_{i=1}^{n} \hat{c}_i$ eine Obergrenze der Gesamtkosten des Algorithmus.
\end{enumerate}
\example{MultiPop~(Fortseztzung)}{\ \\\up
	\begin{enumerate}
		\item $\Phi $ ist die Anzahl $|S|$ der Elemente auf dem Stack $S$
		\item amortisierte Kosten von \push: $\c = 1+ \Phi(D_1) = 1+1=2$
		\item amortisierte Kosten von \multipop($k$): $\c = \min(k, |S|+1)- \min(k, |S|) \in \{0,1\}$
	\end{enumerate}
	Somit ist die Laufzeit linear ($\in \BigO(n)$).
}