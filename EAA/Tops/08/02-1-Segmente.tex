\subsection{Schneiden von Segmenten}
\begin{description}
	\item[Problem:] Gibt es in einer Menge von $n$ Liniensegmenten mindestens ein Paar von sich schneidenden Liniensegmenten?
	\item[mögliches Auftreten:] beim Übereinanderlegen von mehreren Schichten von Informationen auf einer Karte
	\item[Lösen des Problemes:] \ \\\up
		\begin{itemize}
			\item durch Testen aller $\left(\hspace*{-0.2cm}\begin{array}{c} n\\2\end{array}\hspace*{-0.2cm}\right)$ Paaren von Liniensegmenten ob sie sich schneiden\\
			$\Rightarrow \BigO(n^2)$
			\item die Linien können sich jedoch nur schneiden, falls sich ihre Projektionen auf die x-Achse schneiden
			\item Algorithmus von \textit{Shamos und Hoey} löst das Problem mit einem (verikalen) \sweep-Ansatz in $\BigO(n\log n)$
			\item erste Annahmen (einfacher):
				\begin{enumerate}
					\item kein Liniensegment ist vertikal
					\item kein Liniensegment besteht nur aus einem Punkt
					\item Liniensegmente schneiden sich nicht in einem ihrer Endpunkte
					\item höchstens zwei Liniensegmente schneiden sich in einem Punkt
				\end{enumerate}
			\item aus \textbf{1.} folgt, dass ein Liniensegment die \sweep~in höchstens einem Punkt schneidet
			\item der Status der \sweep~ist die Ordnung der Liniensegmente, die die \sweep~schneiden (entsprechend ihrer y-Koordinate des Schnittpunktes mit der \sweep):
				\begin{itemize}
					\item $s_1,s_2$ sind zwei Liniensegmente welche die vertikale Linie $l$ schneiden:\\
					$\Rightarrow s_1 <_l s_2 \Longleftrightarrow s_1$ schneidet $l$ strikt unter $s_2$
					\item Änderung des \sweep~Status:
						\begin{enumerate}
							\item \sweep~ist auf dem linken Endpunkt eines Segmentes  (dann wird ein neues Segment in die Ordnung eingefügt), \textit{oder}
							\item \sweep~ist auf dem rechten Endpunkt eines Segmentes  (dann wird das entsprechende Segment aus der Ordnung entfernt)
							\item zwei Segmente schneiden sich (die Ordnung der Segmente wird vertauscht)
						\end{enumerate}
				\end{itemize}
			\item Algorithmus stoppt, wenn zwei Segmente gefunden wurden, die sich schneiden $\Rightarrow$ der Ereigniszeitplan entspricht der Sequenz von $2n$ Endpunkten der $n$ Segmente, geordnet in nicht-abnehmender Reihenfolge in Bezug auf ihre $x$-Koordinate
			\item \textbf{Annahme:} zwei Liniensegmente schneiden sich, da es keinen Schnittpunkt mit drei Segmenten gibt $\Rightarrow$ es muss ein $x$ geben, sodass $s_1$ ist der direkte Vorgänger oder Nachfolger von $s_2$ in Bezug auf $<_x$
			\item \textbf{Idee von Shamos und Hoey:} wenn man zwei aufeinanderfolgende Segmente findet $\rightarrow$ testen, ob sie schneiden
		\end{itemize}
\end{description}
\topbreak
\up\up
\begin{description}
	\item[]\ \\\up\up
		\begin{itemize}
			\item wenn die \sweep~$l$ den linken Endpunkt $p$ eines Liniensegmentes $s$ müssen wir für ein anderes Liniensegment $s'$, das die \sweep~schneidet, entscheiden, ob $s$ die \sweep~ober- oder unterhalb von $s'$ schneidet\\
			$\Rightarrow$ kann in konstanter Zeit entschieden werden:
				\begin{enumerate}
					\item $p'_l$: linker Endpunkt von $s'$
					\item $p'_r$: rechter Endpunkt von $s'$
				\end{enumerate}
				dann gilt: $s$ schneidet $l$ strikt unter $s' \Longleftrightarrow \rechts{p'_lp}$ ist rechts von $\rechts{p'_lp'_r}$
			\item für die Implementation wird der Status der \sweep~in der Datenstruktur $T$ repräsentiert, welche die folgenden Operationen zulässt:
				\begin{description}
					\item[$T$.\insert$(s)$:] fügt ein Segment $s$ in den \sweep~Status ein
					\item[$T$.\delete$(s)$:] löscht ein Segment $s$ aus dem \sweep~Status
					\item[$T$.\pred$(s)$:] gibt das Segment zurück, das die \sweep~direkt unter $s$ schneidet
					\item[$T$.\succ$(s)$:] gibt das Segment zurück, das die \sweep~direkt über $s$ schneidet
				\end{description}
			mit balancierten binären Suchbäumen können diese Operationen in $\BigO(\log n)$ ausgeführt werden, falls es $\BigO(n)$ Elemente gibt
			\item der Algorithmus von \textit{Shamos und Hoey} kann in $\BigO(n\log n)$ ausgeführt werden:
				\begin{itemize}
					\item $2n$ Endpunkte $\Rightarrow$ können in $\BigO(n\log n)$ sortiert werden
					\item jede der $2n$ Iterationen der For-Schleife braucht eine konstante Anzahl an Suchbaum-Operationen
					\item \algo{Shamos und Hoey (Liniensegmentschnitte)}{arg2}
				\end{itemize}
		\end{itemize}
	\item[Spezialfälle:] \ \\\up
		\begin{description}
			\item[vertikale Segmente:] \ \\\up
				\begin{itemize}
					\item man kann die Richtung der \sweep~stören, sodass die \sweep~mit keinem anderen Liniensegment kollinear ist\\
						$\Rightarrow$ kann fehleranfällig sein
					\item stattdessen wird die \sweep~``virtuell gestört'' durch betrachten des tiefsten Endpunktes eines vertikalen Liniensegmentes als linken Endpunkt und den obersten als seinen rechten Endpunkt
				\end{itemize}
			\item[Punktsegmente:] zweimaliges Hinzufügen des einzelnen Punktes in den Ereigniszeitplans: einmal als linker Endpunkt und einmal als rechter Endpunkt
			\item[Schnitt im Endpunkt:] \ \\\up
				\begin{itemize}
					\item $p$ ist Endpunkt eines Segmentes $s$
					\item 1. Annahme: $p$ ist auch in einem Segment $s' = \oben{p'_lp'_r}$ enthalten
					\item 2. Annahme: $s'$ wurde in den \sweep~Status eingefügt vor dem Betrachten des Ereignispunktes $p$
					\item falls $p$ links von $s$ liegt: Beginn mit Einfügend von $s$ in den \sweep~Status
					\item notwendiger Vergleich: liegt $\rechts{p'_lp}$ rechts von $\rechts{p'_lp'_r}$\\
					$\Rightarrow \rechts{p'_lp}$ und $\rechts{p'_lp'_r}$ sind kollinear (bedeutet $s$ und $s'$ sind ``gleich'' im Bezug auf die aktuelle Ordnung)\\
					$\Rightarrow s$ und $s'$ schneiden sich (bzw. \textbf{allgemein:} Einfügen von $s$ in $T$ übereinstimmend mit der Ordnung $s \leq s'$ falls $(p-p'_l)\times (p'_r - p'_l)\geq 0$)
					\item nach Einfügen von $s$ muss $s'$ der Vorgänger oder Nachfolger von $s$ sein und wir finden einen Schnittpunkt
					\item ist $p$ der linke Endpunkt von $s$ hätte der Algorithmus schon im vorherigen Ereignispunkt einen Schnittpunkt gefunden
				\end{itemize}
			\item[mehr als zwei Segmente schneiden sich in einem Punkt:] für zwei dieser Segmente ist es schon wahr, dass sie Vorgänger und Nachfolger sind für eine geeignet \sweep~links des Schnittpunktes
		\end{description}
\end{description}
\topbreak
\up\up