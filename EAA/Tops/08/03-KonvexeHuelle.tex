\subtop{Konvexe Hülle}
\begin{itemize}[itemsep=0pt]
	\item ein Polygon ist gegeben durch eine Sequenz von Punkten $\langle p_1,\dots,p_n\rangle$ mit den Seiten $\oben{p_1,p_2},\dots,\oben{p_{n-1}p_n},\oben{p_np_1}$
	\item einfaches Polygon: keine zwei Seiten schneiden sich 
	\item eine Menge $S\subseteq \mathbb{R}^2$ ist \textbf{konvex}, falls $\oben{q_1q_2} \subseteq S$ für alle Punkte $p_1,p_2 \in S$
	\item ein Polygon ist konvex, wenn die Vereinigung seiner Begrenzung und seinem Inneren konvex ist
	\item ein Polygon ist konvex $\Longleftrightarrow$ der Innenwinkel ist höchstens $\pi$
	\item eine konvexe Hülle $H(Q)$ einer Menge $Q$ von Punkten ist ein konvexes Polygon mit der minimalen Anzahl an Knoten, sodass
		\begin{enumerate}
			\item die Knoten aus $H(Q)$ sind eine Teilmenge von $Q$
			\item $Q$ ist enthalten im Inneren oder auf der Begrenzung von $H(Q)$
		\end{enumerate}
	\item die Vereinigung des Inneren und der Begrenzung der konvexen Hülle einer Menge $Q$ ist der Schnitt aller konvexen Mengen, die $Q$ enthalten
	\item Knoten und Kanten, die inzident zur Außenfläche des \dg~von $Q$ sind, bilden die konvexe Hülle
	\item eine konvexe Hülle kann in $\BigO(n\log n)$ berechnet werden
\end{itemize}

\subsection{Untere Laufzeitschranke}
\begin{itemize}[itemsep=0pt]
	\item in einem algebraischen Entscheidungsbaum-Modell der $d$-ten Ordnung kann gefragt werden, ob Polynome des Grades höchstens $d$ positiv, null oder negativ bei der Eingabe\\
		$\Rightarrow$ Worst-Case-Laufzeit des Sortierens ist immer noch in $\Omega(n\log n)$
	\item betrachten für eine Menge $X=\{x_1,\dots,x_n\}$ von reellen Zahlen die Menge $Q=\{(x_1,x_1^2),\dots,(x_n,x_n^2)\}$ von Punkten in der Ebene\\
		$\Rightarrow $ alle Punkte aus $Q$ sind in der konvexen Hülle enthalten
	\item $X$ wird sortiert, indem man $H(Q)$ entgegen dem Uhrzeigersinn abarbeitet
	\item $T(n)$ ist die Laufzeit zum Berechnen der konvexen Hülle\\
		$\Rightarrow$ Sortierung kann in $\BigO(T(n)+n)$ erfolgen
	\item Worst-Case-Laufzeit zum berechnen der konvexen Hülle ist $\Omega(n\log n)$
	\item durch das Voronoi-Diagramm und den \dg~können wir die konvexe Hülle in asymptotisch optimaler Zeit konstruieren
	\item dieser Ansatz ist komplizierter als der daraus ziehbare Nutzen
	\item folgende Algorithmen basieren auf einer rotierenden \sweep
\end{itemize}
\subsection{Graham's Scan}
\begin{itemize}[itemsep=0pt]
	\item $Q$ ist die Menge von $n$ Punkten in der Ebene$q_0\in Q$
	\item $q_0\in Q$ ist der am weitesten links liegende unterste Punkt aus $Q$
	\item die Punkte werden mithilfe eines Strahls, ausgehend aus $q_0$ und rotierend entgegen des Uhrzeigersinnes, abgearbeitet
	\item der Algorithmus arbeitet die Punkte $q\in Q\setminus \{q_0\}$ in ansteigender Ordnung der Winkel zwischen $\rechts{q_0q}$ und der horizontalen Linie durch $q_0$ in positiver Richtung
	\begin{center}(in Bezug auf die Ordnung $q<_{q_0}q' \Longleftrightarrow \rechts{q_0q}$ liegt rechts von $\rechts{q_0q'}$)\end{center}
	\item wenn es eine Linie durch $q_0$ und mehrere Punkte aus $Q\setminus \{q_0\}$ gibt, wird nur der am weitesten entfernte Punkt betrachtet (nur dieser kann ein Punkt von $H(Q)$ sein)
	\item als Vorverarbeitung kann man alle Punkte außer den am weitesten entfernten Punkt löschen (bei mehreren Punkten auf einer Linie)
	\item iteratives Berechnen der konvexen Hülle von $\{q_0,\dots,q_i\},~i=3,\dots,m$
\end{itemize}
\topbreak
\up\up
\begin{itemize}[itemsep=0pt]
	\item durch die Vorverarbeitung ist $\langle q_0,q_1,q_2\rangle$ die konvexe Hülle von $\{q_0,q_1,q_2\}$
	\item $S=\langle q_0=p_1,p_2,\dots,p_{|S|}\rangle$ ist die konvexe Hülle von $\{q_0,\dots,q_{i-1}\}$
	\item beim Hinzufügen von $q_i$ muss der Innenwinkel geprüft werden (müssen wir in dem Polygon $S+q_i=\langle p_1,\dots,p_{|S|},q_i\rangle$ an der Stelle $p_{|S|}$ auf dem Weg von $q_i$ nach $p_{|S|-1}$ rechts abbiegen)\\
	ist der Innenwinkel größer als $\pi$, ist $p_{|S|}$ nicht Teil der konvexen Hülle und wird entfernt und wird nicht wieder als potentieller Teil der konvexen Hülle betrachtet
	\item iteratives Wiederholen des letzten Schrittes bis der Innenwinkel wieder kleiner als $\pi$ ist
	\item hieraus erhalten wir ein Polygon $\langle q_0=p_1,p_2,\dots,p_j,q_i\rangle$, wobei der Innenwinkel bei $p_2,\dots,p_{j-1} <\pi$, weil es schon ein konvexes Polygon war
	\item somit liegt $q_i$ links von
		\begin{enumerate}
			\item $\rechts{p_{j-1}p_j}$ durch Konstruktion
			\item $\rechts{p_{1}p_j}$ durch die Ordnung
			\item $\rechts{p_{1}p_2}$ durch die Wahl $p_1=q_0$
		\end{enumerate}
	\item \textbf{Laufzeit:} (\algo{Graham's Scan}{arg2})%TODO Algorithmus
		\begin{enumerate}
			\item Sortieren in $\BigO(n\log n)$\\
				$\Rightarrow$ alle Punkte von $Q$, die im Inneren eines Segmentes zwischen $q_0$ und einem Punkt $q\in Q$ liegen, sind direkt nach $q$ einsortiert
			\item Löschen der Punkte in (gesamt betrachtet) linearer Zeit
			\item alle übrigen Punkte werden einmal auf den Stack gepushed und höchstens einmal gepopped
			\item die for-Schleife liegt in linearer Zeit
			\item somit wird die Laufzeit vom Sortiervorgang dominiert und der Algorithmus liegt in $\BigO(n\log n)$
		\end{enumerate}
\end{itemize}
\subsection{Jarvis' march}
\begin{itemize}[itemsep=0pt]
	\item um die $\Omega(n\log n)$ Laufzeit zur Berechnung einer konvexen Hülle von $n$ Punkten zu erhalten, wird benutzt, dass alle Eingabepunkte auf der konvexen Hülle liegen
	\item ist besser, als die $\Omega(n\log n)$-Grenze, wenn nur wenige Punkte auf der konvexen Hülle liegen
	\item Start: am weitesten links liegender unterster Punkt
	\item der analoge Algorithmus in 3D heißt ``gift wrapping algorithm''
	\item die ersten $k$ Knoten der konvexen Hülle ($q_0=p_1,\dots,p_k$) sind schon entgegen des Uhrzeigersinnes berechnet worden\\
	$\Rightarrow p_{k+1}$ ist der nächste Punkt, den man besucht, wenn man kreisförmig um $p_k$ scannt (Start bei $\oben{p_kp_{k-1}}$)\\
	wenn es mehrere dieser nächsten Punkte gibt, dann ist $p_{k+1}$ der am weitesten von $p_k$ entfernte Punkt
	\item da $p_k$ ein Punkt der konvexen Hülle war, weiß man, dass der Winkel entgegen dem Uhrzeigersinn zwischen $\oben{p_kp_{k-1}}$ und $\oben{p_kq},~q\in Q\setminus \{p_k\}$ größer als $\pi$ ist\\
	$\Rightarrow p_{k+1}$ ist der minimale Punkt aus $Q\setminus \{p_k\}$ im Bezug auf die Ordnung $<_{p_k}$
	\item \textbf{Laufzeit:} (\algo{Jarvis' march}{arg2}) %TODO Algorithmus
		\begin{itemize}
			\item eine Iteration der \textit{repeat}-Schleife für jeden Knoten der konvexen Hülle
			\item jedes Minimum kann in linearer Zeit bestimmt werden
			\item Gesamtlaufzeit: $\BigO(nh)$ mit $h = \#$ Knoten auf der konvexen Hülle
			\item falls $h\in o(\log n)$ ist dieses Algorithmus schneller als Graham's Scan
		\end{itemize}
\end{itemize}
\subsection{Algorithmus von Chan}
\begin{itemize}[itemsep=-1pt]
	\item kombiniert Graham's Scan und Jarvis' march
	\item berechnet die konvexe Hülle in $\BigO(n\log h)$, mit $h=\#$ Knoten auf der konvexen Hülle
	\item es kann gezeigt werden, dass die Berechnung für die Entscheidung, ob eine Menge mit $n$ Punkten $h$ Elemente auf der konvexen Hülle hat, in $\Omega(n\log h)$ geschehen kann (\textit{Kirckpatrick und Seidel})
	\item der Algorithmus ist optimal \textit{output-sensitive}
\end{itemize}
\topbreak
\up\up
\begin{itemize}
	\item berechnet die konvexe Hülle einer Punktemenge durch anwenden des Jarvis' march und die folgenden beiden Tricks:
		\begin{description}
			\item[Trick 1:]\ \\\vspace*{-1.5\baselineskip}
				\begin{enumerate}[itemsep=0pt]
					\item vorläufige Annahme: $h$ ist bekannt
					\item Vorverarbeitung: Aufteilen von $Q$ in $\ceilFrac{n}{h}$ Teilmengen der Größe $\leq h$
					\item Benutzen von Graham's Scan um alle konvexen Hüllen der Teilmengen zu berechnen\\
					$\Rightarrow \BigO(\frac{n}{h}h\log h)$
					\item Anwenden von Jarvis' march
					\item in jedem Schritt von Jarvis' march muss der nächste Knoten auf der konvexen Hülle aus den Knoten der konvexen Hüllen der Teilmengen kommen
					\item der Kandidat aus jeder Teilmenge kann in $\BigO(\log h)$ mithilfe der binären Suche gefunden werden
					\item Laufzeit von Jarvis' march liegt in $\BigO(h\frac{n}{h}\log h)$
				\end{enumerate}
			\item[Trick 2:] \ \\\vspace*{-1.5\baselineskip}
				\begin{enumerate}
					\item wenn man $h$ nicht im Voraus weiß, arbeitet der Algorithmus in verschiedenen Phasen
					\item in jeder Phase wird ein Parameter $m$ (anstelle von $h$) verwendet, der die Menge von Punkten in Teilmengen der maximalen Größe $m$ teilt
					\item im Schritt des Jarvis' march werden nicht notwendigerweise alle Knoten der konvexen Hülle berechnet, aber bis zu $m+1$ Knoten der konvexen Hülle
					\item Start des nächsten Schrittes
					\item Parameter $m$ wird erhöht durch Quadrieren von $2$ bis hinzu $h\leq m<h^2$
				\end{enumerate}
		\end{description}
	\item detailliertere Version des Algorithmus:
		\begin{itemize}[itemsep=-2pt]
			\item Start mit $m=2$, Aufrechterhaltung der Bedingung $m < h^2$
			\item folgende Schritte werden in jeder Phase durchgeführt:
				\begin{enumerate}[itemsep=-2pt]
					\item Aufteilen der Menge $Q$ von Punkten in $r=\ceilFrac{n}{m}$ Mengen $Q_1,\dots,Q_r$ mit jeweils höchstens $m$ Elementen
					\item für alle $i=1,\dots,r$ wird Graham's Scan angewendet, um in $\BigO(m\log m)\subseteq \BigO(m\log h)$ die konvexe Hülle von $Q_i$ zu berechnen und die geordnete Sequenz ihrer Knoten in einem Array $H(Q_i)$ zu speichern\\
					$\Rightarrow$ insgesamt ergibt sich eine Laufzeit von $\BigO(rm\log m)=\BigO(n\log m)$ pro Phase
					\item mit Jarvis' march werden in $\BigO(mr\log m)=\BigO(n\log m)$ höchstens $m+1$ Knoten der konvexen Hülle von $Q$ wie folgt berechnet:
						\begin{itemize}
							\item[$\rhd$] Beginn mit $k=1$, dem am weitesten links liegende Punkt der untersten Punkte $p_1$ von $Q$ und $H(Q)=\langle p_1\rangle$
							\item[$\rhd$] Durchführen, solange $k\leq m$ und $H(Q)$ das Folgende nicht abgeschlossen hat:
								\begin{enumerate}
									\item nächster Knoten $p_{k+1}$ von $H(Q)$ ist das Minimum im Bezug auf $<_{p_k}$\\
									da $p_{k+1}$ der nächste Knoten aus einer der $H(Q_i)$, kann $p_{k+1}$ in zwei Schritten berechnet werden:
										\begin{enumerate}
											\item für $i=1,\dots, r$ benutzt man binäre Suche, um in $\BigO(\log m)$ den Punkt $q_i$ zu berechnen, der das Minimum im Bezug auf $<_{p_k}$ist, aus allen Knoten aus $H(Q_i)$\\
											$\Rightarrow$ pro Phase braucht dieser Schritt insgesamt $\BigO(mr\log m)=\BigO(n\log m)$
											\item Berechnen des minimalen Punktes $p_{k+1}$ in Bezug auf die Ordnung $<_{p_k}$ aus allen Punkten $q_1,\dots,q_r$\\
											$\Rightarrow$ pro Phase braucht dieser Schritt insgesamt $\BigO(mr)=\BigO(n)$
										\end{enumerate}
									\item falls $p_{k+1}=p_1 \Rightarrow H(Q)$ ist vollständig
									\item sonst wird $p_{k+1}$ zu $H(Q)$ hinzugefügt und $k$ um eins erhöht 
								\end{enumerate}
						\end{itemize}
					\item falls $H(Q)$ noch nicht vollständig ist, wird $m$ auf $m^2$ erhöht und eine neue Phase begonnen
				\end{enumerate}
		\end{itemize}
\end{itemize}
\topbreak
\up\up
\begin{itemize}
	\item \textbf{Laufzeit:}
		\begin{itemize}
			\item für jede Phase: $\BigO(n\log m)$ (mit $m=2^{2^i}$)
			\item Algorithmus stoppt, sobald $m\geq h$
			\item $k$ wird so gewählt, dass $2^{2^{k-1}}<h\leq 2^{2^k}$ gilt
			\item Gesamtlaufzeit: $\BigO(\sum\limits_{i=0}^{k}n\log 2^{2^i})$ mit $\sum\limits_{i=0}^{k}n\log 2^{2^i} = n\sum\limits_{i=0}^{k} 2^i = n(2^{k+1}-1)<4n2^{k-1}<4n\log h$
		\end{itemize}
\end{itemize}