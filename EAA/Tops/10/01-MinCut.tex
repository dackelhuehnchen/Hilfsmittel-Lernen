\subtop{Randomisierter Minimaler Schnitt in ungerichteten Graphen (mit mindestens 2 Knoten)}
\begin{itemize}[itemsep=-1pt]
	\item die Anzahl der Kanten im minimalen Schnitt soll minimal sein ($|E(S,V\setminus S)|=|\{\{v,w\};v\in S, w\notin S\}|$ ist minimal zu allen Schnitten)
	\item $\lambda = |E(S,V\setminus S)|$ ist die \textit{Kantenkonnektivität} eines Grpahen
	\item wenn ein Graph nicht verbunden ist, ist jede nicht-leere Aufteilung in die Vereinigung der Zusammenhangkomponenten ein minimaler Schnitt
	\item der Algorithmus \textit{Contract} berechnet einen minimalen Schnitt in einem verbundenen ungerichteten Graphen (\algo{Contract}{arg2}) %TODO Algorithmus
	\item \textit{Contract} kann aber nur minimale Schnitte berechnen, deren Teilmengen verbunden sind
	\item Knoten, die über mehreren Kanten verbunden sind, werden eher verbunden, als Knoten, die nur über eine Kante verbunden sind
	\item \textit{Contract} gibt einen (spezifischen) minimalen Schnitt eines verbundenen ungerichteten Graphen mit $n$ Knoten mit der Wahrscheinlichkeit $1/\scriptsize{\aoverb{n}{2}}$
		\vspace*{-1.5\baselineskip}\Proof\up
			\begin{itemize}
				\item der Schnitt $C=(S,V\setminus S)$ wird nicht ausgegeben $\Longleftrightarrow$ der Algorithmus nimmt irgendwann eine Kante der $\lambda$ Kanten aus $C$ und verschmilzt seine Endknoten
				\item Wahrscheinlichkeit, dass wir eine Kante aus $C$ nehmen, ist $\frac{\lambda}{\text{aktuelle Anzahl von Kanten}}$
				\item es gibt keinen Knoten $v$ mit $deg(v)< \lambda$
				\item $G_j=(V_j,E_j)$ ist der Graph nach der $j$-ten Operation:
					\begin{enumerate}
						\item jeder Schnitt in $G_j$ entspricht einem Schnitt in $G$\\
						$\Rightarrow \lambda_{G_j} \geq \lambda$ und $deg(v)\geq \lambda,~\forall v\in V_j\\
						\Rightarrow |E_j| \geq \half{1}\lambda(n-j)$
						\item Wahrscheinlichkeit, dass eine Kante aus $C$ im $j+1$-ten Schritt genommen wird beträgt $\dfrac{\lambda}{|E_j|}\leq \dfrac{\lambda}{\frac{1}{2}\lambda(n-j)}=\dfrac{2}{n-j}$
					\end{enumerate}
				\item Wahrscheinlichkeit, dass keine Kante aus $C$ in allen der $n-2$ Schritten genommen wird beträgt:\\
					$\begin{array}{rcl}
						\prod\limits_{j=0}^{n-3} \left(1-\dfrac{2}{n-j}\right)& = & \prod\limits_{j=0}^{n-3} \left(\dfrac{n-j-2}{n-j}\right)\\
						&=&\dfrac{2}{n(n-1)}\\
						&=&1/\scriptsize{\aoverb{n}{2}}
					\end{array}$
			\end{itemize}
\end{itemize}
\topbreak
\up\up
\begin{itemize}
	\item in einem ungerichteten Graphen gibt es höchstens $\aoverb{n}{2}$ minimale Schnitte
	\item Wahrscheinlichkeit, dass der Algorithmus mit \scriptsize{$\aoverb{n}{2}$}\normalsize Durchläufen fehlschlägt, ist höchstens
		\begin{center}
		$\left(1-\dfrac{1}{\scriptsize{\aoverb{n}{2}}}\right)^{\scriptsize{\aoverb{n}{2}}}<\dfrac{1}{e}<0.368$
		\end{center}
	\item $\left(1-\frac{1}{k}\right)^k$ ist monoton fallend und konvergiert zu $\frac{1}{e}$
	\item wenn man den Algorithmus \scriptsize{$\aoverb{n}{2}\cdot \ln n$}\normalsize-mal laufen lässt, erhält man eine Wahrscheinlichkeit, dass er fehlschlägt, von
		\begin{center}
		$\left(1-\dfrac{1}{\scriptsize{\aoverb{n}{2}}}\right)^{\scriptsize{\aoverb{n}{2}}\cdot \ln n}<\dfrac{1}{e^{\ln n}}=\dfrac{1}{n}$
		\end{center}
	\item \textit{Contract} kann in $\BigO(n^2)$ implementiert werden, wenn Mehrfachkanten durch eine Kante mit einem ganzzahligen Gewicht ersetzt werden
	\item wie oben beschrieben läuft \textit{Contract} jedoch in $\BigO(n^4\log n)$
	\item in Kombination mit einem Divide and Conquer Ansatz kann der Algorithmus in $\BigO(n^2\log^3 n)$ einen minimalen Schnitt berechnen (mit Fehlschlagwahrscheinlichkeit von $\frac{1}{n}$, \algo{FastCut}{arg2})\vspace*{-0.5\baselineskip}%TODO Algorithmus
		\begin{itemize}
			\item Rekursionsgleichung von \textit{FastCut}: $T(n)\leq 2\cdot T\left(\ceil{1+\frac{n}{\sqrt{2}}}\right)+c_1\cdot n^2\\
			\Rightarrow T(n)\in \BigO(n^2\log n)$
			\item \textit{FastCut} findet einen minimalen Schnitt erfolgreich mit der Wahrscheinlichkeit $\geq \frac{c}{\log n}$ für eine Konstante $c>0$
			\item wenn der Algorithmus $ln^2 \frac{n}{c}$-mal aufgerufen wird, haben wir die oben genannte Laufzeit und Wahrscheinlichkeit
		\end{itemize}
\end{itemize}