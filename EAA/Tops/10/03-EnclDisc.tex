\subtop{Kleinste umhüllende Scheibe}{-1.72}
\vspace*{-0.5\baselineskip}\begin{itemize}[itemsep=-1pt]
	\item Problem: für einer Menge $P$ mit $n$ Punkten in der Ebene soll ein Kreis gefunden werden mit minimalem Radius, der alle Punkte aus $P$ enthält
	\item gibt viele Möglichkeiten mit randomisierten Algorithmen dieses Problem zu lösen
	\item ein Punkt $p$ ist innerhalb oder auf der Begrenzung eines eindeutigen Kreises durch drei nicht-kollineare Punkte $p_1,p_2,p_3$ (sortiert entgegen des Uhrzeigersinns) gdw.
		\[det\left(\begin{array}{cccc}
			1 &x_1&y_1&x_1^2+y_1^2\\
			1 &x_2&y_2&x_2^2+y_2^2\\
			1 &x_3&y_3&x_3^2+y_3^2\\
			1 &x&y&x^2+y^2
		\end{array}\right)\leq 0\]
	\item \textit{MinDisc($P,\emptyset$)} berechnet in $\BigO(n)$ für eine Menge $P$ mit $n\geq 2$ Punkten in der Ebene eine kleinste umhüllende Scheibe (\algo{MinDisc}{\begin{algorithm}[H]
	\SetAlgoVlined
	\SetKwProg{Fn}{Function}{}{end}
	\KwIn{Menge $P$ mit $n\geq 2$ Punkten in der Ebene $B\subseteq P$ (Voraussetzung: die kleinste umhüllende Scheibe von $P$ hab $B$ auf ihrer Begrenzung)}
	\KwOut{kleinste umhüllende Scheibe von $P$}
	\BlankLine
	\Begin{
		$p_1,\dots,p_n \leftarrow$ zufällige Permutation von $P$, sodass $\{p_1,\dots,p_{|B|}\}=B$\\
		\If{$|B|\geq 3$}{
			\KwRet{eindeutigen Kreis durch $p_1,p_2,p_3$}
		}
		$C\leftarrow $ Kreis mit Durchmesser $\oben{p_1p_2}$\\
		\For{$i\leftarrow 3,\dots,n$}{
			\If{$p_i$ nicht im Inneren oder auf der Begrenzung von $C$}{
				$C\leftarrow \textsc{MinDisc}(\{p_1,\dots,p_i\}, B\cup \{p_i\})$
			}
		}
		\KwRet{$C$}
	}
\end{algorithm}})
		\vspace*{-1.5\baselineskip}\Proof\up
			\begin{itemize}[itemsep=-1pt]
				\item $B\subseteq P$ sind einige Punkte, von denen man weiß, dass sie auf dem Umkreis der kleinsten umhüllenden Scheibe von $P$ liegen
				\item $p_1,\dots,p_n$ ist die Ordnung, gewählt durch \textit{MinDisc($P,B$)}
				\item für $i=1,\dots,n$ gilt $P_i=\{p_1,\dots,p_i\}$ und $C_i$ ist die kleinste umhüllende Scheibe von $P_i$
				\item Zu zeigen (Korrektheit): wenn $p_i$ nicht innerhalb oder auf dem Umkreis von $C_{i-1}$, dann liegt $p_i$ auf dem Umkreis von $C_i$ und somit auch $P_i$ ist enthalten im Schnitt des Inneren und der Begrenzung von $C_i$ und $C_{i-1}\\
				\Rightarrow C_i$ und $C_{i-1}$ schneiden sich in zwei Punkten $v_1,v_2$
				\item Annahme: $v_i$ liegt nicht auf dem Umkreis von $C_i$, $C_i'$ ist der eindeutige Kreis durch $v_1,v_2,p_i\\
				\Rightarrow$ der Schnitt des Inneren und der Begrenzung von beiden ($C_i,C_{i-1}$) und somit auch $P_{i-1}$, ist enthalten im Inneren oder der Begrenzung von $C'_i$
			\end{itemize}
\end{itemize}
\topbreak\up\up
\begin{itemize}
	\item[] \begin{itemize}
		\item genauso ist $p_i$ auf dem Umkreis von $C'_i$ und $C'_i$ hat einen kleineren Radius als $C_i$\\
		\scalebox{0.9}{\usetikzlibrary{calc,through,positioning}
\begin{tikzpicture}[node distance=0cm]



\node(1) at(1,3){$\bullet$};
\node(2) at(1,1){$\bullet$};
\node(3) at(4,.5){$\bullet$};
\node(4) at(4.5,.5){};
\node(p1)[above left=of 1,xshift=0.3cm,yshift=-0.3cm]{$v_1$};
\node(p2) [below=of 2,yshift=0.3cm,xshift=-0.1cm]{$v_2$};
\node(p3) [above left=of 3,xshift=0.3cm,yshift=-0.3cm]{$p_i$};

\coordinate (13) at ($(1)!0.5!(3)$);
\coordinate (131) at ($(13)!2cm!90:(3) $);

\coordinate (12) at ($(1)!0.5!(2)$);
\coordinate (121) at ($(12)!-2cm!90:(2)$);

\coordinate (23) at ($(2)!0.5!(3)$);
\coordinate (231) at ($(23)!-2cm!90:(3)$);

\coordinate (c) at (intersection of 23--231 and 12--121);

\coordinate (14) at ($(1)!0.5!(4)$);
\coordinate (141) at ($(14)!2cm!90:(4) $);
\coordinate (24) at ($(2)!0.5!(4)$);
\coordinate (241) at ($(24)!2cm!90:(4) $);
\coordinate (34) at ($(3)!0.5!(4)$);
\coordinate (341) at ($(34)!2cm!90:(4) $);
\coordinate (c1) at (intersection of 14--141 and 24--241);

\def\ringb{(12) circle (1cm)}
\def\ringa{(c1) circle  (2.175)}

\begin{scope}
	\clip \ringa;
	\fill[fill=black!30] \ringb;
\end{scope}

\node[draw,circle through={(3)},dashed,color=blue] at(c) (v) {};
\node[left=of v,xshift=-0.6cm,color=red]{$C_{i-1}$};
\node[draw,circle through={(4)}] at(c1) (v1) {};
\node[right=of v1,xshift=-1.1cm,color=blue]{$C'_{i}$};
\node[draw,circle through={(1)},color=red] at(12) (v2) {};
\node[above right=of v2,xshift=3.1cm]{$C_{i}$};
\node[right=of 12]{$P_{i-1}$};



\end{tikzpicture}}
	\end{itemize}
	\item \begin{description}
		\item[Laufzeit:] \ \\\up
			\begin{description}
				\item[Fall 1 ($|B|\geq 3$):] \textit{MinDisc($P,B$)} kann in konstanter Zeit implementiert werden: Aussuchen von 3 Punkten aus $B$
				\item[Fall 2 ($|B|= 2$):] linear, da jeder rekursiver Aufruf konstante zeit benötigt
				\item[Fall 3 ($|B|=1$):]\ \\\up
					\begin{enumerate}
						\item linear, falls die Zeit für rekursive Aufrufe nicht beachtet wird
						\item $p_i$ ist nicht im Inneren oder der Begrenzung von $C_{i-1}$ enthalten $\Longleftrightarrow $ die Begrenzung von $C_i$ enthält höchstens $3$ Punkten von $P_i$ und $p_i$ liegt auf der Begrenzung von $C_i$
						\item da $p_1$ als ersten Knoten auf der Begrenzung fixiert wurde, muss der Kreis nur dann vergrößert werden, falls $p_i$ einer der anderen höchstens zwei Punkte aus $P_i$, die auf der Begrenzung von $C_i$ liegen
						\item Wahrscheinlichkeit für (3) liegt bei höchstens $\frac{2}{i}\\
						\Rightarrow$ Laufzeit: $n+\sum\limits_{i=3}^{n}i\cdot \frac{2}{i}\in \BigO(n)$
					\end{enumerate}
				\item[Fall 4 ($|B|=0$):] gleiches Argument wie bei Fall 3, mit dem rekursiven Aufruf mit Wahrscheinlichkeit höchstens $\frac{3}{i}$
			\end{description}
	\end{description}
\end{itemize}