\documentclass[10pt,a4paper]{scrartcl}
\usepackage{../gd}
\newcommand{\card}[1]{
\begin{tikzpicture}
\node[minimum height=4.5cm,minimum width=6cm,rotate=90] at (0,0) {
\begin{minipage}[h]{5.5cm}
\centering #1?
\end{minipage}
};
\end{tikzpicture}}

\begin{document}
\pagenumbering{roman}
\hspace*{-0.13cm}\noindent\card{Was ist eine orthogonale Beschreibung}\hspace*{-0.15cm}
\noindent\card{Was sind orthogonale Gitterlayouts}\hspace*{-0.15cm}
\noindent\card{Wie kann man eine zulässige Knotenreihenfolge zur Berechnung eines orthogonalen Layouts bestimmen}\hspace*{-0.15cm}
\noindent\card{Wie arbeitet die Schwerpunkt-Heuristik}\hspace*{-0.15cm}\\\hspace*{0.125cm}
\noindent\card{Wann ist ein Graph maximal planar}\hspace*{-0.15cm}
\noindent\card{Was ist die \textit{power iteration}}\hspace*{-0.15cm}
\noindent\card{Wie kann man einen knickminimierten Graphen kompaktieren (der Graph besteht nur aus Rechtecken)}\hspace*{-0.15cm}
\noindent\card{Was ist ein \textit{(minimal) feedback arc set}}\hspace*{-0.15cm}\\\hspace*{0.125cm}
\noindent\card{Was ist die \textit{Kapazitätsbedingung}}\hspace*{-0.15cm}
\noindent\card{Was bedeutet \textit{transitiv orientierbar}}\hspace*{-0.15cm}
\noindent\card{Wann ist ein Layout ein Schwerpunkt-Layout}\hspace*{-0.15cm}
\noindent\card{Wofür wird $L(G)$ benötigt und was ist es}\hspace*{-0.15cm}\\\hspace*{0.125cm}
\noindent\card{Welche Heuristiken gibt es zur Kreuzungsminimierung auf zwei Lagen}\hspace*{-0.15cm}
\noindent\card{Was ist die Grundidee von MDS}\hspace*{-0.15cm}
\noindent\card{Wie kann man die wirkenden Kräfte beim Spring-Embedder dem Hooke'schen Gesetz angleichen}\hspace*{-0.15cm}
\noindent\card{Was ist eine planare Einbettung}\hspace*{-0.15cm}\\\hspace*{0.125cm}
\noindent\card{Wie kann man eine $s$-$t$-Ordnung in Linearzeit herstellen}\hspace*{-0.15cm}
\noindent\card{Was ist eine kombinatorische Einbettung}\hspace*{-0.15cm}
\noindent\card{Wie können Gitterlayouts für Binärbäume bestimmt werden}\hspace*{-0.15cm}
\noindent\card{Wie berechnet man das orthogonale Layout einer Komponente}\hspace*{-0.15cm}\\\hspace*{0.125cm}
\noindent\card{Wie funktioniert das Bestimmen eines Layouts für Binärbäume mittels Konturen}\hspace*{-0.15cm}
\noindent\card{Wie kann man eine $s$-$t$-Orientierung in Linearzeit konstruieren}\hspace*{-0.15cm}
\noindent\card{Was ist die untere Schranke der Winkelauflösung in planaren geradlinigen Graphen}\hspace*{-0.15cm}
\noindent\card{Wie wird für das Flussmodell zur Berechnung der unteren Schranke der Winkelauflösung konstruiert}\hspace*{-0.15cm}\\\hspace*{0.125cm}
\noindent\card{Wann ist ein Layout ein Spektral-Layout}\hspace*{-0.15cm}
\noindent\card{Was bedeutet es, wenn eine Zuweisung von Winkelwerten \textit{lokal konsistent} ist}\hspace*{-0.15cm}
\noindent\card{Was ist eine topologische Sortierung}\hspace*{-0.15cm}
\noindent\card{Was ist ein allgemeines Flussmodell}\hspace*{-0.15cm}\\\hspace*{0.125cm}
\noindent\card{Welche Repräsentationen von Graphen gibt es}\hspace*{-0.15cm}
\noindent\card{Was ist das Besondere an Gitterlayouts}\hspace*{-0.15cm}
\noindent\card{Was ist eine $s$-$t$-Orientierung}\hspace*{-0.15cm}
\noindent\card{Was sind die Nachteile des Spring-Embedders}\hspace*{-0.15cm}\\\hspace*{0.125cm}
\noindent\card{Was ist die Facettenbedingung zum Flussnetzwerk der Winkelauflösung}\hspace*{-0.15cm}
\noindent\card{Wie ist das Flussnetzwerk zur Berechnung einer orthogonalen Beschreibung definiert}\hspace*{-0.15cm}
\noindent\card{Was ist die \textit{Manhatten Distanz}}\hspace*{-0.15cm}
\noindent\card{Was sind Gütebedingungen}\hspace*{-0.15cm}\\\hspace*{0.125cm}
\noindent\card{Welche Binärbaumdurchläufe gibt es}\hspace*{-0.15cm}
\noindent\card{Welche Eigenschaften haben seriell-parallele Graphen}\hspace*{-0.15cm}
\noindent\card{Wie ist die Definition des Flussnetzwerkes $N_{s,t}(G)$}\hspace*{-0.15cm}
\noindent\card{Was ist ein Fluss mit minimalen Kosten}\hspace*{-0.15cm}\\\hspace*{0.125cm}
\noindent\card{Warum ist die Breitenminimierung von Binärbäumen $\mathcal{NP}$-schwer}\hspace*{-0.15cm}
\noindent\card{Welche Kriterien gibt es zur Darstellung genereller Graphen}\hspace*{-0.15cm}
\noindent\card{Gibt es zu jeder lokal konsistenten Winkelzuweisung eine Einbettung}\hspace*{-0.15cm}
\noindent\card{Was ist die \textit{shift}-Methode, wofür braucht man sie und wie funktioniert sie}\hspace*{-0.15cm}\\\hspace*{0.125cm}
\noindent\card{Was ist die Konnektivität eines Graphen}\hspace*{-0.15cm}
\noindent\card{Wie kann man allgemeine planare knickminimierte Graphen kompaktieren}\hspace*{-0.15cm}
\noindent\card{Was bedeutet \textit{chordal}}\hspace*{-0.15cm}
\noindent\card{Was sind \textit{Konturen}}\hspace*{-0.15cm}\\\hspace*{0.125cm}
\noindent\card{Wovon ist die Anzahl der Kreuzungen in einem Lagenlayout abhängig}\hspace*{-0.15cm}
\noindent\card{Was ist ein maximal azyklischer Teilgraph}\hspace*{-0.15cm}
\noindent\card{Wie ist die Definition des Flussnetzwerkes $N(G)$}\hspace*{-0.15cm}
\noindent\card{Welche Formeln gelten für planare Graphen}\hspace*{-0.15cm}\\\hspace*{0.125cm}
\noindent\card{Wie kann das Kriterium zur Darstellung von generellen Graphen \textit{die Knoten sind gleichverteilt} realisiert werden}\hspace*{-0.15cm}
\noindent\card{Was ist die \textit{Flusserhaltungsbedingung}}\hspace*{-0.15cm}
\noindent\card{Wie kann man eine zulässige Lagenzuordnung finden}\hspace*{-0.15cm}
\noindent\card{Was ist eine (offene) Ohrendkomposition}\hspace*{-0.15cm}\\\hspace*{0.125cm}
\noindent\card{Was Berechnet $N_{hor}$}\hspace*{-0.15cm}
\noindent\card{Welche Annäherung hat die Schwerpunkt-Heuristik}\hspace*{-0.15cm}
\noindent\card{Warum ist die Minimierung der Höhe bei vorgegebener Breite $\mathcal{NP}$-schwer}\hspace*{-0.15cm}
\noindent\card{Wie kann man Kreise in gerichteten Graphen entfernen}\hspace*{-0.15cm}\\\hspace*{0.125cm}
\noindent\card{Wie kann man die Höhe eines Lagen-Layouts berechnen}\hspace*{-0.15cm}
\noindent\card{Warum benötigt jedes geradlinige kreuzungsfreie Aufwärtslayout im schlechtesten Fall exponentielle Fläche}\hspace*{-0.15cm}
\noindent\card{Was wird bei der Kreuzungsminimierung auf zwei Lagen gesucht}\hspace*{-0.15cm}
\noindent\card{Warum ist es wichtig, Graphen \glqq gut \grqq~zu repräsentieren}\hspace*{-0.15cm}\\\hspace*{0.125cm}
\noindent\card{Was ist die Zielfunktion $B(p)$}\hspace*{-0.15cm}
\noindent\card{Was sagt das klassische Dualitätsresultat}\hspace*{-0.15cm}
\noindent\card{Was ist eine zweifache Zusammenhangskomponente}\hspace*{-0.15cm}
\noindent\card{Was ist der Nachteil von Greedy-switch}\hspace*{-0.15cm}\\\hspace*{0.125cm}
\noindent\card{Welche besondere Form von seriell-parallelen Graphen gibt es}\hspace*{-0.15cm}
\noindent\card{Was ist ein seriell-paralleler Graph}\hspace*{-0.15cm}
\noindent\card{Was ist \textit{classical Scaling}}\hspace*{-0.15cm}
\noindent\card{Wie kann man bei der Darstellung von Graphen Distanzen erhalten}\hspace*{-0.15cm}\\\hspace*{0.125cm}
\noindent\card{Was ist die untere Schranke der Winkelauflösung in triangulierten planar eingebetteten Graphen}\hspace*{-0.15cm}
\noindent\card{Wie kann man ein Lagen-Layout konstruieren}\hspace*{-0.15cm}
\noindent\card{Wie kann man für planare Graphen ein kreuzungsfreies geradliniges Layout konstruieren}\hspace*{-0.15cm}
\noindent\card{Was ist eine kanonische Ordnung}\hspace*{-0.15cm}\\\hspace*{0.125cm}
\noindent\card{Was ist die \textit{Zwei-Lagen-Kreuzungsminimierung}}\hspace*{-0.15cm}
\noindent\card{Wie kann man zweifache Zusammenhangskomponenten berechnen}\hspace*{-0.15cm}
\noindent\card{Was wird als Gewichtsfunktion beim klassischen Scaling beim Zeichnen von Graphen verwendet}\hspace*{-0.15cm}
\noindent\card{Was ist die Knotenbedingung zum Flussnetzwerk der Winkelauflösung}\hspace*{-0.15cm}\\\hspace*{0.125cm}
\noindent\card{Wie funktioniert im Allgemeinen der Beweis, dass ein Problem in $\mathcal{NP}$ liegt}\hspace*{-0.15cm}
\noindent\card{Was ist \textit{double-centering}}\hspace*{-0.15cm}
\noindent\card{Was ist das Spektral-Layout}\hspace*{-0.15cm}
\noindent\card{Welche Arten von Kraft gibt es bei einem Spring-Embedder}\hspace*{-0.15cm}\\\hspace*{0.125cm}
\noindent\card{Wie sieht ein klassisches Flussmodell aus}\hspace*{-0.15cm}
\noindent\card{Wie werden die Knoten ins Gitter einer Komponente für ein orthogonales Layout gelegt}\hspace*{-0.15cm}
\noindent\card{Wie wird das Schwerpunkt-Layout berechnet}\hspace*{-0.15cm}
\noindent\card{Wie werden die $x$ und $y$-Koordinaten für ein lokales Minimum aller Knoten bei der Darstellung von generellen Graphen bestimmt}\hspace*{-0.15cm}\\\hspace*{0.125cm}
\noindent\card{Wann ist ein Layout vollständig bestimmt}\hspace*{-0.15cm}
\noindent\card{Wie kann man seriell-parallele Graphen auf quadratischer Fläche zeichnen}\hspace*{-0.15cm}
\noindent\card{Wie kann das Kriterium zur Darstellung von generellen Graphen \textit{adjazente Knoten sind nah beieinander} realisiert werden}\hspace*{-0.15cm}
\noindent\card{Welche Darstellungen von Binärbäumen gibt es}\hspace*{-0.15cm}\\\hspace*{0.125cm}
\noindent\card{Was ist das Schwerpunkt-Layout}\hspace*{-0.15cm}
\noindent\card{Wann und warum gibt es zu jedem planaren Graphen mit $\Delta_G \leq 4$ und kombinatorischer Einbettung genau eine orthogonale Beschreibung mit $k$ Knicken}\hspace*{-0.15cm}
\noindent\card{Wie ist die Optimalitätsbedingung zum Schwerpunkt-Layout}\hspace*{-0.15cm}
\noindent\card{Was ist di Optimalitätsbedingung}\hspace*{-0.15cm}\\\hspace*{0.125cm}
\noindent\card{Warum kann man zweifache Zusammenhangskomponenten in Linearzeit berechnen}\hspace*{-0.15cm}
\noindent\card{Was bedeutet es, wenn ein seriell-paralleler Graph linkslastig ist}\hspace*{-0.15cm}
\noindent\card{Was ist \textit{LIST-SCHEDULING}}\hspace*{-0.15cm}
\noindent\card{Wie funktioniert stress Minimierung}\hspace*{-0.15cm}\\\hspace*{0.125cm}
\noindent\card{Was ist ein \textit{(minimal) feedback set}}\hspace*{-0.15cm}
\noindent\card{Wie arbeitet die Median-Heuristik}\hspace*{-0.15cm}
\noindent\card{Warum spielt die Skalierung beim Spektral-Layout keine Rolle}\hspace*{-0.15cm}
\noindent\card{Was sind Vorteile der stress Minimierung}\hspace*{-0.15cm}\\\hspace*{0.125cm}
\noindent\card{Was ist \textit{stress Minimierung}}\hspace*{-0.15cm}
\noindent\card{Was ist das \textit{Scheduling mit Vorgängerbedingung}}\hspace*{-0.15cm}
\noindent\card{Welche Eigenvektoren werden bei der Berechnung eines Spektral-Layouts benutzt und warum}\hspace*{-0.15cm}
\noindent\card{Was ist eine $s$-$t$-Ordnung}\hspace*{-0.15cm}\\\hspace*{0.125cm}
\noindent\card{Was sind Nachteile des Spektral-Layouts}\hspace*{-0.15cm}
\noindent\card{Als was wird die dissimilarity Matrix beim Zeichnen von Graphen interpretiert}\hspace*{-0.15cm}
\noindent\card{Was ist ein Lagenzuordnung}\hspace*{-0.15cm}
\noindent\card{Was ist der Vorteil vom Einführen einer dritten Kraft $f_{attr}$ beim Spring-Embedder}\hspace*{-0.15cm}\\\hspace*{0.125cm}
\noindent\card{Wann gibt es ein eindeutiges Schwerpunkt-Layout für $G$}\hspace*{-0.15cm}
\noindent\card{Was berechnet $N_{ver}$}\hspace*{-0.15cm}
\noindent\card{Wie kann man ein orthogonales Gitterlayout berechnen}\hspace*{-0.15cm}
\noindent\card{Was sind Nachteile der stress Minimierung}\hspace*{-0.15cm}\\\hspace*{0.125cm}
\noindent\card{Wie berechnet man die untere Schranke der Winkelauflösung in geradlinigen Layouts}\hspace*{-0.15cm}
\noindent\card{Was ist eine \textit{dissimilarity matrix}}\hspace*{-0.15cm}
\noindent\card{Wie wird eine Ohrendekomposition berechnet}\hspace*{-0.15cm}
\noindent\card{Was sind Nebenbedingungen}\hspace*{-0.15cm}\\\hspace*{0.125cm}
\noindent\card{Was ist ein \textit{chord}}\hspace*{-0.15cm}
\noindent\card{Wie berechnet man eine $s$-$t$-Orientierung}\hspace*{-0.15cm}
\noindent\card{Wann und warum gibt es für jeden planaren Dreiecksgraphen mit kombinatorischer Einbettung und vorgegebener Winkelzuweisung eine geradlinige Realisierung}\hspace*{-0.15cm}
\noindent\card{Was ist ein orthogonales Layout}\hspace*{-0.15cm}\\\hspace*{0.125cm}
\noindent\card{Was ist der Vorteil von klassischem Scaling}\hspace*{-0.15cm}
\noindent\card{Was ist die Hybrid-Methode}\hspace*{-0.15cm}
\noindent\card{Wie kann man die Knicke in einem planaren Graphen mit vorgegebener Einbettung minimieren}\hspace*{-0.15cm}
\noindent\card{Wie groß ist das benötigte Gitte maximal für das orthogonale Layout einer Komponente}\hspace*{-0.15cm}\\\hspace*{0.125cm}
\noindent\card{Wie arbeitet Greedy-switch}\hspace*{-0.15cm}
\noindent\card{Wann ist eine orthogonale Beschreibung korrekt}\hspace*{-0.15cm}
\noindent\card{Wie kann man aus einem allgemeinen Flussmodell ein $s$-$t$-Flussmodell konstruieren}\hspace*{-0.15cm}
\noindent\card{Was ist ein Dreiecksgraph}\hspace*{-0.15cm}\\\hspace*{0.125cm}
\noindent\card{Was ist ein Lagen-Layout}\hspace*{-0.15cm}
\noindent\card{Was ist die Kreuzungszahl}\hspace*{-0.15cm}
\noindent\card{Wie funktioniert ein Spring-Embedder}\hspace*{-0.15cm}
\noindent\card{Wie werden die Layouts der einzelnen Komponenten zu einem orthogonalen Layout zusammengefügt}\hspace*{-0.15cm}\\\hspace*{0.125cm}
\noindent\card{Wie kann man \textit{CLIQUE} auf $PRE-SCHED_B\{<\}$ reduzieren}\hspace*{-0.15cm}
\noindent\card{Was ist der Vorteil des Schwerpunkt-Layout-Algorithmus'}\hspace*{-0.15cm}
\noindent\card{Welche Annäherung hat die Median-Heuristik}\hspace*{-0.15cm}
\noindent\card{Was sind die Vorteile des Spring-Embedders}\hspace*{-0.15cm}\\\hspace*{0.125cm}
\noindent\card{Was ist der Nachteil von klassischem Scaling}\hspace*{-0.15cm}
\noindent\card{Wie wird das Spektral-Layout berechnet}\hspace*{-0.15cm}
\end{document}