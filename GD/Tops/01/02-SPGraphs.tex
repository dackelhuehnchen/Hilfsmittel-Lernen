\subtop{Serien-parallele (SP) Graphen}{-1.45}
Ein \textit{SP}-Graph ist:
	\begin{itemize}[itemsep=-1pt]
		\item gerichtet
		\item besteht entweder aus zwei Knoten $s,t$ und der Kante $\{s,t\}$ oder
		\item aus zwei \textit{SP}-Graphen $G_1,G_2$ mit $s_1,s_2,t_1,t_2$, entstanden aus
			\begin{description}
				\item[serielle Komposition:] $t_1,s_2$ werden verschmolzen, $s_1 \rightarrow s,t_2\rightarrow t$
				\item[parallele Komposition:] jeweils $s_1,s_2$ und $t_1,t_2$ werden zu $s,t$ verschmolzen
			\end{description}
	\end{itemize}
Fakten:
\begin{itemize}[itemsep=-1pt]
	\item Jeder \textit{SP}-Graph ist \textit{azyklisch} und \textit{planar}.
	\item Jedes kreuzungsfreie Aufwärtslayout für geordnete \textit{SP}-Graphen benötigt im schlechtesten Fall eine Gitter, das exponentiell groß in der Anzahl der Knoten des Graphen ist.
		\vspace*{-1.5\baselineskip}
		\Proof
		\usetikzlibrary{positioning,arrows,calc,intersections,patterns}
\begin{tikzpicture}[node distance=0.75cm]
%\draw[step=1.0,black,thin,color=red!40] (-8.5,-4.5) grid (7.5,10.5);

\newcommand\ZeichneGerade[7]{% 
  \coordinate (Punkt1) at (#1,#2); 
  \coordinate (Punkt2) at (#3,#4); 
  \pgfmathsetmacro\m{(#4-#2)/(#3-#1)}% 
  \pgfmathsetmacro\n{#2-\m*#1}% 
  \draw[color=blue,dashed] plot[domain=#5:#6] (\x,{\m*\x+\n});
	\coordinate (1-#7) at (#5,{\m*#5+\n});
%	\node at (1-#7) {1-#7};
	\coordinate (2-#7) at (#6,{\m*#6+\n});
%	\node at (2-#7) {2-#7};
} 
\newcommand\ZeichneGeradeColor[8]{% 
  \coordinate (Punkt1) at (#1,#2); 
  \coordinate (Punkt2) at (#3,#4); 
  \pgfmathsetmacro\m{(#4-#2)/(#3-#1)}% 
  \pgfmathsetmacro\n{#2-\m*#1}% 
  \draw[color=#8] plot[domain=#5:#6] (\x,{\m*\x+\n});
	\coordinate (1-#7) at (#5,{\m*#5+\n});
%	\node at (1-#7) {1-#7};
	\coordinate (2-#7) at (#6,{\m*#6+\n});
%	\node at (2-#7) {2-#7};
} 
\newcommand\Schnittpunkt[7]{
	\path[name path=#3](#1.center)--(#2.center);
	\path[name path=#6](#4.center)--(#5.center);
	\fill[name intersections={of=#3 and #6,by={#7}}]
	(#7) circle (2.3pt);
}
\coordinate (xminL) at (-8,0);
\coordinate (xminU) at (-8,4);
\coordinate (xmaxL) at (6,0);
\coordinate (xmaxU) at (6,4);
\draw[color=black!30](xminU)--(xmaxU);
\draw[color=black!30](xminL)--(xmaxL);


\def\b{$\bullet$}
\coordinate[label={270:$s_n$}] (s) at (0,0); 
\fill[draw](s)circle [radius=2pt];

\coordinate[label={90:$t_n$}] (t) at (1,4); 
\fill[draw](t)circle [radius=2pt];

\coordinate[label={0:$s_{n-1}$}] (s1) at (2,2.5); 
\fill[draw](s1)circle [radius=2pt];

\path[draw](t)--(s)--(s1)--(t);

\ZeichneGerade{0}{0}{2}{2.5}{2}{6}{2}
\ZeichneGerade{1}{4}{0}{3.5}{-8}{6}{3}
\Schnittpunkt{1-3}{2-3}{1}{xminL}{xmaxL}{2}{a}
\Schnittpunkt{1-3}{2-3}{1}{1-2}{2-2}{2}{b}

\fill[pattern=north west lines,pattern color=purple](xminL)--(a)--(1-3);
\fill[pattern=north west lines,pattern color=red](b)--(2-2)--(2-3);

\node (sn) [below=of a,color=purple] {$s_{n+1}$};
\draw[->,>=latex,very thick,color=purple](sn)--++(-0.75,0.9);
\node (tn) [above left=of b,color=red] {$t_{n+1}$};
\draw[->,>=latex,very thick,color=red](tn)--++(1.8,-0.3);

\draw[color=green!70!black] plot[domain=-2.2:1] (\x,{1.25*\x+2.75});
\ZeichneGeradeColor{3.2}{4}{-3.8}{0}{-3.8}{3.2}{4}{orange}

\node[anchor=west] (u) at (6,4) {$u$};
\node[anchor=west] (l) at (6,0) {$\ell$};
\node[anchor=west] (p) at (3.1,3.8) {$p$};
\node[anchor=north] (x) at (-2.2,0) {$x$};
\node[anchor=south] (y) at (-0.85,1.7) {$y$};
\node[anchor=south] (a) at (a) {$a$};
\node[anchor=south] (b) at (b) {$b$};
\draw[fill](3.2,4)circle(2pt);
\draw[fill](-2.2,0)circle(2pt);
\draw[fill](-0.85,1.7)circle(2pt);

\node[anchor=north west,text width=10cm] at(-8,7.5){$\pi=(x,s_n,p,t_n)$\\$\Delta_1=(a,x,t_n),\Delta_2=(t_n,p,b)$\\$A(\pi)\geq 2A(\Delta_{G_n})$\\$A(p,y,t_n)=A(\Delta_2)$\\Das Viereck $(x,s_n,p,y)$ ist Teil des Dreiecks $\Delta_1$.\\$\Rightarrow A(\Delta_{G_{n+1}})= A(\pi)+A(\Delta_1)+A(\Delta_2)$\\$\geq 2A(\pi)\geq 4A(\Delta_{G_n})$};
\end{tikzpicture}
\end{itemize}
\topbreak
\vspace*{-2\baselineskip}
\subsection{Divide-and-Conquer Ansatz zur Erstellung von \textit{SP}-Graphen auf einem Gitter der Größe $\BigO(n^2)$}
\begin{itemize}[itemsep=-1pt]
	\item im Dekompositionsbaum stehen Q-Knoten nur rechts von einem P-Knoten (\textit{\textbf{linkslastig}})
	\item Algorithmus:\\
		\begin{minipage}{0.3\textwidth}
			\usetikzlibrary{positioning,patterns}
\begin{tikzpicture} [every node/.style={draw,circle,fill},node distance=0.2cm,scale=0.75]
\newcommand*{\rechterWinkelRadius}{.5cm}
\newcommand*{\rechterWinkel}[2]{% #1 = point, #2 = start angle
   \draw[shift={(#2:\rechterWinkelRadius)}] (#1) arc[start angle=#2, delta angle=90, radius = \rechterWinkelRadius];
   \fill[shift={(#2+45:\rechterWinkelRadius/2)}] (#1) circle[radius=3\pgflinewidth];
}

\def\ymin{-6}
\def\xmin{-5}
\def\ymax{5}
\def\xmax{5}

%\foreach \x in {\xmin,...,\xmax}
%	\draw[color=black!30](\x,\ymin-0.5)--(\x,\ymax+0.5);
%\foreach \y in {\ymin,...,\ymax}
%	\draw[color=black!30](\xmin-0.5,\y)--(\xmax+0.5,\y);

\foreach \l/\n/\a/\x/\y in {t/t/90/0/3,s/s/270/0/-3,ht//0/0/2,hs//0/0/-2}{
	\coordinate[label={\a:$\n$}] (\l) at (\x,\y);
	\draw[fill](\l)circle[radius=1.5pt];
	\draw(\x+0.2,\y)--(\x+0.5,\y);
}
\draw(s)--(t)--(-3,0)--cycle;
\rechterWinkel{-3,0}{-45}

\draw(0.35,3)--(0.35,2);
\draw(0.35,-3)--(0.35,-2);
\node[draw=none,fill=none] (hs) at (0.6,-2.5) {$h_s$};
\node[draw=none,fill=none] (ht) at (0.6,2.5) {$h_t$};

\draw(1,3)--(1,-3);
\draw(0.85,3)--(1.15,3);
\draw(0.85,-3)--(1.15,-3);
\draw[dashed](-3,0)--(1.5,0);
\node[draw=none,fill=none] at (1.25,1.5) {$\frac{h}{2}$};
\node[draw=none,fill=none] at (1.25,-1.5) {$\frac{h}{2}$};
\node[draw=none,fill=none] at (-1,-1) {$\Delta(G)$};

\end{tikzpicture}
		\end{minipage}
		\begin{minipage}{0.55\textwidth}
			\begin{enumerate}
				\item Layout von $G$ liegt in einem rechtwinkligen, gleichschenkligen Dreieck $\Delta_G$, mit vertikaler Basis und linksliegenden Scheitel
				\item auf der unteren / oberen Ecke von $\Delta(G)$ liegt die Quelle / Senke von $G$ aber kein Knoten liegt auf den linken Ecken von $\Delta(G)$
				\item falls $v$ Nachbar der Quelle / Senke von $G$ ist, dann liegt kein anderer Knoten rechts der Senkrechten durch $v$ und unterhalb der fallenden / oberhalb der steigenden Diagonalen durch $v$
			\end{enumerate}
		\end{minipage}\\
		\vspace*{\baselineskip}\\
		\begin{minipage}{0.3\textwidth}
			\underline{Serielle Komposition}:\\
			\usetikzlibrary{positioning,patterns}
\begin{tikzpicture} [every node/.style={},node distance=0.2cm,scale=0.75]
\newcommand*{\rechterWinkelRadius}{.5cm}
\newcommand*{\rechterWinkel}[2]{% #1 = point, #2 = start angle
   \draw[shift={(#2:\rechterWinkelRadius)}] (#1) arc[start angle=#2, delta angle=90, radius = \rechterWinkelRadius];
   \fill[shift={(#2+45:\rechterWinkelRadius/2)}] (#1) circle[radius=3\pgflinewidth];
}

\def\ymin{-3}
\def\xmin{-10}
\def\ymax{10}
\def\xmax{2}

%\foreach \x in {\xmin,...,\xmax}
%	\draw[color=black!30](\x,\ymin-0.5)--(\x,\ymax+0.5);
%\foreach \y in {\ymin,...,\ymax}
%	\draw[color=black!30](\xmin-0.5,\y)--(\xmax+0.5,\y);

\foreach \l/\n/\a/\x/\y in {t1//90/0/3,s/s/270/0/-3,ht//0/0/6,hs//0/0/-2,t/t/90/0/7}{
	\coordinate[label={\a:$\n$}] (\l) at (\x,\y);
	\draw[fill](\l)circle[radius=1.5pt];
}
\draw(s)--(t1)--(-3,0)--cycle;
\draw(t1)--(t)--(-2,5)--cycle;
\draw[dashed](-2,5)--(-5.5,1.5);
\draw[dashed](-3,0)--(-5.5,2.5);

\node at (-1.25,0) {$\Delta(G_1)$};
\node at (-0.85,5) {$\Delta(G_2)$};


\end{tikzpicture}
		\end{minipage}
		\begin{minipage}{0.3\textwidth}
			\underline{Parallele Komposition}:\\
			\usetikzlibrary{positioning,patterns}
\begin{tikzpicture} [every node/.style={},node distance=0.2cm,scale=0.75]
\newcommand*{\rechterWinkelRadius}{.5cm}
\newcommand*{\rechterWinkel}[2]{% #1 = point, #2 = start angle
   \draw[shift={(#2:\rechterWinkelRadius)}] (#1) arc[start angle=#2, delta angle=90, radius = \rechterWinkelRadius];
   \fill[shift={(#2+45:\rechterWinkelRadius/2)}] (#1) circle[radius=3\pgflinewidth];
}

\def\ymin{-6}
\def\xmin{-10}
\def\ymax{6}
\def\xmax{2}

%\foreach \x in {\xmin,...,\xmax}
%	\draw[color=black!30](\x,\ymin-0.5)--(\x,\ymax+0.5);
%\foreach \y in {\ymin,...,\ymax}
%	\draw[color=black!30](\xmin-0.5,\y)--(\xmax+0.5,\y);

\foreach \l/\n/\a/\x/\y in {ht/v_{t_2}/0/0/2,hs/v_{s_2}/0/0/-2,t/t/160/0/5,s/s/240/0/-5,s1//0/-3/-2,t1//0/-3/2}{
	\coordinate[label={\a:$\n$}] (\l) at (\x,\y);
	\draw[fill](\l)circle[radius=1.5pt];
}
\draw(0,-3)--(0,3)--(-3,0)--cycle;
\draw(t1)--(s1)--(-5,0)--cycle;
\draw[dashed](t1)--(t)--(0.5,5.5);
\draw[dashed](s1)--(s)--(0.5,-5.5);
\draw[dashed](0,3)--(t)--(0,5.5);
\draw[dashed](0,-3)--(s)--(0,-5.5);

\node at (-1.35,0) {$\Delta(G_2)$};
\node at (-3.75,0) {$\Delta(G_1)$};

\end{tikzpicture}
		\end{minipage}
		\begin{minipage}{0.3\textwidth}
			Bild (1):\\
			\usetikzlibrary{positioning,patterns}
\begin{tikzpicture} [every node/.style={},node distance=0.2cm,scale=0.75]
\newcommand*{\rechterWinkelRadius}{.5cm}
\newcommand*{\rechterWinkel}[2]{% #1 = point, #2 = start angle
   \draw[shift={(#2:\rechterWinkelRadius)}] (#1) arc[start angle=#2, delta angle=90, radius = \rechterWinkelRadius];
   \fill[shift={(#2+45:\rechterWinkelRadius/2)}] (#1) circle[radius=3\pgflinewidth];
}

\def\ymin{-6}
\def\xmin{-10}
\def\ymax{6}
\def\xmax{2}

%\foreach \x in {\xmin,...,\xmax}
%	\draw[color=black!30](\x,\ymin-0.5)--(\x,\ymax+0.5);
%\foreach \y in {\ymin,...,\ymax}
%	\draw[color=black!30](\xmin-0.5,\y)--(\xmax+0.5,\y);

\foreach \l/\x/\y in {1/-3.2/-1,2/-0.3/-1.5,s1/-3/-2,s2/0/-3,s/0/-5}{
	\coordinate[] (\l) at (\x,\y);
	\draw[fill](\l)circle[radius=1.5pt];
}
\foreach \l/\x/\y in {t/0/5,t1/-3/2}{
	\coordinate[] (\l) at (\x,\y);
}
\draw(0,-3)--(0,3)--(-3,0)--cycle;
\draw(t1)--(s1)--(-5,0)--cycle;
\draw[dashed](t1)--(t)--(0.5,5.5);
\draw[dashed](s1)--(s)--(0.5,-5.5);
\draw[dashed](0,3)--(t)--(0,5.5);
\draw[dashed](0,-3)--(s)--(0,-5.5);
\draw(s1)--(1)--(s);
\draw(s2)--(2)--(s);
\end{tikzpicture}
		\end{minipage}\\
		\vspace*{\baselineskip}\\
		Aus den Höhenunterschieden und Höhen zu den Komponenten können mittels eines \textit{preorder} Durchlaufes durch den Dekompositionsbaum die absoluten Koordinaten ermittelt werden (ähnlich wie bei $x$-Offsets der Binärbäume).\\
		Der Algorithmus erstellt in Linearzeit aus einem linkslastig geordneten Dekompositionsbaum eines einfachen \textit{SP}-Graphes ein Gitterlayout, das
			\begin{itemize}
				\item kreuzungsfrei ist (Kreuzungen können nur bei paralleler Komposition entstehen): aus (3.) folgt das Bild (1)
				\item höchstens quadratische Fläche benötigt: per Induktion beweisbar (alle Teile des Graphen $G$ liegen in seinem Dreieck $\rightarrow$ bleibt zu zeigen, dass die Höhe von $\Delta(G)$ linear in der Anzahl der Knoten ist).
			\end{itemize}
	\item Schönere Darstellung mit \textit{Sichtbarkeitsrepräsentation}
	\item Erweiterung auf die Darstellung mit \textit{orthogonalen Buskanten}
\end{itemize}