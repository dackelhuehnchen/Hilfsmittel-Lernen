\begin{TOP}{Lagen-Layouts (Layered Layout)}
Es sollen gerichtete hierarchische Graphen so dargestellt werden, dass die Knoten auf unterschiedlichen Lagen liegen. Es sollen im End-Layout folgende Bedingungen gelten:
	\begin{enumerate}[itemsep=-1pt]
		\item möglichst alle Kanten sind aufwärts gerichtet
		\item möglichst wenig Kanten erzeugen Kreuzungen
		\item alle Kanten sind möglichst vertikal und geradlinig dargestellt
		\item alle Knoten sind gleichmäßig verteilt und lange Kanten werden vermieden
	\end{enumerate}
{\Large{\textbf{Algorithmus zur Konstruktion eines Lagen-Layouts:}}}
\begin{description}
	\item[Schritt 1: Kreise entfernen] Finden einer minimalen Anzahl an Kanten, durch deren Entfernung der Graph azyklisch ist, und drehe ihre Richtung um
	\item[Schritt 2: Lagenzuordnung] Berechnung einer guten Zuordnung der Knoten auf Lagen ($y$-Koordinaten), sodass alle Kanten aufwärts gerichtet sind. Alle Kanten, die mindestens eine Lage überqueren werden ersetzt durch einen Pfad mit Dummy-Knoten auf jeder kreuzenden Lage.
	\item[Schritt 3: Kreuzungsreduktion] Berechnung einer Anordnung für jede Lage, für welche die Anzahl der entstehenden Kreuzungen minimal ist
	\item[Schritt 4: Knoten-/Kantenpositionierung (horizontale Koordinatenzuweisung)] Berechnung der $x$-Koordinaten der Knoten (\& Dummy-Knoten), so dass keine Überlappungen entstehen. Einfügen der Kanten geradlinig und Entfernen der Dummy-Knoten. Wiederherstellen der ursprünglichen Kantenrichtungen
\end{description}
\example{\text{Sugiyma Framework}}{\ \\\vspace*{-\baselineskip}
\hspace*{-1cm}
	\begin{minipage}{0.24\textwidth}
		\usetikzlibrary{positioning,patterns,calc,arrows,decorations.markings}

\begin{tikzpicture}[every node/.style={draw,fill,circle},node distance=0cm,scale=0.75]
\tikzset{%
    > = triangle 45,
        arrow over path/.style = {%
        decoration = {%
        markings,
        mark = at position .65 with {\arrow {>};}
        },
        postaction = decorate
    }
}

\foreach \x/\y/\a/\l in {0/0/180/1,0/4/180/5,4/4/0/3,4/0/0/2,2/6/0/4}{
	\coordinate[label={\a:$\l$}] (v\l) at (\x,\y);
	\draw[fill](v\l)circle[radius=4pt];
}

\foreach \a/\b in {1/2,1/3,2/3,3/5,4/3,5/1,5/2,5/4}
	\draw[arrow over path,very thick](v\a)--(v\b);
\draw[color=blue,thick] (1.65,4)circle[radius=10pt];
\end{tikzpicture}\\
		\centering Input
	\end{minipage}
	\begin{minipage}{0.24\textwidth}
		\usetikzlibrary{positioning,patterns,calc,arrows,decorations.markings}

\begin{tikzpicture}[every node/.style={draw,fill,circle},node distance=0cm,scale=0.75]
\tikzset{%
    > = triangle 45,
        arrow over path/.style = {%
        decoration = {%
        markings,
        mark = at position .65 with {\arrow {>};}
        },
        postaction = decorate
    }
}

\foreach \x/\y/\a/\l in {0/0/180/1,0/4/180/5,4/4/0/3,4/0/0/2,2/6/0/4}{
	\coordinate[label={\a:$\l$}] (v\l) at (\x,\y);
	\draw[fill](v\l)circle[radius=4pt];
}

\foreach \a/\b in {1/2,1/3,2/3,5/3,4/3,5/1,5/2,5/4}
	\draw[arrow over path,very thick](v\a)--(v\b);
\draw[color=blue,thick] (2.4,4)circle[radius=10pt];
\end{tikzpicture}\\
		\centering Schritt 1
	\end{minipage}
	\begin{minipage}{0.48\textwidth}
		\usetikzlibrary{positioning,patterns,calc,arrows,decorations.markings}

\begin{tikzpicture}[every node/.style={draw,fill,circle},node distance=0cm,scale=0.75]
\tikzset{%
    > = triangle 45,
        arrow over path/.style = {%
        decoration = {%
        markings,
        mark = at position .75 with {\arrow {>};}
        },
        postaction = decorate
    }
}

\foreach \x/\y/\a/\l in {-2/2/180/1,0/0/180/5,0/6/0/3,-2/4/180/2,2/2/0/4}{
	\coordinate[label={\a:$\l$}] (v\l) at (\x,\y);
	\draw[fill](v\l)circle[radius=4pt];
}

\foreach \a/\b in {1/2,1/3,2/3,5/3,4/3,5/1,5/2,5/4}
	\draw[arrow over path,very thick](v\a)--(v\b);

\foreach \x/\y/\a/\l in {-2/2/180/1,0/0/180/5,0/6/0/3,-2/4/180/2,2/2/0/4}{
	\coordinate[label={\a:$\l$}] (v\l) at (\x+6,\y);
	\draw[fill](v\l)circle[radius=4pt];
}
\foreach[count=\c] \x/\y in {5.3/2,6.6/2,5.3/4,6.6/4,8/4}{
	\coordinate (c\c) at (\x,\y);
	\draw(c\c)circle[radius=4pt];
}
\foreach \a/\b in {v5/v1,v5/c1,v5/c2,v5/v4,v1/v2,v1/c3,c1/v2,c2/c4,v4/c5,v2/v3,c3/v3,c4/v3,c5/v3}
	\draw[arrow over path,very thick](\a)--(\b);


\end{tikzpicture}\\
		\centering Schritt 2
	\end{minipage}\\
	\vspace*{\baselineskip}\\
	\phantom{1}\hfill\begin{minipage}{0.24\textwidth}
		\usetikzlibrary{positioning,patterns,calc,arrows,decorations.markings}

\begin{tikzpicture}[every node/.style={draw,fill,circle},node distance=0cm,scale=0.75]
\tikzset{%
    > = triangle 45,
        arrow over path/.style = {%
        decoration = {%
        markings,
        mark = at position .75 with {\arrow {>};}
        },
        postaction = decorate
    }
}

\foreach \x/\y/\a/\l in {-2/2/180/1,0/0/180/5,0/6/0/3,-0.7/4/180/2,2/2/0/4}{
	\coordinate[label={\a:$\l$}] (v\l) at (\x+6,\y);
	\draw[fill](v\l)circle[radius=4pt];
}
\foreach[count=\c] \x/\y in {5.3/2,6.6/2,4/4,6.6/4,8/4}{
	\coordinate (c\c) at (\x,\y);
	\draw(c\c)circle[radius=4pt];
}
\foreach \a/\b in {v5/v1,v5/c1,v5/c2,v5/v4,v1/v2,v1/c3,c1/v2,c2/c4,v4/c5,v2/v3,c3/v3,c4/v3,c5/v3}
	\draw[arrow over path,very thick](\a)--(\b);


\end{tikzpicture}\\
		\centering Schritt 3
	\end{minipage}
	\hspace*{0.5cm}
	\begin{minipage}{0.24\textwidth}
		\usetikzlibrary{positioning,patterns,calc,arrows,decorations.markings}

\begin{tikzpicture}[every node/.style={draw,fill,circle},node distance=0cm,scale=0.75]
\tikzset{%
    > = triangle 45,
        arrow over path/.style = {%
        decoration = {%
        markings,
        mark = at position .75 with {\arrow {>};}
        },
        postaction = decorate
    }
}

\foreach \x/\y/\a/\l in {-2/2/180/1,0/0/180/5,0/6/0/3,-0.7/4/180/2,2/2/0/4}{
	\coordinate[label={\a:$\l$}] (v\l) at (\x+6,\y);
	\draw[fill](v\l)circle[radius=4pt];
}
\foreach[count=\c] \x/\y in {5.3/2,6.6/2,4/4,6.6/4,8/4}{
	\coordinate (c\c) at (\x,\y);
%	\draw(c\c)circle[radius=4pt];
}
%\foreach \a/\b in {v5/v1,v5/c1,v5/c2,v5/v4,v1/v2,v1/c3,c1/v2,c2/c4,v4/c5,v2/v3,c3/v3,c4/v3,c5/v3}
%	\draw[arrow over path,very thick](\a)--(\b);
\draw[arrow over path,very thick](v5)--(v1);
\draw[arrow over path,very thick](v1)--(c3)--(v3);
\draw[arrow over path,very thick](v5)--(c1)--(v2);
\draw[arrow over path,very thick](v2)--(v3);
\draw[arrow over path,very thick](v1)--(v2);
\draw[arrow over path,very thick](v3)--(c4)--(c2);
\draw[very thick](c2)--(v5);
\draw[arrow over path,very thick](v5)--(v4);
\draw[arrow over path,very thick](v4)--(c5)--(v3);
\end{tikzpicture}\\
		\centering Schritt 4
	\end{minipage}\hfill\phantom{1}
}
\topbreak
\loadTop{04/01-Kreise}
\loadTop{04/02-Zuordnung}
\topbreak
\loadTop{04/03-Minimierung}
\end{TOP}