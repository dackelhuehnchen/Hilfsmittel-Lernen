\subtop{Lagenzuordnung}{-1.725}
Für einen $DAG$ (gerichteter azyklischer Graph) soll eine zulässige Partition der Knotenmenge auf Lagen $L_y$ gefunden werden (alle positiven ganzzahligen $y$-Koordinaten für alle $v\in V$) mit $y(u)<y(v), \forall (u,v)\in A$.
\begin{description}
	\item[Bemerkung (Optimierungskriterium):] \ \\\vspace*{-\baselineskip}
		\begin{itemize}
			\item kompaktes layering
				\begin{itemize}
					\item Höhe $=\#$ Lagen $h$
					\item Breite $=\max|L_i|,~0\leq i\leq h$
				\end{itemize}
			\item \glqq richtig\grqq: Kanten gehen nur über eine Lage, falls nicht werden Dummy-Knoten eingefügt
			\item Minimierung der Anzahl an Dummy-Knoten
		\end{itemize}
	\item[Minimierung der Höhe:] hierfür wird ein \textit{longest-path-layering} verwendet:\\
		weise $v$ der Ebene $L_p$ zu, falls $p$ die Länge des längsten Pfades von einer Quelle ist (in Linearzeit für DAGs, \algo{Longest Path Layering}{\begin{algorithm}[H]
	\SetAlgoVlined
	\SetKwProg{Fn}{Function}{}{end}
	\KwIn{kreisfreier gerichteter Graph $D=(V,A)$}
	\KwOut{Longest path layering von $D$}
	\KwData{$V_{source}$: die Menge aller aktuellen \textit{source}-Knoten ($d(v)=0$)\\$d(v)=deg^{-}(v)$ für alle $v\in V$\\$L_i$: Liste aller Knoten auf dem Layer $i$}
	\BlankLine
	\Begin{
		$i\leftarrow 0$;\\
		\While{$V\neq \emptyset$}{
			\ForEach{$v\in V_{source}$}{
				Zuweisung von $v$ zu $L_i$;\\
				\ForEach{$(v,w)\in A$}{
					$d(w)\leftarrow d(w)-1$;
				}
				löschen von $v$;
			}
			$i\leftarrow i+1$;
		}
	}
\end{algorithm}})
	\item[Minimierung der Höhe bei vorgegebener Breite:] ist $\mathcal{NP}$-schwer (auch mit Einheitsbearbeitungsdauern), äquivalent zu $PRE-SCHED_B\{<\}$
\end{description}
\topbreak
\vspace*{-2\baselineskip}
\subsection{Scheduling mit Vorgängerbedingung}
Aus $n$ Jobs $J_1,\dots,J_n$ mit Bearbeitungsdauern $p_1,\dots,p_n$ und der Bedingung $J_i<J_k$ ($J_i$ muss vor $J_k$ abgeschlossen werden) mit $B$ gleichen Maschinen, soll ein Schedule der Jobs auf den Maschinen gefunden werden, der die Vorgängerbedingungen erfüllt und minimale Bearbeitungsdauer hat.\\
Es gilt $PRE-SCHED_2\{<\}\in \mathcal{P}$\\
Zu entscheiden, ob es zu $n$ Jobs und $T\in\mathbb{N}$ ein Scheduling auf $B$ Maschinen gibt mit $T_i\leq T, \forall 1\leq i\leq n$ ist $\mathcal{NP}$-vollständig.
\vspace*{-1\baselineskip}
\Proof
\vspace*{-.5\baselineskip}
Es wird gezeigt, dass $QLIQUE \propto PRE-SCHED_B\{< 3\}$
\begin{description}
	\item[$CLIQUE$] Gibt es für einen Graphen $G=(V,E)$ und $K\in\mathbb{N},~K\leq |V|$ einen vollständigen Teilgraphen mit mindestens $K$ Knoten?
\end{description}
Definitionen für das Bilden einer Instanz von $CLIQUE$ für $PRE-SCHED_B\{< T\}$ mit gegebenen $G,K$:
	\begin{itemize}[itemsep=-1pt]
		\item $K'=|V|-K$
		\item $L=K\dfrac{K-1}{2}$
		\item $L'=|E|-L$
		\item $B=\max\{K,L+K',L'\}+1$
		\item Dummy-Jobs mit $X_j<Y_r<Z_s$:
			\begin{itemize}
				\item $X_j,~j=1,\dots,B-K$
				\item $Y_r,~r=1,\dots,B-L-K'$
				\item $Z_s,~s=1,\dots,B-L'$
			\end{itemize}
	\end{itemize}
Für jeden Knoten $v$ wird ein Job $J_v$ eingeführt, sowie für jede Kante $e$ ein Job $J_e$.\\
$\Rightarrow$ Gesamtzahl der Jobs $n=3B$.\\
$\Rightarrow PRE-SCHED_B\{< 3\}$ muss alle Slots füllen, bei $3$ Maschinen gilt (falls es ein Clique der Größe $K$ gibt):
	\begin{itemize}[itemsep=-1pt]
		\item im ersten Zeitslot (von oben nach unten) stehen alle $CLIQUE$-Knoten, aufgefüllt mit Dummy-Jobs ($K+(B-K)=B$)
		\item im zweiten Zeitslot stehen (von oben nach unten) alle $CLIQUE$-Kanten, dann alle übrigen Knoten, aufgefüllt mit Dummy-Jobs ($L+(B-L-K')+K'=B$)
		\item im dritten Zeitslot stehen (von oben nach unten) alle übrigen Kanten, aufgefüllt mit Dummy-Jobs ($L'+B-L'=B$)
	\end{itemize}
Wenn keine Clique der Größe $K$ existiert, gilt:
	\begin{itemize}[itemsep=-1pt]
		\item in jedem Schedule  müssen $B$ Jobs in der ersten Zeiteinheit ausgeführt werden
		\item $K$ davon müssen Knoten-Jobs sein
		\item[$\Rightarrow$] die existieren aber nicht
	\end{itemize}
\begin{description}
	\item[Bemerkung:] mit der minimalen Anzahl der Dummy-Knoten kann es effizient durchgeführt werden, aber die Kombination von minimale Höhe und minimaler Anzahl an Dummy Knoten ist auch in $\mathcal{NP}$
	\item[]Ist $h_{OPT}$ minimale Höhe $h$ eines Layerings mit Breite $B$, dann erfüllt die resultierende Höhe aus $LIST-SCHEDULING$ (Jobs stehen in einer Liste, wenn eine Maschine frei ist, wird der erste Job aus der Liste abgearbeitet)
	\[h\leq (2-\frac{1}{B})h_{OPT}\]
	$\Rightarrow$ \glqq $(2-\frac{1}{B})$\grqq-Approximation
\end{description}