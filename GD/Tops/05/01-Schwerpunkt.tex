\topbreak
\subtop{Schwerpunktlayout}{-1.25}
\vspace*{-0.75\baselineskip}
\begin{itemize}[itemsep=-1pt]
	\item um zu verhindern, dass alle Knoten auf einem Punkt liegen, kann eine Menge von Knoten $V_0\subseteq V$ als Nebenbedingung schon feste Positionen $\hat{p}_v={\hat{x}_v\choose \hat{y}_v},~v\in V_0$ erhalten (\textit{Boundary-Knoten})
	\item Umschreibung der Optimalitätsbedingungen (mit $L(G)^{V_0}$ bezeichnet die Laplace'sche Matrix ohne Zeilen und Spalten mit $v\in V_0$):
		\[L(G)^{V_0}\cdot x_{V\setminus V_0}=\left(\sum\limits_{u\in N(v)\cap V_0}\hat{x}_u\right)_{v\in V\setminus V_0}\]
	(entsprechend für $y$-Koordinaten)
	\item ein Layout heißt Schwerpunktlayout, falls
		\[p_v=\left\{\begin{array}{ll}
			\hat{p}_v&v\in V_0\\
			\dfrac{1}{d_G(v)}\sum\limits_{u\in N(v)}p_u&v\notin V_0
		\end{array}\right.\]
	\item es gibt ein eindeutiges Schwerpunktlayout, falls $V_0$ aus mindestens einem Knoten aus jeder Komponente besteht
		\vspace*{-1.5\baselineskip}
		\ProofIdea
		\begin{itemize}
			\item die Lösung ist eindeutig, falls $|L(G)^{v_t}|$ positiv ist
			\item das gilt, falls man aus jeder Komponente mindestens einen Knoten herausnimmt und auf die \textit{Boundary} legt (Matrix-tree-theorem, $|L(G)^{v_t}|=\#$ Spannbäume von $G$)
		\end{itemize}
	\item \algobreak\algo{Schwerpunktlayout (Gauss-Seidel-Iteration)}{\begin{algorithm}[H]
	\SetAlgoVlined
	\SetKwProg{Fn}{Function}{}{end}
	\KwIn{ungerichteter Graph, fixierte Knotenpositionen $\hat{p}_v,~v\in V_0$}
	\KwOut{Knotenpositionen $p_v,~v\in V~(p_v=\hat{p}_v,~\forall v\in V_0)$}
	\BlankLine
	\Begin{
		$p\leftarrow 0$\\
		\ForEach{$v\in V_0$}{
			$p_v\leftarrow\hat{p}_v$
		}
		\While{$p$ ändert sich nicht mehr nennenswert}{
			\ForEach{$v\in V\setminus V_0$}{
				$p_v\leftarrow \dfrac{1}{d_G(v)}\left(\sum\limits_{u\in N(v)}p_u\right)$
			}
		}
	}
\end{algorithm}}
	\item jeder Knoten $v\in V\setminus V_0$ wird in den Schwerpunkt seiner Nachbarn gesetzt, durch die besondere Struktur von $L(G)$ konvergiert das Näherungsverfahren sehr schnell
\end{itemize}