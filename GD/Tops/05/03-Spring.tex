\subtop{Spring Embedder}{-1.61}
\begin{itemize}[itemsep=-1pt]
	\item Realisierung von (1') und (2)
	\item $B(p)$ entspricht einem physikalisches Modell des Graphen, mit Kanten als Federn mit Ideallänge
	\item Potenzialenergie einer Feder der Ideallänge $l$ zwischen $p$ und $q$ beträgt $c\cdot (\|p-q\|-l)^2$ mit $c$ als Dehnbarkeitskonstante
	\item möglich: Einheitslänge festlegen
	\item zwischen jedes Paar an Knoten werden Federn gespannt:
		\begin{description}
			\item[Fall 1:] $\{u,v\}\in E$: Kraft $f_{spring}$ wirkt (anziehend)
			\item[Fall 2:] $\{u,v\}\notin E$: Kraft $f_{rep}$ wirkt (anstoßend)
		\end{description}
\end{itemize}
\topbreak
\vspace*{-2\baselineskip}
\begin{itemize}[itemsep=-1pt]
	\item gestartet wird von einem Initial-Layout
	\item \glqq Let go\grqq~ und iteratives Anwenden der Kräfte
	\item bis das System in einem energieminimalen Zustand ist
	\algo{Spring Embedder}{\begin{algorithm}[H]
	\SetAlgoVlined
	\SetKwProg{Fn}{Function}{}{end}
	\KwIn{ungerichteter Graph, initial layout $p=(p_v)_{v\in V}$, Terminierungskriterium $\epsilon>0$, $K\in \mathbb{N}$}
	\KwData{Kräfte $F_v(t)$ auf dem Knoten $v$ in Iteration $t$, Kühlfaktor $\delta(t)$ für Iteration $t$}
	\KwOut{Positionen $p$ mit geringer Spannung (\glqq low tension\grqq)}
	\BlankLine
	\Begin{
		$t\leftarrow 1$\\
		\While{$t<K$ und $\max\limits_{v\in V}\|F_v(t)\|>\epsilon$}{
			\ForEach{$v\in V$}{
				$F_v(t)\leftarrow \sum\limits_{u~:~\{u,v\}\notin E}f_{rep}(p_u,p_v)+\sum\limits_{u~:~\{u,v\}\in E}f_{spring}(p_u,p_v)$
			}
			\ForEach{$v\in V$}{
				$p_v\leftarrow p_v+\delta(t)\cdot F_v(t)$
			}
			$t\leftarrow t+1$
		}
	}
\end{algorithm}}
	\item Berechnungsmöglichkeiten:
		\begin{description}
			\item[Eades:]\ \\ \vspace*{-\baselineskip}
				\begin{itemize}
					\item $f_{rep}(p_u,p_v)=\dfrac{c_{rep}}{\|p_v-p_u\|^2}\cdot \overrightarrow{p_up_v}$
					\item $f_{spring}(p_u,p_v)=c_{spring}\cdot \log \dfrac{\|p_v-p_u\|}{l}\cdot \overrightarrow{p_vp_u}$
					\item $c_{spring}$ ist die \glqq Verschiebungskonstante\grqq
				\end{itemize}
			\item[Fruchtermann/Reingold:] Zur besseren Annäherung an physikalisches Gesetz wird eine dritte Kraft eingeführt \\ \vspace*{-\baselineskip}
				\begin{itemize}
					\item $f_{rep}(p_u,p_v)=\dfrac{l^2}{\|p_v-p_u\|}\cdot \overrightarrow{p_up_v}$
					\item $f_{attr}(p_u,p_v)=\dfrac{\|p_v-p_u\|^2}{l}\cdot \overrightarrow{p_vp_u}$
					\item $f_{spring}(p_u,p_v)=f_{rep}(p_u,p_v)+f_{attr}(p_u,p_v)$
					\item $f_{spring}$ ähnlich wie Hooke's Gesetz (lineare Federn)
					\item Kraft verschwindet, falls $\|p_u-p_v\|=l$
					\item Kraft ist immer proportional zur Distanz
				\end{itemize}
		\end{description}
\end{itemize}
\begin{description}
	\item[Bemerkungen:]\ \\\vspace*{-\baselineskip}
		\begin{description}[itemsep=-1pt]
			\item[Laufzeit:] pro Iteration
				\begin{itemize}[itemsep=-1pt]
					\item $\BigO(m)$ für $f_{spring}$
					\item $\BigO(n^2)$ für $f_{rep}$
				\end{itemize}
			\item[Vorteile:] \ \\\vspace*{-\baselineskip}
				\begin{itemize}[itemsep=-1pt]
					\item intuitives Konzept (einfach zu verstehen)
					\item einfach zu implementieren
					\item gute Layouts für kleine / spärliche Graphen (zeigt Strukturen/Symmetrien)
					\item sehr flexible
				\end{itemize}
			\item[Nachteile:] \ \\\vspace*{-\baselineskip}
				\begin{itemize}[itemsep=-1pt]
					\item endet nicht unbedingt in einem stabilen System / kann in schlechten lokalen Minima enden
					\item abhängig vom Initial-Layout
				\end{itemize}
		\end{description}
\end{description}