\begin{TOP}{Allgemeines}
\subtop{Modellierung}{-1.225}
\begin{itemize}[itemsep=-1pt]
	\item Informationsvisualisierung
	\item Informationsgehalt
		\begin{itemize}
			\item kein \glqq Rezept\grqq~ zur Darstellung von Graphen
			\item möglicherweise Attribute gegeben
		\end{itemize}
	\item Design der Visualisierung
		\begin{itemize}
			\item festgelegte Regeln müssen beachtet werden
			\item Formalisierung:
				\begin{description}
					\item[Repräsentation:] Menge von graphischen Elementen der Klassen \textit{Punkte, Linien, Flächen, Körper}
					\item Punkt-Strich: Standardrepräsentation
					\item Knoten$\rightarrow$Flächen (Kanten implizit): Inklusionsrepräsentation
					\item Knoten$\rightarrow$Flächen  (Kanten implizit): Berührungsrepräsentation
					\item Knoten$\rightarrow$Linien (Kanten implizit): Intervallrepräsentation
					\item Knoten$\rightarrow$Linien (Kanten$\rightarrow$Linien): Sichtbarkeitsrepräsentation
					\item[graphische Variablen:] $x-,y-,z-$Koordinaten, Größe, Form, Orientierung, Farbe, Helligkeit, Textur, Transparenz
					\item[Nebenbedingungen:] Eigenschaften, die auf jeden Fall erfüllt sein müssen
					\item[Gütebedingungen:] Eigenschaften, die so gut wie möglich erfüllt sein sollten
				\end{description}
		\end{itemize}
	\item Algorithmen zum Realisierung der Layouts
		\begin{itemize}
			\item Ziel (durch Bedingungen): getreues und lesbares Layout
			\item nicht nur von theoretischem Interesse, da sich seine Komplexität unmittelbar auf Anwendbarkeit und seine Besonderheiten und eventuelle Artefakte auf die Verlässlichkeit der Interpretation / Informationsvisualisierung auswirken
		\end{itemize}
\end{itemize}
\end{TOP}