\begin{TOP}{Glyphen}
\vspace*{-1.5\baselineskip}
\begin{itemize}
	\item unterscheiden sich von Icons, Indizes und Symbolen
	\item Mappings:
		\begin{description}
			\item[Many-to-One] alle Attribute werden auf die gleiche visuelle Variable gemappt
			\item[One-to-One] alle Attribute werden auf unterschiedliche visuelle Variablen gemappt
			\item[One-to-Many] manche Attribute werden auf mehrere visuelle Variablen gemappt
		\end{description}
	\item Layout-Techniken (Positionierung der Datenpunkte):
		\begin{itemize}
			\item strukturgetrieben
				\begin{itemize}
					\item Reihenfolge
					\item Hierarchien
					\item Netzwerke
				\end{itemize}
			\item datengetrieben
				\begin{itemize}
					\item Positionierung der Daten auf Grundlage der expliziten Werte
				\end{itemize}
		\end{itemize}
	\item Anwendungsgebiete
		\begin{itemize}
			\item Linguistik
			\item Sport-Analyse
		\end{itemize}
\end{itemize}
\end{TOP}