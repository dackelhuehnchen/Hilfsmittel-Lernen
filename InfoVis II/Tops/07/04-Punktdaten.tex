\subtop{Punktdaten}{-1.08}
\begin{description}
	\item[DotMaps] \ \\\vspace*{-\baselineskip}
		\begin{itemize}
			\item viel Overlap
			\item Kreis repräsentiert die räumliche Position eines Events
			\item keine Skalierung
		\end{itemize}
	\item[PixelMaps] \ \\\vspace*{-\baselineskip}
		\begin{itemize}
			\item Overlap-frei
			\item Punkt so nah wie möglich an Originalposition
			\item skaliert Karte neu
		\end{itemize}
	\item[visuelle Explorationsziele] \ \\\vspace*{-\baselineskip}
		\begin{itemize}
			\item kein Overlap für \glqq normale\grqq\ Bildschirmauflösung
				\begin{itemize}
					\item ergibt effektive Visualisierung
					\item kein Informationsverlust
				\end{itemize}
			\item Clustering
				\begin{itemize}
					\item generelle geo-spatial Relationen sichtbar
					\item geo-Pattern werden sichtbar
				\end{itemize}
			\item Positionserhaltung
				\begin{itemize}
					\item Pixel-Zusammenhänge erkennbar
					\item kleine Cluster sichtbar
				\end{itemize}
		\end{itemize}
	\item[effiziente Implementierung] Neuskalierung der Kartenregionen, damit die 3D-Punkt-Wolken besser auf den Bildschirm passen
\end{description}