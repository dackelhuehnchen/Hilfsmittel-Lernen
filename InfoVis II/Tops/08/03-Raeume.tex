\subtop{Räume}{-1.58}
\begin{description}
	\item[Bildschirmraum (Pixel)] \ \\\vspace*{-\baselineskip}
		\begin{itemize}
			\item Pixel-Regionen werden vergrößert/verkleinert für detailreichere Darstellung
			\item (semantisches) Zoomen
		\end{itemize}
	\item[Datenwertraum (Multivariate Daten)] Transformationen werden auf Dimensionalität angewendet
	\item[Datenstrukturraum (Teile der Datenorganisation)] \ \\\vspace*{-\baselineskip}
		\begin{itemize}
			\item Auswahl von Knoten in Hierarchie
			\item Hervorheben in allen relevanten Knoten durch Benutzerinteraktion (nicht nur ein Knoten auf einmal)
			\item resamplen der Datenpunkte bei zoomen
		\end{itemize}
	\item[Attributraum (Teile/Eigenschaften von grafischen Einheiten)] \ \\\vspace*{-\baselineskip}
		\begin{itemize}
			\item modifizieren von einem oder mehreren Attributen in grafischen Objekten
			\item Kontrast anpassen
		\end{itemize}
\end{description}
\topbreak
\vspace*{-\baselineskip}
\begin{description}
	\item[Objektraum (3D-Oberflächen)] \ \\\vspace*{-\baselineskip}
		\begin{itemize}
			\item Verzerrung eines Objektes durch Auswahl von Daten
			\item Bsp.: Parallel-Coordinate wird auf Wände gemappt, eine Wand als Hervorhebung ausgewählt, alle anderen verzerrt
		\end{itemize}
	\item[Visualisierungsraum] \ \\\vspace*{-\baselineskip}
		\begin{itemize}
			\item modifizieren des unterliegenden Strukturelementes der Visualisierung
			\item Bsp.: Anordnung der Parallel-Coordinate Dimensionen
		\end{itemize}
\end{description}