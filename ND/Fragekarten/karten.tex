\documentclass[a8paper,print]{kartei-nd}
\usepackage{nd-karten}
%\usepackage{nd}
%\usepackage{ngerman}
\begin{document}

\setlength{\parindent}{0em}

\card{Was ist das Relative-Agreement-Modell}{1}{
	\begin{itemize}[nosep,leftmargin=1em,labelwidth=*,align=left]
	\item $n$ Agenten
	\item Meinung und Unsicherheit
	\item (gerichteter) Kommunikationskanal
	\item Kommunikation \"uber $(i,j)$
	\item Aktualisierungsregel:
		\begin{itemize}[nosep,leftmargin=1em,labelwidth=*,align=left]
			\item je mehr \"Uberzeugung und Zustimmung, desto h\"oher der Einfluss
			\item konvexe Kombination mit $x_j \underset{\infty}{\rightarrow} x_i$
		\end{itemize} 
\end{itemize}
}
\card{Was ist das Relative-Agreement-Modell mit Extremisten}{1}{
	\begin{itemize}[nosep,leftmargin=1em,labelwidth=*,align=left]
	\item positive und negative Extremisten
	\item zentrales Clustering
	\item Bipolarisierung
	\item einfache Polarisierung
\end{itemize}
}
\card{Wie sieht das zentrale Clustering aus}{1}{
	\noindent\inclPic{1-central}{\textwidth}
}
\card{Wie sieht die Bipolarisierung aus}{1}{
	\noindent\inclPic{1-bipol}{\textwidth}
}
\card{Wie sieht die einfache Polarisierung aus}{1}{
	\noindent\inclPic{1-single}{\textwidth}
}
\card{Warum gilt $h_{ij}\leq h'_{ij}$ wenn $h'_{ij}$ die \"Uberlappung nach einer Interaktion zwischen $i$ und $j$ ist}{1}{
	\begin{itemize}[nosep,leftmargin=1em,labelwidth=*,align=left]
	\item ergibt sich aus der Aktualisierungsregel
	\item diese ist eine konvexe Kombination der\\ \"Uberzeugungen und Abweichungen
\end{itemize}
}
\card{Was sind Netzwerkdaten}{2.1}{
	\begin{itemize}[nosep,leftmargin=1em,labelwidth=*,align=left]
	\item Daten referenzieren auf Einheiten
	\item gemeinsames Attribut f\"ur alle Elemente (numerisch)
	\item Einheiten sind Dyaden
\end{itemize}
}
\card{Was ist der Unterschied zwischen dyadischer und Netzwerkanalyse}{2.1}{
	\begin{itemize}[nosep,leftmargin=1em,labelwidth=*,align=left]
	\item dyadisch: unabh\"angige Dyaden
	\item Netzwerk: abh\"angige Dyaden
\end{itemize}
}
\card{Welche zeitabh\"angigen Datenformen gibt es}{2.1}{
	\begin{itemize}[nosep,leftmargin=1em,labelwidth=*,align=left]
	\item Panel-Daten: alle Elemente zu mindestens zwei Zeitpunkten
	\item Zeitreihendaten: ein Element zur gesamten Zeit (Ver\"anderung)
	\item Querschnittsdaten: alle Elemente zu einem Zeitpunkt
	\item Ereignisdaten: ein Element mit einem Attributwert zu einem Zeitpunkt
	\item h\"aufig: Umwandlung von Ereignisdaten in Panel-Daten
\end{itemize}
\vspace*{-0.5cm}
}
\card{Was ist die Netzwerkdarstellung}{2.1}{
	\glqq wie\grqq\ das Netzwerk dargestellt wird
}
\card{Was ist ein Netzwerk}{2.1}{
	\begin{itemize}[nosep,leftmargin=1em,labelwidth=*,align=left]
	\item Attribute auf Interaktionsbereich (Kanten, Netzwerkattribute)
	\item Attribute auf Elementen (kann eine leere Menge sein, Verhaltensattribute)
\end{itemize}
}
\card{Was ist ein Interaktionsbereich}{2.1}{
	\begin{itemize}[nosep,leftmargin=1em,labelwidth=*,align=left]
	\item symmetrische Relation $I\subseteq A\times A$
	\item im Netzwerk (Darstellung als Graph) die Kanten
\end{itemize}
}
\card{Wie kann ein Netzwerk repr\"asentiert werden}{2.1}{
	\begin{itemize}[nosep,leftmargin=1em,labelwidth=*,align=left]
	\item Graph
	\item Matrix
	\item Relation
\end{itemize}
}
\card{Was ist der Zugeh\"origkeitsbereich}{2.1}{
	Eine Relation $A\times S$ mit disjunkten Mengen $A$ und $S$
}
\card{Was ist ein Two-Mode Netzwerk}{2.1}{
	\begin{itemize}[nosep,leftmargin=1em,labelwidth=*,align=left]
	\item bipartiter Graph mit (m\"oglicherweise) unter-\\schiedlichen Grundmengen
	\item Attribute auf Zugeh\"origkeitsbereich
	\item Attribute auf $A$ und $A$ (k\"onnen auch leere Mengen sein)
\end{itemize}
}
\card{Was ist eine One-Mode-Projektion}{2.1}{
	\begin{itemize}[nosep,leftmargin=1em,labelwidth=*,align=left]
	\item $I_1=A\times A$ und $I_2=S\times S$
	\item $X=\R^{\|A\|.\|S\|}$
	\item $XX^T$ und $X^TX$ sind one-mode-Projektionen
	\item Projektion von $A\times S$ auf $A\times A$ oder $S\times S$
	\item Kanten geben an, ob zwei Agenten in einem Two-Mode-Netzwerk eine Kante zum selben Knoten haben
\end{itemize}
}
\card{Was sind Zeitabh\"angige Netzwerke}{2.1}{
	\begin{itemize}[nosep,leftmargin=1em,labelwidth=*,align=left]
	\item Attribute auf $I$ und Zugeh\"origkeitsbereichen
	\item Ver\"anderung der Attribute im Laufe der Zeit
\end{itemize}
}
\card{Welche zeitabh\"angige Netzwerke betrachten wir}{2.1}{
	\begin{itemize}[nosep,leftmargin=1em,labelwidth=*,align=left]
	\item zeit-diskrete
	\item Zeitschritte sind diskret ($\N$)
\end{itemize}
}
\card{Welches dynamische Verhalten haben Netzwerke}{2.1}{
	\begin{itemize}[nosep,leftmargin=1em,labelwidth=*,align=left]
	\item Einzelnetzwerkattribut, $x:I\rightarrow R$
	\item unendliche Sequenz $(x_t)_{t\in \N}$ (identische Kopien von $x$)
	\item alle m\"oglichen $R$-Sequenzen sind ein Prozess
	\item einzelne Sequenz ist Trajektorie
	\item dynamisches $F$ ist ein Mechanismus zum\\ Ausw\"ahlen einer Trajektorie (iterative Abbildung)
\end{itemize}
}
\card{Wie h\"angen \glqq Prozess\grqq, \glqq Trajektorie\grqq, \glqq Dynamik\grqq\ und \glqq iterierte Abbildung\grqq\ zusammen}{2.1}{
	\scalebox{0.43}{\inclPgf{2-iteratedMap}}
}
\card{Was ist eine Dynamik}{2.1}{
	Mechanismus zum W\"ahlen einer Trajektorie
}
\card{Was ist ein Prozess}{2.1}{
	Menge aller m\"oglichen Sequenzen
}
\card{Was ist eine Trajektorie}{2.1}{
	eine spezifische Sequenz
}
\card{Was ist der Unterschied zwischen einer Dynamik und einer iterierten Abbildung}{2.1}{
	eine iterierte Abbildung ist eine Dynamik ohne\\Ged\"achtnis
}
\card{Was ist eine iterative Netzwerkabbildung}{2.2}{
	$F:(I\rightarrow R) \rightarrow (I\rightarrow R), x\mapsto F(x)$
}
\card{Was ist ein Orbit}{2.2}{
	\begin{itemize}[nosep,leftmargin=1em,labelwidth=*,align=left]
	\item grunds\"atzliches Konzept von iterierten Abbildungen
	\item totale Abbildung $F: J\rightarrow J$
	\item Sequenz $(z_0,F(z_0),\dots, F^k(z_0),\dots)$
	\item Orbits sind disjunkt oder ab bestimmtem Zeitpunkt gleich
	\item Orbits werden im Phasenraum zusammengefasst
\end{itemize}
}
\card{Warum sind Orbits entweder disjunkt oder ab einem bestimmten Zeitpunkt gleich}{2.2}{
	ergibt sich aus der Eindeutigkeit von $F$
}
\card{Was ist ein Fixpunkt}{2.2}{
	$F(x)=x$
}
\card{Wann ist ein Zustand in einem Orbit periodisch}{2.2}{
	\begin{itemize}[nosep,leftmargin=1em,labelwidth=*,align=left]
	\item es gibt ein $k\in\N_+$ mit $F^k(x)=x$
	\item kleinstes $k_0\neq 0$ mit dieser Eigenschaft ist die periodische Ordnung von $x$
\end{itemize}
}
\card{Wann ist ein Zustand eines Orbits transient}{2.2}{
	wenn der Zustand nicht periodisch ist
}
\card{Was ist ein Attraktor}{2.2}{
	\begin{itemize}[nosep,leftmargin=1em,labelwidth=*,align=left]
	\item Teilmenge des Orbits
	\item beinhaltet die Periode
	\item f\"ur zwei Zust\"ande sind die Attraktoren entweder gleich oder disjunkt
\end{itemize}
}
\card{Was ist ein \glqq Basin of attraction\grqq}{2.2}{
	Attraktor mit inzidenten B\"aumen
}
\card{Was ist ein Zustandsgraph}{2.2}{
	\begin{itemize}[nosep,leftmargin=1em,labelwidth=*,align=left]
	\item Assoziation von der Abbildung $F:J\rightarrow J$ zu gerichtetem Graphen
	\item Kantenmenge $E=\{(x,F(x))|x\in J)\}$
	\item eindeutig zerlegbar in
		\begin{itemize}[nosep,leftmargin=1em,labelwidth=*,align=left]
			\item disjunkte Kreise (Attraktoren)
			\item disjunkte B\"aume (inzident zu genau\\
			einem Kreis, transiente Zust\"ande)
		\end{itemize}
\end{itemize}
}
\card{Was sind stochstisch iterierte Netzwerkabbildungen}{2.2}{
	\begin{itemize}[nosep,leftmargin=1em,labelwidth=*,align=left]
	\item iterierte Zufallsabbildungen
	\item $F=\{f_\omega|\omega \in\Omega\}, f_\omega : J\rightarrow J$
	\item $\Omega$ ist Wahrscheinlichkeitsraum
	\item $\mu$ ist die Wahrscheinlichkeitsverteilung auf $\Omega$
	\item f\"ur $x\in J$ wird $\omega$ gem\"aß $\mu$ gew\"ahlt und zu $f_\omega(x)$ gegangen
	\item Sequenz der Zufallsvariablen: $X_n=f_{\omega_n}(X_{n-1})$
\end{itemize}
}
\card{Welche Zufallsquellen gibt es in Dynamiken}{2.2}{
	\begin{itemize}[nosep,leftmargin=1em,labelwidth=*,align=left]
	\item Fehlen von Informationen auf Parametern oder Attributwerten
	\item Simulationsverwendung
\end{itemize}
}
\card{Was ist ein Markovkette}{2.2}{
	\begin{itemize}[nosep,leftmargin=1em,labelwidth=*,align=left]
	\item endliche Folge von Zufallsvariablen, $X_t:\Omega \rightarrow J$
	\item \"Uberf\"uhrungsmatrix: $p_{ij} = \mathbb{P}[X_{t+1}=j|X_t=i]$
\end{itemize}
}
\card{Wann ist eine Markovkette zeithomogen}{2.2}{
	\begin{itemize}[nosep,leftmargin=1em,labelwidth=*,align=left]
	\item wenn die Zeit keine Rolle spielt
	\item $\mathbb{P}[X_{t+1}=j|X_t=i,X_{t-1}=z_{t-1},$\\\hspace*{01.6cm}$\dots,X_0=z_0] = p_{ij}$
\end{itemize}
}
\card{Wie ist die Verteilung einer Markovkette}{2.2}{
	\begin{itemize}[nosep,leftmargin=1em,labelwidth=*,align=left]
	\item $q^{(t)} = (q_1^{(t)},\dots,q_n^{(t)})$
	\item $q_i^{(t)} = \mathbb{P}[X_t=i]$
	\item $q^{(0)}$ ist die Initialverteilung
	\item $q^{(t)}=q^{(0)}\cdot P^t$
\end{itemize}
}
\card{Wann ist eine Verteilung einer Markovkette station\"ar}{2.2}{
	$\pi\cdot P=\pi$
}
\card{Wann ist ein Zustand absorbierend}{2.2}{
	$p_{ij}=0$ f\"ur alle $j\neq i$
}
\card{Wann ist ein Zustand transient}{2.2}{
	$\mathbb{P}[\exists t>0 : X_t=i | x_0=i]<1$
}
\card{Wann ist ein Zustand wiederkehrend}{2.2}{
	$\mathbb{P}[\exists t>0 : X_t=i | x_0=i]=1$
}
\card{Wann ist eine Markovkette nicht reduzierbar}{2.2}{
	es gibt ein $t>0$ sodass $(P^t)_{ij}>0$
}
\card{Was ist die Periode einer Markovkette}{2.2}{
	$d(i)=ggt\{t>0|(P^t)_{ii}>0\}$
}
\card{Wann ist eine Markovkette aperiodisch}{2.2}{
	\begin{itemize}[nosep,leftmargin=1em,labelwidth=*,align=left]
	\item $d(i)=1$ Zustand $i$ ist aperiodisch
	\item $d(i)=1$ f\"ur alle $i$, dann ist die Kette aperiodisch
\end{itemize}
}
\card{Was sind die Eigenschaften einer ergodischen Markovkette}{2.2}{
	\begin{itemize}[nosep,leftmargin=1em,labelwidth=*,align=left]
	\item nicht reduzierbar
	\item aperiodisch
\end{itemize}
}
\card{Was ist der Grenzwert der Verteilung einer Markovkette}{2.2}{
	\begin{itemize}[nosep,leftmargin=1em,labelwidth=*,align=left]
	\item station\"are Verteilung $\pi$
	\item unabh\"angig von $q^{(0)}$
\end{itemize}
}
\card{Aus was besteht ein Spiel mit Nutzen}{3.1}{
	\begin{itemize}[nosep,leftmargin=1em,labelwidth=*,align=left]
	\item Tupel $(A,(S_1,\dots,S_n),(u_1,\dots,u_n))$
	\item Menge von Agenten ($A$)
	\item Menge von Strategien f\"ur jeden Agenten ($S$)
	\item Menge von Nutzenfunktionen ($u$)
	\item Vektornutzenfunktion $u:S\rightarrow\R^n$
\end{itemize}
}
\card{Was ist ein einmalieges nicht-kooperatives Spiel}{3.1}{
	\begin{itemize}[nosep,leftmargin=1em,labelwidth=*,align=left]
	\item Agenten w\"ahlen unabh\"angig von einander und ohne Wissen von den Entscheidungen der anderen, ihre Strategie
	\item Ergebnis ist das Strategieprofil $s$
	\item Auswertung von $s$ f\"ur jeden Agenten mittels der Nutzenfunktion $u_i$
\end{itemize}
}
\card{Was ist ein Nashgleichgewicht}{3.1}{
	f\"ur alle $s_i\in S_i$ und alle Agenten gilt: $s^{*}$ ist ein NG $\Leftrightarrow u_i(s_i^{*},s_{-i})\geq u_i(s_i,s_{-i})$
}
\card{Was ist die beste Antwort f\"ur einen einzelnen Agenten}{3.1}{
	\begin{itemize}[nosep,leftmargin=1em,labelwidth=*,align=left]
	\item Funktion $\beta_i : S_{-i}\rightarrow \mathcal{P}(S_i)$
	\item $\beta_i(s_{-i}) = \{s_i\in S_i| u_i(s_i,s_{-i})$\\
	\hspace*{2.5cm}$=\max\limits_{s'_i\in S_i}(u_i(s'_i,s_{-i}))\}$
\end{itemize}
}
\card{Was ist die beste Antwort f\"ur alle Agenten}{3.1}{
	\begin{itemize}[nosep,leftmargin=1em,labelwidth=*,align=left]
	\item $\beta : S\rightarrow \bigtimes\limits_{i=1}^n \mathcal{P}(S_i)$
	\item $\beta(s) = \beta_1(s_{-1})\times\dots\times \beta_n(s_{-n})$
\end{itemize}
}
\card{Wie ist das Nashgleichgewicht f\"ur die beste Antwort definiert}{3.1}{
	$s^{*}$ ist NG $\Leftrightarrow s^{*}\in\beta(s^{*})$
}
\card{Was ist das \glqq connections\grqq-Modell}{3.1}{
	\vspace*{-2\baselineskip}\begin{itemize}[nosep,leftmargin=1em,labelwidth=*,align=left]
	\item statisches Formierugnsmodell
	\item Menge von Agenten mit Interaktionsbereich (alle Kanten ohne Schleifen)
	\item $x:I\rightarrow \{0,1\}$
	\item $G=G(x)$ ist der Graph des Netzwerkes
	\item Auszahlung f\"ur jeden Agenten ist $\delta^{d(i,j)}$ f\"ur jede Verbindung, wobei der Wert $0$ ist falls der Abstand unendlich ist
	\item Kosten $c>0$ f\"ur die Aufrechterhaltung von direkten Verbindungen
	\item Nutzenfunktion $u_i(G)=\sum\limits_{i\neq j} \delta^{d(i,j)}-\sum\limits_{(i,j)\in E(G)} c$
\end{itemize}
\vspace*{-0.5cm}
}
\card{Wann ist der Graph zu einem connections-Modell stabil}{3.1}{
	\begin{itemize}[nosep,leftmargin=1em,labelwidth=*,align=left]
	\item Nutzenfunktion von $G$ ist gr\"oßer oder gleich der Nutzenfunktion von $G$ ohne die Kante $(i,j)$, f\"ur alle $i\in A$ und alle $(i,j)\in E$
	\item ist die Nutzenfunktion von $G$ plus der Kante $(i,j)$ abz\"uglich beliebig vieler Kanten ausgehend von $i$ oder $j$ gr\"oßer als die Nutzenfunktion von $G$, dann ist bewirkt die obige Ver\"anderung einen Nachteil f\"ur $j$
\end{itemize}
}
\card{Was ist das dynamische Netzwerkformierungsmodell}{3.1}{
	\vspace*{-2\baselineskip}\begin{itemize}[nosep,leftmargin=1em,labelwidth=*,align=left]
	\item Ausgangssituation: $G$ ist leer
	\item diskrete Zeitschritte ($T$), Sequenz von Graphen $(G_t)_{t\in T}$
	\item Agenten sind myopisch; treffen Entscheidungen als bessere Antwort, wenn m\"oglich; keine Beachtung von m\"oglichen weiterf\"uhrenden Nachteilen
	\item gleichm\"aßiges und zuf\"alliges W\"ahlen einer Dyade zu jedem Zeitpunkt
		\tiny{\begin{itemize}[nosep,leftmargin=1em,labelwidth=*,align=left]
			\item Dyade ist Kante im Graph: beide Teilnehmer k\"onnen unabh\"angig von einander die Verbindung kappen
			\item Dyade ist keine Kante im Graphen: beide Teilnehmer m\"ussen der Verbindung zustimmen; beide k\"onnen beliebig viele andere Verbindungen kappen
		\end{itemize}}
\end{itemize}
\vspace*{-.5cm}
}
\card{Was sind strukturelle L\"ocher}{3.2}{
	\begin{itemize}[nosep,leftmargin=1em,labelwidth=*,align=left]
	\item Verbindungen begr\"unden Vorteile und Kosten
	\item redundaten Verbindungen haben weniger Vorteile mit gleichen Kosten
	\item strukturelle L\"ocher sind Regionen in sozialen\\ Netzwerken, wo das Bilden von Verbindungen\\
	fehlgeschlagen ist
\end{itemize}
}
\card{Wie sieht das Modell von strukturellen L\"ochern aus}{3.2}{
	\vspace*{-2\baselineskip}\begin{itemize}[nosep,leftmargin=1em,labelwidth=*,align=left]
	\item strategisches Spiel
	\item Menge von Agenten, Strategien (Nachbarn)
	\item Nutzenfunktion\\
	$u_i(s_1,\dots,s_n)=\alpha_0\left(\|s_i\|+\|\{j | (j,i)\in S_j\}\|\right)$\\
	\hspace*{1cm}$+ \sum\limits_{(i,j),(i,k)\in s_i, j\neq k}\beta(r_{j,k})-\sum\limits_{(i,j)\in s_i}c_{i,j}$
	\item $r_{j,k}$ ist die Anzahl von L\"ange-2-Pfaden, wobei der Wert $0$ ist, falls es eine Verbindung (in beliebiger Richtung) zwischen $j$ und $k$ gibt
	\item $\beta$ ist eine fallende, nicht-negative Funktion, die den Vorteil angibt, den ein Agent hat, der in der Mitte von $r$ L\"ange-2-Pfaden liegt
\end{itemize}
\vspace*{-0.5cm}
}
\card{Wie identifiziert man Gleichgewichtsgraphen}{3.2}{
	\vspace*{-2.2\baselineskip}\begin{itemize}[nosep,leftmargin=1em,labelwidth=*,align=left]
	\item Unterklassen von Gleichgewichtsgraphen sind multipartite Graphen, wobei von allen Knoten aus $V_i$ eine Kante zu allen Knoten $V_j$ existiert, falls $j<i$
	\item $n$ ist die Anzahl der Agenten, $k$ die Anzahl der Parteien
	\item Ver\"anderung des Nutzens, durch L\"oschen aller Kanten von $v$ ist $B(n,k)=k(\alpha_0-1)+{k\choose 2}\cdot \beta(n-k)$
	\item $B(n,k)\geq 0\Rightarrow$ Knoten beh\"alt alle Kanten
	\item $B(n,k)< 0\Rightarrow$ Knoten l\"oscht alle Kanten zu der anderen Menge
	\item f\"ur alle $n$ gibt es ein $k$, sodass $G_{n,k}$ ein Nashgleichgewicht ist
\end{itemize}
\vspace*{-1cm}
}
\card{Was ist ein ordinales Potentialspiel}{3.3}{
	\begin{itemize}[nosep,leftmargin=1em,labelwidth=*,align=left]
	\item es gibt eine ordinale Potentialfunktion
	\item Nutzenfunktionsdifferenz hat das gleiche Vorzeichen wie Potentialfunktionsdifferenz
	\item $s^{*}$ ist ein NG $\Leftrightarrow P(s^{*})\geq P(s_i,s^{*}_{-i})$
	\item alle ordinalen Potenzialspiele haben ein Nashgleichgewicht
\end{itemize}
}
\card{Warum hat jedes endliche ordinale Potentialspiel ein Nashgleichgewicht}{3.3}{
	es gibt ein Maximum der ordinalen Potentialfunktion
}
\card{Was ist ein Potentialspiel}{3.3}{
	\begin{itemize}[nosep,leftmargin=1em,labelwidth=*,align=left]
	\item es gibt eine Potentialfunktion
	\item Nutzenfunktionsdifferenz ist gleich der Potentialfunktionsdifferenz
	\item $I(\Gamma,p)=\sum\limits_{k=1}^n(u_{i_k}(s^k)-u_{i_k}(s^{k-1}))$
	\item $\Gamma$ ist ein Potentialspiel, falls $I(\Gamma,p)=0$ f\"ur alle endlichen, geschlossenen Pfade / endlichen einfachen geschlossenen Pfade / endlichen einfachen geschlossenen Pfade der L\"ange 4
\end{itemize}
}
\card{Warum reicht es nur Pfade der L\"ange 4 zu betrachten}{3.3}{
	es ist m\"oglich l\"angere Kreise in mehrere Kreise der L\"ange 4 zu zerlegen; das Ergebnis ist dasselbe
}
\card{Wie sieht die Orbit-basierende Charakterisierung von Potentialspielen aus}{3.3}{
	\begin{itemize}[nosep,leftmargin=1em,labelwidth=*,align=left]
	\item Spiel mit Nutzen
	\item $(s^t)_{t\in T}$ (endliche) Sequenz von Strategieprofilen
	\item $(s^t)_{t\in T}$ ist ein Verbesserungspfad, wenn es f\"ur jedes $t$ ohne $0$ ein $i\in A$ gibt, sodass $i$ im Schritt von $t$ seiner Strategie abweicht
	\item ist jeder Verbesserungspfad endlich, dann hat das Spiel die \glqq endliche Verbesserungseigenschaft\grqq\\ (FIP)
\end{itemize}
}
\card{Wann ist ein Spiel endlich und nicht-entartet}{3.3}{
	\begin{itemize}[nosep,leftmargin=1em,labelwidth=*,align=left]
	\item nicht-entartet: es gibt kein $i$ mit\\ $u_i(s_i,s_{-i})=u_i(s'_i,s_{-i})$
	\item ist erf\"ullt, falls das Spiel die FIP hat und ein ordniales Potentialspiel ist
\end{itemize}
}
\card{Was ist das \glqq congestion\grqq-Modell}{3.3}{
	\begin{itemize}[nosep,leftmargin=1em,labelwidth=*,align=left]
	\item Menge von Agenten und \textit{facilities}
	\item Strategiemengen $S_i=\mathcal{P}(F)$
	\item Kostenfunktion f\"ur jede \textit{facility} $f$:\\ $\omega_f : \{1,\dots,n\}\rightarrow \R$ f\"ur alle Agenten gleich
	\item $\omega_f(k)$ sind die Kosten, falls $k$ Agenten die \textit{facility} $f$ benutzen
\end{itemize}
}
\card{Was sind \glqq congestion\grqq-Spiele}{3.3}{
	\begin{itemize}[nosep,leftmargin=1em,labelwidth=*,align=left]
	\item $u_i(s)=\sum\limits_{f\in s_i}\omega_f(\sigma(s))$
	\item $\sigma_f(s)=\|\{i\in A | f\in s_i\}\|$
	\item jedes congestion-Spiel ist ein Potentialspiel
	\item jedes Potentialspiel ist isomorph zu einem\\ congestion-Spiel
	\item Rosethal-Potential: $P(s) =\sum\limits_{f\in\bigcup\limits_{i\in A}s_i}\sum\limits_{k=1}^{\sigma_f(s)}\omega_f(k)$
\end{itemize}
}
\card{Wie entsteht eine Meinungsformierung}{4.1}{
	\scalebox{0.4}{\inclPgf{3-needCognition}}
	\vspace*{-0.5cm}
}
\card{Wie wird ein Konsens erreicht}{4.2}{
	\vspace*{-2\baselineskip}\begin{itemize}[nosep,leftmargin=1em,labelwidth=*,align=left]
	\item es soll gelten $o_1=\dots=o_n$
	\item Meinungspools: iterierte Abb. $P:R^A\rightarrow R^A$
	\item $P(o)$ ist das aktualisierte Meinungsprofil
	\item linearer Meinungspool: $o_i^{(k+1)}=\sum_{j=1}^n w_{ij}\cdot o_j^{(k)}$
	\item Meinungspools sind stochstisch
	\item die Iterierung von $P$ ergibt den Orbit auf $o^{(0)}$
	\item $o^{(k)}=P^k\cdot o^{(0)}$
	\item $o^{*}$ ist ein Konsens $\Leftrightarrow$ f\"ur alle $i\in A$ ten gilt $\lim_{k\rightarrow\infty}o_i^{(k)}=o^{*}$
	\item Konsens existiert, wenn es einen Vektor $\pi=(\pi_1,\dots,\pi_n)$ mit $\lim_{k\rightarrow\infty}p_{ij}^{(k)}=\pi_{j}$ f\"ur alle $i\in A$ gibt
\end{itemize}
\vspace*{-1cm}
}
\card{Welche Bedingungen gibt es an die Funktion $P:R^A\rightarrow R^A$ f\"ur Konvergenz}{4.2}{
	\begin{itemize}[nosep,leftmargin=1em,labelwidth=*,align=left]
	\item Werte in der Matrix entsprechen $w_{ij}$
	\item $P$ ist eine Markovkette
	\item $p_{ij}$ ist die Wahrscheinlichkeit, dass $i$ die Meinung von $j$ \"ubernimmt
\end{itemize}
}
\card{Was ist ein hinreichendes Kriterium zum Erreichen eines Konsens}{4.2}{
	es gibt ein $m\in\N_{+}$ sodass jedes Element in mindestens einer Spalte einen Wert gr\"oßer $0$ stehen hat
}
\card{Was ist das Friedkin-Johnson-Modell}{4.3}{
	\begin{itemize}[nosep,leftmargin=1em,labelwidth=*,align=left]
	\item Informationen kommen auch von außen
	\item lineare (konvexe) Kombination von exo- und endogenen Informationen
	\item Hompgenit\"at der Werte wird angenommen zur Vereinfachung des Modells
\end{itemize}
}
\end{document}