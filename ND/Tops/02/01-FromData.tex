\subtop{Netzwerke aus Daten}{-1.45}
\subsection{Netzwerkdaten}
	\begin{description}
		\item[Daten] referenzieren auf Einheiten bzw. Beobachtungseinheiten
			\begin{itemize}
				\item $A$ ist die Menge von Elementen
				\item für alle Elemente aus $A$ gibt es ein gemeinsames Attribut $x_i$, mit $x:A\rightarrow R$
				\item $R$ ist der Umfang von $x$, normalerweise numerische Attribute
				\item $A$ ist die Domäne von $x$
			\end{itemize}
		\item[Einheiten] müssen nicht atomar sein (dyadische Daten)
			\begin{itemize}
				\item Dyade ist ein Paar von Elementen
				\item Zwei Dyaden überlappen, wenn sie einen Teilnehmer teilen
				\item Netzwerkdaten sind charakterisiert durch
					\begin{itemize}
						\item Beobachtungseinheiten sind Dyaden
						\item Dyaden überlappen
					\end{itemize}
				\item dyadische Datenanalyse: Dyaden sind unabhängig
				\item Netzwerkanalysis: Dyaden sind abhängig
			\end{itemize}
		\item[Zeitabhängige Daten] (Attribute verändern sich im Laufe der Zeit)
			\begin{description}
				\item[Panel-Daten] Attributwerte aller Elemente für mindestens zwei Zeitpunkte
				\item[Zeitreihendaten] Attributwerte eines einzelnen Elements im Laufe der Zeit
				\item[Querschnittsdaten] Attributwerte aller Elemente zu einem spezifischen Zeitpunkt
				\item[Ereignisdaten] Attributwerte für ein mit einem Zeitstempel versehenen Element
			\end{description}
			Ereignisdaten werden typischerweise in Panel-Daten umgewandelt
	\end{description}
%----------------------------------------------------------------------------------------%
\subsection{Netzwerkrepräsentation}
	\begin{description}
		\item[Netzwerkdarstellung] Repräsentation eines bestimmten Formats, das \glqq wie\grqq\ der Darstellung
		\item[Interaktionsbereich] symetrische Relation $I\subseteq A\times A$
		\item[Netzwerk] \ \\\vspace*{-\baselineskip}
			\begin{itemize}
				\item besteht aus einer Menge von Attributen auf einem Interaktionsbereich und einer (möglicherweise leeren) Menge von Attributen auf der Menge der Elemente
				\item Elemente repräsentieren die Akteure
				\item Attributwerte $x_{ij}$ für $(i,j)\in I$ sind Kanten
				\item Attribute auf $I$ sind Netzwerkattribute
				\item Attribute auf $A$ sind Verhaltensattribute
				\item Repräsentation eines Netzwerks:
					\begin{enumerate}
						\item Graphen
						\item Matrizen
						\item Relationen
					\end{enumerate}
			\end{itemize}
		\item[Zugehörigkeitsbereich] Relation $A\times S$ auf disjunkten Mengen $A$ und $S$
	\end{description}
	\topbreak
	\vspace*{-\baselineskip}
	\begin{description}
		\item[Two-Mode-Netzwerk] \ \\\vspace*{-\baselineskip}
			\begin{itemize}
				\item besteht aus einer Menge von Attributen auf einem Zugehörigkeitsbereich und einer (möglicherweise leeren) Menge von Attributen auf $A$ und $S$
				\item alle Notationen für Netzwerke gelten auch für Two-Mode-Netzwerke
				\item sind bipartit
			\end{itemize}
		\item[One-Mode-Projektionen]\ \\\vspace*{-\baselineskip}
			\begin{itemize}
				\item Interaktionsbereiche $A\times A$, $S\times S$
				\item $X=\mathbb{R}^{\|A\|,\|S\|}$
				\item $XX^T$ und $X^TX$ sind One-Mode-Projektionen
			\end{itemize}
	\end{description}
%----------------------------------------------------------------------------------------%
\subsection{Zeitabhängige Netzwerke}
	\begin{itemize}
		\item Attribute auf Interaktions- und Zugehörigkeitsbereichen verändern sich im Laufe der Zeit
		\item Menge an Attributen auf einem Interaktionsbereich $A\times A$ und (möglicherweise leere) Menge von Attributen auf $A$
		\item alle Attribute sind abhängig von der Zeit
		\item nur Betrachtung von zeitdiskreten Netzwerken
		\item formale Notation für dynamisches Verhalten von Netzwerken
			\begin{itemize}
				\item Einzelattributnetzwerk mit $x:I\rightarrow R$ als Netzwerkattribut
				\item unendliche Sequenz von identischen Kopien von $x$: $(x(t))_{t\in \mathbb{N}}$
				\item alle möglichen $R$-Sequenzen werden Prozess genannt
				\item eine einzelne Sequenz wird Trajektorie genannt
				\item dynamisches $F$ ist eine Mechanismus zum Auswählen von Trajectorien eines Prozesses (iterative Abbildung)
			\end{itemize}
	\end{itemize}