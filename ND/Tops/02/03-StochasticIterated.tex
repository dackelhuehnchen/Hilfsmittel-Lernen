\subtop{Stochastisch iterierte Netzwerkabbildungen}{-1.6}
\begin{itemize}
	\item iterierte Zufallsabbildungen
	\item zwei Zufallsquellen in Dynamiken:
		\begin{itemize}
			\item Fehlen von Informationen auf Parametern oder Werten der Attribute
			\item Benutzen von Simulationen
		\end{itemize}
\end{itemize}
\begin{description}
	\item[Formaler Ansatz] \ \\\vspace*{-\baselineskip}
		\begin{itemize}
			\item $F=\{f_\omega|\omega\in\Omega\}, f_\omega:J\rightarrow J$
			\item $\Omega$ ist ein Wahrscheinlichkeitsraum
			\item $\mu$ ist Wahrscheinlichkeitsverteilung auf $\Omega$
			\item für $x\in J $ wird $\omega$ entsprechend $\mu$ gewählt und zu $f_\omega(x)$ gegangen
			\item Sequenz der Zufallsvariablen
				\[X_0=x_0, X_1=f_{\omega_1}(x_0), X_2=f_{\omega_2}(f_{\omega_1}(x_0))\]
			\item induktiv: $X_n=f_{\omega_n}(X_{n-1})$ mit unabhängigen $\omega$
		\end{itemize}
\end{description}
%----------------------------------------------------------------------------------------- %
\subsection{Markovkette}
	\begin{description}
		\item[homogene Markovkette] endliche Folge von Zufallsvariablen $X_t:\Omega\rightarrow J$ mit endlichem Zustandsraum $J$, falls
			\begin{description}
				\item[Zeithomogen] ($t$ spielt keine Rolle)
					\[\mathbb{P}[X_{t+1}=j|X_t=i]\]
				\item[Überführungsmatrix] ($P=\mathbb{R}^{n\times n}$)
					\[p_{ij}=\mathbb{P}[X_{t+1}=j|X_t=i]\]
			\end{description}
		\item[Verteilung] \ \\\vspace*{-\baselineskip}
			\begin{itemize}
				\item $q^{(t)}=(q_1^{(t)},\dots,q_n^{(t)})$ mit $q_i^{(t)}=\mathbb{P}[X_t=i]$
				\item $q^{(0)}$ ist Initialverteilung
				\item $q^{(t)}=q^{(0)}\cdot P^t$
			\end{itemize}
	\end{description}
%----------------------------------------------------------------------------------------- %
\subsection{Stationäre Verteilung}
	\begin{description}
		\item[stationäre Verteilung] $\pi=\pi\cdot P$
		\item[absorbierender Zustand auf Markovketten] $p_{ij}=0$ für alle $j\neq i$
		\item[transienter Zustand auf Markovketten] $\mathbb{P}[\exists t>0 : X_t=i|X_0=i]<1$
		\item[wiederkehrender Zustand auf Markovketten] $\mathbb{P}[\exists t>0 : X_t=i|X_0=i]=1$
	\end{description}
\topbreak
\vspace*{-2\baselineskip}
	\begin{description}
		\item[nicht reduzierbare Markovkette] $\exists t>0$ sodass $(P^t)_{ij}>0$
		\item[Periode einer Markovkette] $d(i)=ggt\{t<0|(P^t)_{ii}>0\}$
			\begin{enumerate}
				\item falls $d(i)=1$, dann ist $i$ aperiodisch, gilt es für alle $i$, dann ist die Kette aperiodisch
			\end{enumerate}
		\item[ergodische Markovkette] nicht reduzierbar und aperiodisch
		\item[Grenzwert] $\lim\limits_{t\rightarrow\infty} q^{(t)}=\pi$ (unabhängig von $q^{(0)}$), wobei $\pi$ die stationäre Verteilung ist
	\end{description}