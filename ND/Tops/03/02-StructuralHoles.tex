\subtop{Strategische Netzwerkformation mit strukturellen Löchern}{-1.725}
\begin{itemize}
	\item Zugang zu Ressourcen (z.B. Informationen)
	\item durch Verbindungen entstehen Vorteile und Kosten, redundante Verbindungen weniger Vorteile aber gleiche Kosten
	\item strukturelle Löcher sind Regionen in sozialen Strukturen, in denen das Bilden von Verbindungen fehlgeschlagen ist
\end{itemize}
%----------------------------------------------------------------------------------------- %
\subsection{Das Modell}
	\begin{itemize}
		\item strategisches Spiel
		\item $A=\{1,\dots,n\}$ Menge von Agenten
		\item $S=S_1\times\dots\times S_n$ mit $S_i=\mathcal{P}(A\setminus\{i\})$ (die Menge der Nachbarn)
		\item $u=(u_1,\dots,u_n)$ mit $u_i(N_1,\dots,N_n)=\alpha_0\cdot\|L_i\|+\sum\limits_{\{i,j\}\subseteq L_i}\beta(r_{jk})-c\cdot \|N_i\|$, wobei
			\begin{itemize}
				\item $(N_1,\dots,N_n)\in S$
				\item $\alpha_0$ Vorteil einer direkten Verbindung
				\item $L_i=N_i\cup\{j|i\in N_j\}$
				\item ungerichteter Graph $G=(A,E)$ mit $E=\{\{i,j\}|j\in L_i\}$
				\item $r_{jk}$ ist die Anzahl der Pfade mit Länge zwei $(j,\ell,k)$ für $\{j,\ell\},\{\ell,k\}\in E,\ell\neq j,k$
				\item $\beta$ ist eine fallende nicht-negative Funktion
				\item $\beta(r)$ ist der dazwischenliegende Vorteil
			\end{itemize}
	\end{itemize}
%----------------------------------------------------------------------------------------- %
\subsection{Existenz von Gleichgewichten}
	\begin{description}
		\item[Identifikation von Gleichgewichtsgraphen] \ \\\vspace*{-\baselineskip}
			\begin{itemize}
				\item $q=\left\lfloor\frac{n}{k}\right\rfloor$
				\item $n$ ist Anzahl an Knoten
				\item $k$ ist Anzahl von Parteien
				\item $q$ ist Anzahl unabhängiger Mengen der Größe $k$ und eine unabhängige Menge der Größe \mbox{$\ell = n \mod k$}
				\item $G_(n,k)$ ist der vollständige, multipartite Graph mit
					\begin{itemize}
						\item $V= V_1 \cup V_2 \cup \dots \cup V_q \cup V_{q+1}$
						\item $V_i \cap V_j = \emptyset$ für alle $i\neq j$
						\item $\| V_1\|= \dots =\|V_q\|=k$
						\item $\|V_{q+1}\|=\ell$
					\end{itemize}
				\item für alle $i\in V_j$ gilt $N_i=\bigcup\limits_{\ell=1}^{i-1}V_\ell$
			\end{itemize}
	\end{description}
	\begin{itemize}
		\item für einen ungerichteten Graphen mit
			\begin{itemize}
				\item unabhängigen Mengen $I\subseteq V$
				\item $\|I\|=k$
				\item alle Knoten $v\notin I$ sind adjazent zu allen Knoten aus $I$
			\end{itemize}
			gilt
			\begin{itemize}
				\item Veränderung des Nutzens für $v\notin I$ durch Löschen aller Kanten zu $I$ ist
					\[B(n,k)=k(\alpha_0-1)+{k \choose 2}\beta(n-k)\]
			\end{itemize}
	\end{itemize}
\topbreak
	\vspace*{-1.5\baselineskip}
	\begin{itemize}
		\item[] \ \vspace*{-\baselineskip} 
			\begin{itemize}
				\item $B(n,k)\geq 0\Rightarrow v\notin U$ behält alle Kanten zu $I$
				\item $B(n,k) < 0 \Rightarrow v\notin I$ lässt alle Kanten zu $I$ fallen
			\end{itemize}
		\item für alle Mengen von $n$ Agenten gibt es ein $k$, sodass $G_{n,k}$ ein Nashgleichgewicht ist
	\end{itemize}