\subtop{Strategische Netzwerkformation mit strukturellen Löchern}{-1.725}
\begin{itemize}
	\item Zugang zu Ressourcen (z.B. Informationen)
	\item durch Verbindungen entstehen Vorteile und Kosten, redundante Verbindungen weniger Vorteile aber gleiche Kosten
	\item strukturelle Löcher sind Regionen in sozialen Strukturen, in denen das Bilden von Verbindungen fehlgeschlagen ist
\end{itemize}
%----------------------------------------------------------------------------------------- %
\subsection{Das Modell}
	\begin{itemize}
		\item strategisches Spiel
		\item $A=\{1,\dots,n\}$ Menge von Agenten
		\item $S=S_1\times\dots\times S_n$ mit $S_i=\mathcal{P}(A\setminus\{i\})$ (die Menge der Nachbarn)
		\item $u=(u_1,\dots,u_n)$ mit $u_i(N_1,\dots,N_n)=\alpha_0\cdot\|L_i\|+\sum\limits_{\{i,j\}\subseteq L_i}\beta(r_{jk})-c\cdot \|N_i\|$, wobei
			\begin{itemize}
				\item $(N_1,\dots,N_n)\in S$
				\item $\alpha_0$ Vorteil einer direkten Verbindung
				\item $L_i=N_i\cup\{j|i\in N_j\}$
				\item ungerichteter Graph $G=(A,E)$ mit $E=\{\{i,j\}|j\in L_i\}$
				\item $r_{jk}$ ist die Anzahl der Pfade mit Länge zwei $(j,\ell,k)$ für $\{j,\ell\},\{\ell,k\}\in E,\ell\neq j,k$
				\item $\beta$ ist eine fallende nicht-negative Funktion
				\item $\beta(r)$ ist der dazwischenliegende Vorteil
			\end{itemize}
	\end{itemize}
%----------------------------------------------------------------------------------------- %
\subsection{Existenz von Gleichgewichten}