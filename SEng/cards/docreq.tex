\card{Anforderungsspezifikation (Requirement Specification)}{
	"`Eine Spezifikation, die die Anforderungen für ein System oder eine Systemkomponente setzt; \dots typischerweise sind funktionale Anforderungen, Anforderungen an Performance, Schnittstellen und Design enthalten, sowie Entwicklungsstandards."'\\
	"`Eine \textit{Software Requirement Specification} ist ein Dokument, das eine vollständige Spezifikation enthält, was das System tut ohne zu sagen wie."'
	\begin{compactenum}
		\item vertragliche Zustimmung ($\Rightarrow$ Rechtsstreit)
		\item soll $\Rightarrow$ verpflichtend
		\item sollte $\Rightarrow$ erwünscht
		\item Mehrdeutigkeiten sollten verhindert werden
		\item IEEE 830.1998
		\item DIN 69905 Lastenheft/Pflichtenheft
	\end{compactenum}
}

\card{Inhalt eines SRS (DIN-Norm)}{
	\begin{compactenum}
		\item Zielbestimmung (Muss-, Wunsch-, Abgrenzungskriterien)
		\item Produkteinsatz (Anwendungsbereiche, Zielgruppen, Betriebsbedingungen)
		\item Produktübersicht
		\item Produktfunktionen (genau \& detailliert)
		\item Produktdaten (langfristig zu speichernde Daten aus Benutzersicht)
		\item Produktleistungen (Anforderungen an Zeit, Genauigkeit)
		\item Qualitätsanforderungen
		\item Benutzeroberfläche (grundlegende Anforderungen, Zugriffsrechte,\\evtl Mock-Up's)
		\item funktionale Anforderungen
		\item nicht-funktionale Anforderungen (Gesetze, Normen, Sicherheitsanforderungen, Plattformabhängigkeiten)
		\item technische Produktumgebung (Software/Hardware, organisatorische Rahmenbedingungen, Schnittstellen)
	\end{compactenum}
}