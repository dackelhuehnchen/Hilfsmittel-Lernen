\card{Was sind Anforderungen?}{
	\begin{enumerate}
		\item "`A requirement is a condition or capability that must be met or professed by a system component to satisfy a contract, standard, specification or other formally imposed document."' (ANSI/IEEE Standad 729-1983)\\
		Eine Anforderung ist eine Bedingung oder Fähigkeit, die eine Systemkomponente erfüllen muss um einen Vertrag, einen Standard, eine Spezifikation oder eine anderes formales, verlangtes Dokument zu erfüllen/befriedigen.
		\item Anforderungen sind eine Beschreibung der von außen sichtbaren Verhalten, das "`was"'
		\item Anforderungen sind Interaktionen zwischen dem System und dem systemrelevanten Teil der Umwelt
	\end{enumerate}
}

\card{Benutzer- (Auftraggeber-) Anforderungen}{
	Aussagen in natürlicher Sprache mit Diagrammen
}

\card{Systemanforderungen}{
	strukturiertes Dokument, detaillierte Beschreibung der Systemfunktionen, Dienstleistungen und optionalen Einschränkungen; kann Teil des Vertrages zwischen Auftraggeber und Auftragnehmer
}

\card{Typen von Anforderungen}{
	\begin{enumerate}
		\item funktionale
		\item nicht-funktionale
			\begin{enumerate}
				\item Produktanforderungen
				\item Unternehmensanforderungen
				\item externe Anforderungen
			\end{enumerate}
		\item Domainanforderungen
	\end{enumerate}
}

\card{funktionale Anforderungen}{
	\begin{compactenum}
		\item high-level "`was"' das System tut, nicht wie
		\item Aussage darüber, welche Services das System anbieten sollte
		\item Interaktion zwischen System und Umwelt (Systemzustände, I/O)
		\item wie sich das System in besonderen Situationen verhalten sollte
		\item beschreibt den Service des System im Detail
		\item oft beschrieben mit soll/sollte
		\item "`funktionale Anforderungen legen fest, welche Dienste (aus Sicht des Benutzers) das System anbieten soll/welche Aufgaben es erfüllen soll
		\item eindeutig/widerspruchsfrei
		\item zentrale Vorgaben für die Systementwicklung
		\item beschrieben mit Use-Cases
	\end{compactenum}
}

\card{nicht-funktionale Anforderungen: Diagramm}{
	\usetikzlibrary{trees}

\begin{tikzpicture}[
  g/.style={rectangle,draw,fill=green!20,rounded corners=.8ex},
  yel/.style={rectangle,draw,fill=yellow!20,rounded corners=.8ex},
  gr/.style={rectangle,draw,fill=black!10,rounded corners=.8ex},
  b/.style={rectangle,draw,fill=blue!20,rounded corners=.8ex},
  grandchild/.style={grow=down,xshift=1em,anchor=west,
    edge from parent path={(\tikzparentnode.south) |- (\tikzchildnode.west)}},
 ggrandchild/.style={grow=right,xshift=1em,anchor=west,
    edge from parent path={(\tikzparentnode.east) |- (\tikzchildnode.west)}},
  first/.style={level distance=4.5ex},
  second/.style={level distance=9ex},
  third/.style={level distance=13.5ex},
  fourth/.style={level distance=18ex},
  level 1/.style={sibling distance=5em}]
    % Parents
    \coordinate
      child[grow=down,level distance=0ex]{node[b, anchor=west]{nicht-funktional}}
    [edge from parent fork down]
    % Children and grandchildren
    child{node[g,xshift=-1.3cm] {produktspezifische}
      child[grandchild,first] {node[yel]{Nutzbarkeit}}
      child[grandchild,second] {node[yel]{Realisierbarkeit}}
      child[grandchild,third] {node[yel] {Mobilit\"at}}
      child[grandchild,fourth,grow=down] {node[yel] {Effizienz}
		child[grandchild,first,xshift=-0.75cm,anchor=east,edge from parent path={(\tikzparentnode.south) |- (\tikzchildnode.east)}] {node[gr] {Platz}}
		child[grandchild,first] {node[gr] {Performance}}}}
    child{node[g] {externe}
      child[grandchild,first] {node[yel]{ethisch}}
      child[grandchild,second] {node[yel]{Kompatibilit\"at}}
      child[grandchild,third] {node[yel]{rechtliche}
		child[grandchild, first]{node[gr]{Sicherheit}}
		child[grandchild, second]{node[gr]{Privatleben}}}
}
    child {node[g,xshift=1.1cm]{organisatorische }
      child[grandchild,first] {node[yel]{Auslieferung}}
      child[grandchild,second] {node[yel]{Implementation}}
      child[grandchild,third] {node[yel]{Standards}}};
\end{tikzpicture}
}

\card{nicht-funktionale Anforderungen}{
	\begin{compactenum}
		\item falls nicht erfüllt, kann das System unbrauchbar/unbenutzbar sein (können sich auf wichtige Systemeigenschaften beziehen)
		\item nicht primär den Funktionen/Services des Systems zugeordnet, sondern zu Qualität und zusätzlichen Charakteristiken (Quantitativ) / selten an einzelne Systemfunktionen gebunden
		\item oft relevanter als funktionale Anforderungen (Unbedienbarkeit, \dots)
		\item oft allgemein formuliert (kann später zu Problemen führen)
		\item direktes Überprüfen schwer (Tests und Metriken (festlegen))
		\item Einschränkungen der Funktionen/Services (können Beschränkungen definieren, "`sichtbare"' Eigenschaften):
		\begin{compactitem}
			\item Zuverlässigkeit \textit{(Verfügbarkeit, Integrität, Sicherheit)},\\Genauigkeit der Ergebnisse, Performance/Timing,\\ Mensch-Computer-Interface-Themen, körperliche und\\ Betriebseinschränkungen, Übertragbarkeit und Kompati-\\bilität, Antwortzeit, Speicheranforderungen, Standards, \dots
			\item auch: bes. System, Prog.sprache oder Entwicklungsmethode
		\end{compactitem}
	\end{compactenum}
}

\card{nicht-funktionale Anforderungen: Typen}{
	\begin{compactenum}
		\item Produktanforderungen:
			\begin{compactenum}
				\item Geschwindigkeit, Speicherbedarf
				\item akzeptable Fehlerquoten
				\item Transportierbarkeitsanforderungen
				\item Anforderungen bezüglich Benutzbarkeit
			\end{compactenum}
		\item Unternehmensanforderungen:
			\begin{compactenum}
				\item Vorschriften für Entwicklung (bestehende Standards, spezielle Anwendungen)
				\item Umsetzungsanforderungen (Programmiersprache, Entwurfsmethode, Lieferanforderungen)
			\end{compactenum}
		\item externe Anforderungen:
			\begin{compactenum}
				\item Kompatibilität (Ausführbarkeit auf div. Geräten/ Betriebssystemen, Sprachpakete)
				\item rechtliche Anforderungen (Datenschutz, Speicherung)
				\item ethische Anforderungen (Langzeitspeicherung, keine fremden Kundendaten anzeigen, freundliches Design)
			\end{compactenum}
	\end{compactenum}
}

\card{Domainanforderungen}{
	\begin{enumerate}
		\item falls nicht erfüllt, könnte es sein, dass das System nicht in der Domain funktioniert (undurchführbar)
		\item abgeleitet von Applikations-Domain
		\item Charakteristiken/Eigenschaften der Domain
		\item Einschränkungen von existierenden Anforderungen
		\item definieren spezifische Berechnungen
		\item können in einer Domain-spezifischen Sprache ausgedrückt werden $\Rightarrow$ oft schwer zu verstehen
		\item oft implizit $\Rightarrow$ schwer herauszufinden
	\end{enumerate}
}