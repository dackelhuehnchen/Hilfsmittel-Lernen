\card{Was ist ein \textbf{fault}?}{
	Fehler, der von Menschenhand während der Laufzeit der Software getätigt wurde (fehlerhaftes Design, Anforderungen, Coding)\\
	(Anschalten der Kühlung (Atomkraftwerk) nicht korrekt programmiert)
}

\card{Was ist ein \textbf{failure}?}{
	Ein Fehler in der gelieferten Software, die nicht mehr ihren Spezifikationen entspricht (nicht mehr das tut, was zu erwarten wäre).\\
	Bzw.: Die Unfähigkeit eines Systems oder einer Komponente seine geforderten Funktionen innerhalb der Leistungsanforderungen durchzuführen.\\
	(Atomkraftwerk explodiert)
}

\card{Charakteristik von technischer Disziplin (Engineering Discipline)}{
	\begin{enumerate}
		\item gut verstandene Technologien
		\item gut definierte Prozesse
		\item Vorhersagbarkeit eines Ergebnisses eines Prozessstils
		\item Wiederholbarkeit von Prozessschritten
	\end{enumerate}
}

\card{Welche Symptome traten in der Softwarekrise auf?}{
	\begin{enumerate}
		\item Produkte wurden zu spät geliefert (hohe Kosten)
		\item Projekte überschritten ihr Budget (hohe Kosten, Verschwendung von Ressourcen)
		\item Produkte taten nicht das, was sie tun sollten/wozu sie entwickelt wurden (ineffizient, hohe Kosten)
		\item Produkte waren defekt (hohe Kosten (failure, Wartung), ethische Überlegungen)
		\item Projekte wurden beendet, bevor sie abgeschlossen waren (Verschwendung von Ressourcen)
	\end{enumerate}
}

\card{Charakteristik von Software}{
	\begin{enumerate}
		\item Software wird entwickelt nicht hergestellt
		\item Software leiert nicht aus (aber: Veränderung von Anforderungen oder der Umwelt)
		\item Software ist komplex
		\item Software ist ein bestimmender Systemfaktor (bis zu 80\% des Entwicklungsaufwandes)
	\end{enumerate}
}

\card{Warum ist Software schwierig zu entwickeln?}{
	\begin{enumerate}
		\item es gibt kein ähnliches System bisher (Probleme sind unbekannt, Annahmen zur Umgebung können falsch sein)
		\item Anforderungen sind nicht gut/ausreichend verstanden/formuliert
		\item Anforderungen verändern sich im Laufe der Entwicklung des Systems
		\item komplexe Interaktion
		\item Natur des Systems: nebenläufige Systeme (Deadlocks, \dots), eingebettete Systeme (Hardwareinteraktion, Timing, \dots), Informationssysteme (Komplexität, \dots)
		\item Software ist einfach zu verändern ("`code and fix"')
		\item Software ist unauffällig (entweder fällt sie aus, oder nicht)
	\end{enumerate}
}

\card{Software-Entwicklungs-Mythen}{
	\begin{enumerate}
		\item Management:\\
			- Standard-Bücher, -Software, -Tools $\Rightarrow$ Software ist schwierig zu standardisieren\\
			- Dem Zeitplan hinterher? Programmieren einstellen $\Rightarrow$ Bemühung neue Leute einzustellen
		\item Kunde:\\
			- allgemeine Erklärung der Ziele sind ausreichend\\
			- Veränderungen (auch in den Anforderungen) sind einfach zu implementieren
		\item Anwender:\\
			- Sobald das Programm läuft ist der Job erledigt\\
			- Bis das Programm läuft gibt es keine Möglichkeit die Qualität des Systems zu überprüfen\\
			- Nur ein arbeitsfähiges Programm ist lieferbar
	\end{enumerate}
}

\card{Was ist Software Engineering?}{
	\begin{enumerate}
	\item Kommunikation zwischen einer großen Anzahl von Parteien (Kunde, Endbenutzer, Software Designer, Entwickler, \dots) wird aufrechterhalten
	\item Die Basis der Softwareproduktion ist der Entwicklungsansatz.
	\item Design der Software als Prozess, der folgendes unterstützt:
		\begin{enumerate}
			\item Korrektheit und Zuverlässigkeit
			\item Kosteneffizienz
			\item Komplexität des Projektes
			\item Langlebigkeit des Produktes, Lebenszyklus, Veränderungen
			\item Kommunikation unter den Parteien
		\end{enumerate}
	\end{enumerate}
}

\card{Definition des Software Engineering Prozessmodelles}{
	alle Phasen des Software Engineeringprozesses: strikte Anwendung des Prozessmodells (wohldefinierter Input/Output)
}

\card{Built-and-Fix-Modell}{
	\begin{enumerate}
		\item keine Prozessschritte
		\item keine Unterteilung von Bedenken
		\item keine Möglichkeit mit Komplexität umzugehen
	\end{enumerate}
	\usetikzlibrary{positioning,arrows}
\begin{tikzpicture}[every node/.style={draw,rounded corners,text width=1.8cm},node distance=0.5cm, dev/.style={->,>=triangle 60},main/.style={dev,dashed}]
\node (0) at (0,0) {Build first version};
\node (1) [right=of 0] {Modify until client is satisfied};
\node (2) [below right=of 1] {Operations Mode};
\node (3) [above right=of 2] {Retirement};
\node (p1) [above=of 0,draw=none] {\ };
\node (p2) [right=of 1,draw=none,xshift=-1cm,yshift=-0.2cm] {\ };
\node (p3) [below=of 1,draw=none,yshift=0.4cm,xshift=0.3cm] {\ };

\path[dev,draw](0)--(p1.center)-|(1);
\path[dev,draw](1)|-(2);
\path[dev,draw](1)to[out=300,in=90](p3.center)-| (p2.center)to[out=180,in=-10](1);
\path[main,draw](2)|-(1);
\path[dev,draw](2)-|(3);

\draw[->,>=triangle 60] (-1,-2)--(1,-2);
\draw[->,>=triangle 60,dashed] (-1,-2.5)--(1,-2.5);
\node (dev)[anchor=west,draw=none] at (1.2,-2) {Development};
\node (main)[anchor=west,draw=none] at (1.2,-2.5) {Maintenance};
\end{tikzpicture}
}