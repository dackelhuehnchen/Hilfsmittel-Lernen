\card{Analyse mit UML}{
	visuelle Notation für Analyse und Design
}
\card{Statische UML Elemente}{
	\begin{enumerate}
		\item Use Case Diagramme
		\item Klassendiagramme
	\end{enumerate}
}
\card{Dynamische UML Elemente}{
	\begin{enumerate}
		\item Sequenzdiagramme
		\item Zustandsgrafik (State Chart Diagram)
		\item Dynamische Modellierung
	\end{enumerate}
}
\card{Use Case Diagramme (Infos)}{
	\begin{enumerate}
		\item oft der erste Schritt in einem \oo en Entwicklungsprozess
		\item ein Use Case repräsentiert ein Anwendungsszenario (atomar)
		\item dokumentiert die Funktionalität, die das System zur Verfügung steht (was das System tut)
		\item Akteure (etwas oder jemand, der mit dem System interagiert, stellt Input und Output zur Verfügung) und Haupt-Use Cases (Interaktion zwischen Akteur und System)
	\end{enumerate}
}
\card{Use Case Diagramme $<<$include$>>$}{
	ein Use Case benutzt die Funktionalität eines anderen Use Cases (Use Case 1 benutzt Use Case 2)\\\ \\
	\hspace*{0.5cm}\begin{tikzpicture}[]
%\begin{umlsystem}[x=4, fill=red!10]{}
\umlusecase{use case1}
\umlusecase[x=6]{use case2}
%\end{umlsystem} 
%\umlinherit{usecase-2}{usecase-1}
\umlinclude[name=incl]{usecase-1}{usecase-2}

\end{tikzpicture}
}
\card{Use Case Diagramme $<<$extend$>>$}{
	ein Use Case benutzt die Funktionalität eines anderen Use Cases (Use Case 1 benutzt Use Case 2) im Ausnahmefall bzw. optional (nicht zwingend wie bei include)\\\ \\
	\hspace*{0.5cm}\begin{tikzpicture}[]
%\begin{umlsystem}[x=4, fill=red!10]{}
\umlusecase{use case1}
\umlusecase[x=6]{use case2}
%\end{umlsystem} 
%\umlinherit{usecase-2}{usecase-1}
\umlextend[name=extd]{usecase-1}{usecase-2}

\end{tikzpicture}
}
\card{Klassendiagramm ("`Bestandteile"')}{
	\begin{enumerate}
		\item Phenomen (Objekt in der Welt, wie es in der Domäne wahrgenommen wird)
		\item Konzept (Eigenschaften eines Phenomen) ist ein 3-Tupel: Name (unterscheidet sie von anderen), Zweck (Eigenschaft, die bestimmt ob ein Phenomen Teil des Konzeptes ist), Elemente (Phenomene, die Teil des Konzepts sind)
		\item Typ (Abstraktion im Kontext des Programmierens (name: int)), Beispiel: \texttt{SimpleWatch}
		\item Klasse (Abstraktion im Kontext der \oo en Sprachen, kapselt Zustand(Variablen) und Verhalten(Methoden))
		\item Instanz (existierende Instanz einer Klasse),\\Beispiel: \texttt{myWatch:SimpleWatch}
	\end{enumerate}
}
\card{Klassendiagramm (Eigenschaften)}{
	\begin{enumerate}
		\item Allgemeingültigkeit (die am allgemeinsten gehaltenen Aspekte des Problems)
		\item Abstraktion (nicht einzelne Objekte, sondern das was sie gemeinsam haben)
		\item "`Die Beschreibung für eine Menge an Objekten, die die gleichen Attribute, Beziehungen und Verhaltensweisen teilen. Eine Klasse repräsentiert ein Konzept, indem das System modelliert wurde."'
		\item wird während der Anforderungsanalyse benutzt (modellieren von problematischen Domainkonzepten), Systemdesign (modellieren von Untersystemen und Schnittstellen), Objektdesign (modellieren von Klassen in einer Programmiersprache)
	\end{enumerate}
}

\card{Klassendiagramm (Verbindungen unter Klassen)}{
	\begin{compactenum}
		\item Association: eine semantische Bedingung oder Einschränkung als Ausdruck repräsentiert
		\item Abhängigkeiten (Dependencies):
		\item Aggregation
		\item Komposition
		\item Generalisierung
		\item Sichtbarkeit (+ public, - private, \# protected)
	\end{compactenum}
}

\card{Klassendiagramm: Association}{
	\input{pictures/asso.pgf}
}

\card{Klassendiagramm: Dependencies}{
	\input{pictures/dep.pgf}
}

\card{Klassendiagramm: Aggregation}{
	\input{pictures/agg.pgf}
}

\card{Klassendiagramm: Komposition}{
	\input{pictures/comp.pgf}
}

\card{Klassendiagramm: Generalisierung}{
	\input{pictures/gen.pgf}
}

\card{Sequenzdiagramm (Infos)}{
	\begin{enumerate}
		\item wird während der Anforderungsanalyse benutzt (dynamisches Verhalten definieren), Design (dokumentieren der Untersystemschnittstellen), gut für Echtzeitdaten
		\item Bestandteile:
			\begin{compactenum}
				\item Klassen $\Rightarrow$ Säulen
				\item Nachrichten $\Rightarrow$ Pfeile
				\item Rückgabewerte $\Rightarrow$ gestrichelte Pfeile
				\item Bestätigungen/Aktivierungen $\Rightarrow$ schmale Rechtecke
				\item lifelines $\Rightarrow$ gepunktete Lininen
			\end{compactenum}
	\end{enumerate}
}

\card{State Chart Diagramm}{
	Beschreibung einer State Machine (nur für dynamisch interessante Objekte)
}

\card{Dynamische Modellierung}{
	Sammlung von State Chart Diagrammen
}