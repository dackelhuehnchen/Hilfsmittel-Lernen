\card{Wasserfall-Modell (Bild)}{
	\usetikzlibrary{positioning}

\begin{tikzpicture}[every node/.style={draw, fill=black!10}, node distance=0.6cm]

\node(0) at(0,0) {System Design};
\node(1)[below=of 0,anchor=north west]{Requirements};
\node(2)[below=of 1,anchor=north west]{Design};
\node(3)[below=of 2,anchor=north west]{Implementation};
\node(4)[below=of 3,anchor=north west]{Integration};
\node(5)[below=of 4,anchor=north west]{Maintenance};

\foreach \x/\y/\z in {0/1/\ ,1/2/SRS,2/3/Design,3/4/Test,4/5/Test}{
	\draw[->](\x.east)to[out=0,in=20]node[right, draw=none,fill=none]{\scriptsize \z}(\y);
}

\foreach \x/\y/\z in {0/1/what ,1/2/how,2/3/\ ,3/4/\ ,4/5/\ }{
	\draw[->,dotted,color=red](\y)to[out=180,in=240]node[left, draw=none,fill=none]{\scriptsize \z}(\x);
}


\end{tikzpicture}
}

\card{WasserFall-Modell: System Design}{
	Probleme und Systemanforderungen, Realisierbarkeitsstudie, Definieren von Haupt-Subsystemen (zugeordnet zu Hard-/Software), System Design Document (SDD, informell, mit dem Kunden, manchmal: Bedienungsanleitungen, Benutzerschnittstellen, Testpläne)
}

\card{Wasserfall-Modell: Requirements}{
	Anforderungsanalyse ("`Was"' soll das System tun (nicht "`wie"'), Softwareanforderungen beschreiben die beobachtbaren externen Verhalten (funktional, non-funktional)), Software Requirement Specification (SRS)
}

\card{Wasserfall-Modell: Design}{
	"`Wie"' das System arbeitet, architektonisches (high level) Design (zerteilt das Problem in Komponenten, globale Datenstrukturen, interne Schnittstellen), detailliertes Design (algorithmisches Design, interne Datenstrukturen, Programmiersprache(n))
}

\card{Wasserfall-Modell: Implementation (und Testing)}{
	Übersetzung der Designmodule in Code, Testen der Module in Isolation
}

\card{Wasserfall-Modell: Integration (und Testing)}{
	Integrieren der getesteten Module zur Bildung des Systems, \textit{integration testing}, Bestätigung (Validation), Kundenakzeptanztests
}

\card{Wasserfall-Modell: Maintenance (Instandhaltung)}{
	Produktbereitstellung, Wartungsarbeiten (korrekt, adaptiv, perfekt), Projekt abschließen
}

\card{Wasserfall-Modell: Vorteile}{
	\begin{enumerate}
		\item \textbf{Dijkstra}: Definition der einzelnen Aufgaben, (Aufteilung der Bedenken) \\
			Aufteilen von komplexen Designproblemen in kleinere Einheiten $\Rightarrow$ vereinfacht die Teamarbeit/Reproduzierbarkeit
		\item Spezifikation und Dokumentation $\Rightarrow$ erzwingt Dokumentation, vereinfacht das Testen (entgegen der Anforderungsspezifikation)
		\item weitere Konzepte:
			\begin{enumerate}
				\item Verifikation/Validierung (Vergleicht Zwischenergebnisse von Anforderungen und Design)
				\item Prototyping (mock-up (Modell) ist früh verfügbar, reduziert Risiken)
				\item evolutionäres Prozessmodell (Unterbringen von Veränderungen)
			\end{enumerate}
	\end{enumerate}
}

\card{Wasserfall-Modell: Mängel}{
	\begin{enumerate}
		\item keine Rückkopplungsschleifen
		\item dokumentorientiert, unflexibel
		\item großer Zeitabstand zwischen Beginn und Abschluss
	\end{enumerate}
}

\card{Wasserfall-Modell: Alternativen}{
	\begin{enumerate}
		\item modifiziertes Wasserfallmodell
		\item V-Modell, V-Modell XT
		\item evolutionäre Prozess-Modelle
		\item Spiralmodell
		\item Rational Unified Process (RUP)
		\item Agile Prozesse
		\item \dots
	\end{enumerate}
}

\card{V-Modell: Bild}{
	\hspace*{-0.5cm}
	\scalebox{0.75}{\usetikzlibrary{positioning,calc,arrows}

\begin{tikzpicture}[every node/.style={draw,fill=white}, node distance=0.6cm]

\node(1) at (0,0)[anchor=north west,xshift=-.5cm]{Requirements Analysis};
\node(2)[below=of 1,anchor=north west,xshift=-0.25cm]{System Design};
\node(3)[below=of 2,anchor=north west,xshift=-0.15cm]{Program Design};
\node(4)[below right=of 3,anchor=north west,xshift=-0.25cm]{Coding};

\node(31)[right=of 3,anchor=west,xshift=0.6cm]{Unit and Integration Testing};
\node(21)[above=of 31,anchor=south west,xshift=-.6cm]{System Testing};
\node(11)[above=of 21,anchor=south west,xshift=-1cm]{Acceptance Testing};
\node(01)[above=of 11,xshift=.5cm] {Operation and Maintenance};

\foreach \x in {1,2,3}{
	\draw[->, dashed, >=triangle 60](\x1.west)to (\x.east);
}
\foreach \x/\y in {1/2,2/3,3/4}{
	\draw[->, >=triangle 60](\x)-- (\y);
}
\foreach \x/\y in {31/21,21/11,11/01}{
	\draw[->, >=triangle 60](\x.north)-- (\y.south);
}
\draw[->, >=triangle 60](4.east)-- (31.south);
\end{tikzpicture}}
}